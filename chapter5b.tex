\documentclass[./FM_mgr.tex]{subfiles}

\begin{document}
\section{Badania} \label{chapter:research}

%W dalszej części pracy konkretny zestaw parametrów będzie nazywany konfiguracją.
%Wyrażenie ,,wynik eksperymentu dla konfiguracji wyniósł \emph{X}'' oznacza, że metaheurystyka została uruchomiona z parametrami (rozmiarem populacji, prawdopodobieństwami, itd.) określonymi przez konfigurację, a jej wynikiem było \emph{X}.
%Należy pamiętać, że algorytm ewolucyjny to metaheurystyka losowa, więc wynik dla danej konfiguracji za każdym uruchomieniem może być inny.

Celem przeprowadzonych badań było porównanie skuteczności działania standardowego algorytmu ewolucyjnego, wybranych metaheurystyk uwzględniających płeć oraz heuerystyki DSEA z różnymi operatorami selekcji płciowej.
Przez skuteczność działania rozumiano wartość funkcji oceny, uśrednioną między wieloma uruchomieniami algorytmu dla tych samych parametrów.

Aby móc tego dokonać wybrano dwa opisane wcześniej problemy, zdefiniowane jako problemy minimalizacji. 
Dla każdego z nich przeprowadzono eksperymenty podzielone na 2 etapy.
Pierwszy z nich miał na celu określenie wartości parametrów zwykłego algorytmu ewolucyjnego dla danego problemu.
Drugi z etapów wykorzystywał parametry znalezione wcześniej do zbadania różnych operatorów selekcji naturalnej i płciowej.

\subsection{Badania wrażliwości metahuerystyki na wartość parametrów}
\label{subsection:init_params}

\subsubsection*{Cel}
Celem tego zestawu badań było zbadanie związku między wartościami użytych parametrów, a wynikami metaheurystyki oraz określenie jak najlepszego zestawu parametrów dla każdego z problemów.

\subsubsection*{Założenia}
Każdy eksperyment na każdym etapie badań wykonano niezależnie dla obu problemów. 

Aby to osiągnąć założono pewien zestaw wartości parametrów, pokazany w tabeli \ref{table:init_params}.
Ponieważ łączna liczba ich możliwych kombinacji była tak duża, że praktycznie uniemożliwiała zbadanie ich wszystkich zdecydowano się wykorzystać iteracyjne, adaptacyjne podejście do dostrajania heurystyki.
Wykorzystaną procedurę nazwano \emph{procedurą eksploracji}, lub krócej \emph{eksploracją}.
Została ona szczegółowo opisana w dodatku \ref{appendix:explore}.
Polega ona na wielokrotnym powtórzeniu iteracji nazwanej nawrotem.
W ramach jednego nawrotu każdy parametr był dobierany osobno tak, aby heurystyka dawała jak najlepsze wyniki.
Polegało to na ,,zamrożeniu'' wartości pozostałych parametrów i wielokrotnym uruchomieniu heurystyki z różną wartością badanego parametru.
Kiedy określono najlepszą wartość, zaczynano badać wartość kolejnego z nich.
Aby zastosować taką procedurę należało określić (prócz samych wartości parametrów) ilość nawrotów, ilość powtórzeń dla każdego zestawu parametrów oraz sposób oceny tego który zestaw jest lepszy.

\subsubsection*{Przebieg}
Cały eksperyment był dodatkowo podzielony na 2 kroki. 
Szczegółowy zapis ich przebiegu zamieszczony jest w dodatku \ref{appendix:explore_flow}
W pierwszym kroku wykorzystywano procedurę eksploracji do badania zakresów parametrów przedstawione w tabeli \ref{table:init_params}.
Wykonano tu trzy nawroty, a w każdym z nich każdy zestaw parametrów był używany w pięciu uruchomieniach heurystyki.
Przez najlepsze wyniki działania heurystyki rozumiano tu najniższy wynik dla danego zestawu parametrów spośród pięciu powtórzeń.

Drugi krok również działał w oparciu o eksplorację, jednak badane zakresy  zależały od wyników kroku poprzedniego.
W tej części badań zmieniano jedynie wartości liczbowe, próbkując lokalne sąsiedztwo wartości danego parametru.
Jeśli przyjmował on wartości z przedziału $\range{0}{1}$, to wynik pierwszego kroku zmieniano o $\pm0,02$ oraz $\pm0,05$ (odrzucając wartości mniejsze od 0 i większe od 1).
Jeśli miał on jednak wartości całkowite, to dotychczasowy wynik zmieniano o $\pm5$ i $\pm10$.
W ten sposób określano założenia dotyczące wartości parametrów w drugim kroku.
Wykonano dwa nawroty, w każdym po trzy uruchomienia heurystyki, przez najlepszy wynik rozumiejąc najniższą średnią ocenę z trzech uruchomień.

\begin{table}
	\caption{Założenia dot. wartości parametrów \label{table:init_params}}
	\centering
	\begin{tabular}{|c|l|}
		\hline
		\textbf{Parametr} & \multicolumn{1}{c|}{\textbf{Zbiór wartości}} \\
		\hline
		\hline
		$\param{rozmiarPopulacji}$ & 10, 20, 50 \\
		\hline
		$\param{wspDomieszek}$ & 0, 0,1, 0,25, 0,5\\
		\hline
		$(\param{prawdKrzyzowania}, \param{prawdMutacji})$ & [0,6, 0,7, 0,8] $\times$ [0,1, 0,2, 0,3]  \\
		\hline
		$\param{max}$ & 25, 50, 100 \\
		\hline		
		$\param{opSelNat}$ & \opName{natSel}(RS), \opName{natSel}(TS(2)), \opName{natSel}(TS(3))\\
		\hline
		\multicolumn{2}{c}{}\\
		\hline
		\textbf{Parametr} & \multicolumn{1}{c|}{\textbf{Wartość}} \\
		\hline
		\hline
		$\param{opSelPlciowej}$ & \opName{stdGenSel}($\bot$, R, R)\\
		\hline
%		$R$ & 3\\
%		\hline
%		$I$ & 5\\
%		\hline
%		$statystyka$ & najlepszy wynik \\
%		\hline
		Zbiór płci  & $\important{G}=\{ \odot \}$ \\
		\hline
	\end{tabular}
\end{table}

\subsubsection*{Wyniki}
Tabele \ref{table:tsp_base} i \ref{table:knapsack_base} opisują znalezione wyniki parametrów kolejno dla problemu komiwojażera i problemu plecakowego.
Przebieg obu kroków eksploracji został szczegółowo opisany w dodatku \ref{appendix:explore_flow}.

Warto zauważyc, że znaleziono duże (tzn. jedne z największych badanych) rozmiary populacji i liczby pokoleń.
Mniejszą liczbę pokoleń określono dla problemu plecakowego, jednak dla niego wykorzystano wartość współczynnika domieszek 0,1 i największy badany rozmiar populacji.
Dodatkowo, dla problemu plecakowego określono wyższe prawdopodobieństwo krzyżowania.

Może to wskazywać na fakt, iż do rozwiązania problemu komiwojażera potrzebna jest słabsza eksploatacja przestrzeni rozwiązań, niż dla problemu plecakowego.
Innymi słowy istnieje podejrzenie, że przestrzeń rozwiązań problemu komiwojażera można próbkować rzadziej niż przestrzeń problemu plecakowego.

\begin{table}
	\caption{Parametry używane w dalszych badaniach problemu komiwojażera \label{table:tsp_base}}
	\centering
	\begin{tabular}{|l|l|}
		\hline
		\textbf{Parametr} & \multicolumn{1}{c|}{\textbf{Wartość}} \\
		\hline
		\hline
		$\param{opSelNat}$ & \opName{natSel}(RS)\\
		\hline
		$\param{opSelPlciowej}$ & \opName{stdGenSel}($\bot$, R, R)\\
		\hline
		$\param{rozmiarPopulacji}$                         & 50                      \\ \hline 
		$\param{wspDomieszek}$                             & 0,0                     \\ \hline
		$\param{prawdKrzyzowania}$ & 0,6 \\ \hline 
		$\param{prawdMutacji}$ & 0,15             \\ \hline
		$\param{max}$                                      & 110                     \\ \hline
	\end{tabular}
\end{table}

\begin{table}
	\caption{Parametry używane w dalszych badaniach problemu plecakowego \label{table:knapsack_base}}
	\centering
	\begin{tabular}{|l|l|}
		\hline
		\textbf{Parametr} & \multicolumn{1}{c|}{\textbf{Wartość}} \\
		\hline
		\hline
		$\param{opSelNat}$ & \opName{natSel}(RS)\\
		\hline
		$\param{opSelPlciowej}$ & \opName{stdGenSel}($\bot$, R, R)\\
		\hline
		$\param{rozmiarPopulacji}$                         & 60                      \\ \hline 
		$\param{wspDomieszek}$                             & 0,1                    \\ \hline
		$\param{prawdKrzyzowania}$ & 0,75 \\ \hline 
		$\param{prawdMutacji}$ & 0,15             \\ \hline
		$\param{max}$                                      & 60                     \\ \hline
	\end{tabular}
\end{table}

\subsection{Badanie efektywności metaheurystyki w zależności od operatorów selekcji} \label{subsection:compare}

\subsubsection*{Cel}
Celem badań przeprowadzonych na tym etapie było porównanie efektywności wybranych metod optymalizacji.
Były to: klasyczny algorytm ewolucyjny, metaheurystyka GGA, metaheurystyka SexualGA oraz algorytm DSEA.
Miarą efektywności była średnia najniższa ocena z pięciu powtórzeń działania heurystyki.

\subsubsection*{Założenia} 
Każdy eksperyment był powtórzony pięciokrotnie dla każdego z problemów oddzielnie, dla każdego zestawu parametrów.

W tabelach \ref{table:ea}-\ref{table:dsea_harem} przedstawiono zestawy parametrów wykorzystywane w wykonanych eksperymentach. 
Jeżeli wartość jakiegoś z parametrów nie jest w nich określona, wykorzystano wartość podaną w tabelach \ref{table:tsp_base} i \ref{table:knapsack_base}, w zależności od problemu.

\begin{table}
	\caption{Założenia dot. wartości parametrów standardowego algorytmu ewolucyjnego \label{table:ea}}
	\centering
	\begin{tabu} to \textwidth {lX}
		\hline
		\multicolumn{1}{|l|}{\textbf{Operator selekcji płciowej}} &
		\multicolumn{1}{l|}{\opName{stdGenSel}($\bot$, R, R)} \\ 
		\hline
		\multicolumn{1}{|l|}{\textbf{Pozostałe parametry}} & 
		\multicolumn{1}{l|}{Zgodne z tabelą \ref{table:tsp_base} lub \ref{table:knapsack_base}} \\ 
		\hline
		%		\multicolumn{1}{|l|}{\textbf{Ilość powtórzeń}} &
		%		\multicolumn{1}{l|}{5} \\ 
		%		\hline
		\multicolumn{1}{|l|}{\textbf{Zbiór płci}} & 
		\multicolumn{1}{l|}{$\important{G} = \{ \odot \}$} \\ 
		\hline
	\end{tabu}
\end{table}

Tabela \ref{table:ea} opisuje parametry wykorzystane w badaniach klasycznego algorytmu ewolucyjnego. 
Wykorzystano całkowicie losowy operator selekcji płciowej i jednoelementowy zbiór płci, aby wyeliminować jakikolwiek wpływ tej cechy na wyniki algorytmu. 

\begin{table}
	\caption{Założenia dot. wartości parametrów metaheurystyki GGA \label{table:gga}}
	\begin{tabularx}{\linewidth}{lX}
		\hline
		\multicolumn{1}{|l|}{\textbf{Operator selekcji płciowej}} & \multicolumn{1}{l|}{\opName{stdGenSel}($\top$, R, R)} \\ 
		\hline
		\multicolumn{1}{|l|}{\textbf{Operator selekcji naturalnej}} & \multicolumn{1}{l|}{\opName{natSel}($W$)} \\ 
		\hline
		\multicolumn{1}{|l|}{\textbf{Pozostałe parametry}} & 
		\multicolumn{1}{l|}{Zgodne z tabelą \ref{table:tsp_base} lub \ref{table:knapsack_base}} \\ 
		\hline
		%		\multicolumn{1}{|l|}{\textbf{Ilość powtórzeń}} & \multicolumn{1}{l|}{5} \\ 
		%		\hline
		\multicolumn{1}{|l|}{\textbf{Zbiór płci}} & 
		\multicolumn{1}{l|}{$\important{G} = \{ \male, \female \}$} \\ 
		\hline
		& \\
		\hline
		\multicolumn{1}{|c|}{\textbf{Parametr}} & 
		\multicolumn{1}{c|}{\textbf{Zbiór wartości}} \\ 
		\hline \hline
		\multicolumn{1}{|l|}{$W$} & 
		\multicolumn{1}{l|}{RS, TS(2), TS(3)} \\ 
		\hline
	\end{tabularx}
\end{table}

Tabela \ref{table:gga} pokazuje wartości użyte w badaniach heurystyki GGA. 
Operator selekcji płciowej wybierał rodziców losowo, każdego z inną płcią. 
Zbadano efektywność dla różnych operatorów selekcji naturalnej.

\begin{table}
	\caption{Założenia dot. wartości parametrów metametaheurystyki SexualGA \label{table:sexual_ga}}
	\centering
	\begin{tabularx}{\linewidth}{lX}
		\hline
		\multicolumn{1}{|l|}{\textbf{Operator selekcji płciowej}} &
		\multicolumn{1}{l|}{\opName{stdGenSel}($\bot$, $\param{opWyboru1}$, $\param{opWyboru2}$)} \\ 
		\hline
		\multicolumn{1}{|l|}{\textbf{Operator selekcji naturalnej}} &
		\multicolumn{1}{l|}{\opName{natSel}(R)} \\ 
		\hline
		\multicolumn{1}{|l|}{\textbf{Pozostałe parametry}} & 
		\multicolumn{1}{l|}{Zgodne z tabelą \ref{table:tsp_base} lub \ref{table:knapsack_base}} \\ 
		\hline
		%		\multicolumn{1}{|l|}{\textbf{Ilość powtórzeń}} &
		%		\multicolumn{1}{l|}{5} \\ 
		%		\hline
		\multicolumn{1}{|l|}{\textbf{Zbiór płci}} & 
		\multicolumn{1}{l|}{$\important{G} = \{ \odot \}$} \\ 
		\hline
		& \\ 
		\hline
		\multicolumn{1}{|c|}{\textbf{Parametr}} & 
		\multicolumn{1}{c|}{\textbf{Zbiór wartości}} \\ 
		\hline \hline
		\multicolumn{1}{|l|}{$\param{opWyboru1}$} & 
		\multicolumn{1}{l|}{RS, TS(2), TS(3)} \\ 
		\hline
		\multicolumn{1}{|l|}{$\param{opWyboru2}$} & 
		\multicolumn{1}{l|}{R, RS, TS(2), TS(3)} \\ 
		\hline
	\end{tabularx}
\end{table}

Tabela \ref{table:sexual_ga} mówi o parametrach heurystyki SexualGA.
Wykorzystano tu losowy operator selekcji naturalnej, przez co przeszukiwaniem przestrzeni rozwiąząń sterował jedynie operator selekcji płciowej.
Zbadano 12 zestawów parametrów tego operatora, nie rozróżniając osobników po płci, jednak każdego rodziców wybierając w inny sposób.
Zbadano również sytuacje, w których oboje rodziców jest wybierane tym samym operatorem.
Pominięto taką, w której oba te operatory wyboru są losowe (ponieważ wtedy heurystyka zachowywałaby się zupełnie losowo).

\begin{table}
	%gender, true
	\caption{Założenia dot. wartości parametrów  metaheurystyki DSEA z uogólnionym operatorem selekcji płciowej \label{table:dsea_general}}
	\centering
	\begin{tabularx}{\linewidth}{lX}
		\hline
		\multicolumn{1}{|l|}{\textbf{Operator selekcji płciowej}} &
		\multicolumn{1}{l|}{\opName{stdGenSel}($p$, $\param{opWyboru1}$, $\param{opWyboru2}$)} \\ 
		\hline
		\multicolumn{1}{|l|}{\textbf{Pozostałe parametry}} &
		\multicolumn{1}{l|}{Zgodne z tabelą \ref{table:tsp_base} lub \ref{table:knapsack_base}} \\ 
		\hline
		%		\multicolumn{1}{|l|}{\textbf{Ilość powtórzeń}} &
		%		\multicolumn{1}{l|}{5} \\ 
		%		\hline
		& \\ 
		\hline
		\multicolumn{1}{|c|}{\textbf{Parametr}} & 
		\multicolumn{1}{c|}{\textbf{Zbiór wartości}} \\ 
		\hline \hline
		\multicolumn{1}{|l|}{$p = \param{plecMaZnaczenie}$} & 
		\multicolumn{1}{l|}{$\top, \bot$} \\
		\hline
		\multicolumn{1}{|l|}{$\param{opWyboru1}$} & 
		\multicolumn{1}{l|}{RS, TS(2), TS(3)} \\ 
		\hline
		\multicolumn{1}{|l|}{$\param{opWyboru2}$} & 
		\multicolumn{1}{l|}{R, RS, TS(2), TS(3)} \\
		\hline
		\multicolumn{1}{|l|}{\textbf{Zbiór płci}} & 
		\multicolumn{1}{l|}{ $|\important{G}| \in \{1, 2\}$, zależnie od $\param{plecMaZnaczenie}$ } \\ 
		\hline
	\end{tabularx}
\end{table}

W tabeli \ref{table:dsea_general} przedstawiono założenia dotyczące wartości parametrów dla eksperymentów.
Wykorzystano konfigurację przedstawioną w tabelach \ref{table:tsp_base} i \ref{table:knapsack_base}, zmieniając jedynie operator selekcji płciowej.
Zbadano każdą kombinację operatorów wyboru (z pominięciem dwóch operatorów losowych) dla obu wartości $\param{plecMaZnaczenie}$, co daje łącznie 24 zestawy parametrów.

\begin{table}
	%harem
	\caption{Założenia dot. wartości parametrów  metaheurystyki DSEA z haremowym operatorem selekcji płciowej \label{table:dsea_harem}}
	\begin{tabularx}{\linewidth}{lX}
		\hline
		\multicolumn{1}{|l|}{\textbf{Operator selekcji płciowej}} &
		\multicolumn{1}{l|}{\opName{harem}(a, b, WA, WB, WP)} \\ 
		\hline
		\multicolumn{1}{|l|}{\textbf{Pozostałe parametry}} &
		\multicolumn{1}{l|}{Zgodne z tabelą \ref{table:tsp_base} lub \ref{table:knapsack_base}} \\ 
		\hline
		%		\multicolumn{1}{|l|}{\textbf{Ilość powtórzeń}} &
		%		\multicolumn{1}{l|}{5} \\ 
		%		\hline
		\multicolumn{1}{|l|}{\textbf{Zbiór płci}} & 
		\multicolumn{1}{l|}{$\important{G} = \{ \male, \female \}$} \\ 
		\hline
		& \\ 
		\hline
		\multicolumn{1}{|c|}{\textbf{Parametr}} & 
		\multicolumn{1}{c|}{\textbf{Zbiór wartości}} \\ 
		\hline \hline
		\multicolumn{1}{|l|}{a = $\param{liczbaAlfa}$} & 
		\multicolumn{1}{l|}{1, 3, 5} \\ 
		\hline
		\multicolumn{1}{|l|}{b = $\param{wspBeta}$} & 
		\multicolumn{1}{l|}{0, 0,1, 0,25} \\ 
		\hline
		\multicolumn{1}{|l|}{WA = $\param{opWyboruAlfa}$} & 
		\multicolumn{1}{l|}{R, RS, TS(2), TS(3)} \\ 
		\hline
		\multicolumn{1}{|l|}{WB = $\param{opWyboruBeta}$} & 
		\multicolumn{1}{l|}{R, RS, TS(2), TS(3)} \\ 
		\hline
		& \\ 
		\hline
		\multicolumn{1}{|c|}{\textbf{Parametr}} & 
		\multicolumn{1}{c|}{\textbf{Wartość}} \\ 
		\hline \hline
		\multicolumn{1}{|l|}{WP = $\param{opWyboruPartnerow}$} &
		\multicolumn{1}{l|}{R} \\ 
		\hline
	\end{tabularx}
\end{table}

Tabela \ref{table:dsea_harem} przedstawia zestawy parametrów wykorzystane do zbadania heurystyki DSEA z haremowym operatorem selekcji płciowej.
Wykorzystano parametry znalezione na poprzednim etapie badań, jednak zbadano różne kombinacje odpowiednich parametrów liczbowych ($\param{liczbaAlfa}$ oraz $\param{wspBeta}$) i operatorów wyboru ($\param{opWyboruAlfa}$ oraz $\param{opWyboruBeta}$).
Operator wyboru partnerów przyjmował stałą wartość operatora losowego.

%Nie pominięto sytuacji w której wszystkie operatory były losowe, aby zbadać, czy sama strategia bardzo drastycznej selekcji jednej z płci ma znaczenie. \todo{chcemy na pewno?}

Zbadano wszystkie kombinacje zestawów podanych w kolumnie \textbf{Zbiór wartości}.
Było ich łącznie 144.

W sumie, na tym etapie wykonano pięć uruchomień heurystyki dla każdego z 181 zestawów parametrów, dla każdego z problemów osobno.

\subsubsection*{Wyniki}

Rezultaty przeprowadzonych eksperymentów zostaną przedstawione osobno dla każdego z problemów.

\paragraph{Problem komiwojażera}

W tabeli \ref{table:tsp_results_summary} i na wykresie \ref{figure:tsp_results_summary} przedstawiono zestawienie najlepszych wyników różnych heurystyk.
Heurystyce SexualGA poświęcono dwa wiersze, pokazujące dwa najlepsze zestawy parametrów, które jednocześnie pokazują sytuacje w której oba operatory wyboru są takie same lub różne.
Dla algorytmu DSEA przeznaczono 6 wierszy. 
Pierwsze 4 to najlepsze wyniki dla uogólnionego operatora selekcji płciowej.
Ponadto są to sytuacje zarówno z dwoma różnymi i dwoma takimi samymi operatorami wyboru, jak i z różnym podejściem do płci rodziców.
Ostatnie dwa wiersze poświęcone są na najlepsze wyniki dla operatora haremowego.
Równocześnie są to zestawy parametrów z dwoma różnymi lub takimi samymi operatorami wyboru alfa i beta.

\newpage

\begin{table}[H]
	\centering
	\caption{Najlepsze wyniki różnych heurystyk dla problemu komiwojażera, pokazane na wykresie \ref{figure:tsp_results_summary} \label{table:tsp_results_summary}}
	\begin{tabular}{|l|l|l|l|r@{$\pm$}l|}
		\hline
		\multicolumn{1}{|c|}{\multirow{2}{*}{{\bf Nr.}}} &\multicolumn{1}{c|}{\multirow{2}{*}{{\bf Heurystyka}}} & \multicolumn{1}{c|}{{\bf Op. selekcji}} & \multicolumn{1}{c|}{{\bf Op. selekcji}} & \multicolumn{2}{c|}{\multirow{2}{*}{{\bf Ocena}}} \\
		& \multicolumn{1}{c|}{}                                  & \multicolumn{1}{c|}{{\bf naturalnej}}          & \multicolumn{1}{c|}{{\bf płciowej}}        & \multicolumn{2}{c|}{}                             \\ \hline \hline
		1 & Alg. ewolucyjny                    & opNatSel(RS)                                          & stdGenSel($\bot$, R, R)                                 & 37830,7834      & 1865,7587      \\ \hline
		2 & GGA                                    & opNatSel(RS)                                          & stdGenSel($\top$, R, R)                                 & 68582,5919      & 2240,5888      \\ \hline
		3 & \multirow{2}{*}{SexualGA}              & \multirow{2}{*}{opNatSel(R)}                          & stdGenSel($\bot$, RS, RS)                               & 65454,5598      & 4541,7191      \\ \cline{1-1}\cline{4-6} 
		4 & &                                                       & stdGenSel($\bot$, RS, R)                                & 72023,5817      & 5135,7596      \\ \hline
		5 & \multirow{6}{*}{DSEA}                  & \multirow{6}{*}{opNatSel(RS)}                         & stdGenSel($\bot$, RS, RS)                               & 40199,6226      & 3708,8267      \\ \cline{1-1}\cline{4-6} 
		6 & &                                                       & stdGenSel($\bot$, RS, R)                                & 45328,6013      & 967,5844       \\ \cline{1-1}\cline{4-6} 
		7 & &                                                       & stdGenSel($\top$, RS, RS)                               & 41555,6377      & 3146,2442      \\ \cline{1-1}\cline{4-6} 
		8 & &                                                       & stdGenSel($\top$, RS, R)                                & 46760,4675      & 4193,2035      \\ \cline{1-1}\cline{4-6} 
		9 & &                                                       & harem(5, 0,1, RS, RS, R)                                & 49180,7887      & 4249,2504      \\ \cline{1-1}\cline{4-6} 
		10 & &                                                       & harem(5, 0,1, RS, R, R)                                 & 48915,0643      & 799,6703       \\ \hline
	\end{tabular}
\end{table}

%\begin{table}[h]
%	\centering
%	\caption{My caption}
%	\label{my-label}
%	\begin{tabular}{|l|l|l|r@{$\pm$}l|}
%		\hline
%		\multicolumn{1}{|c|}{\multirow{2}{*}{{\bf Heurystyka}}} & \multicolumn{1}{c|}{{\bf Op. selekcji}} & \multicolumn{1}{c|}{{\bf Op. selekcji}} & \multicolumn{2}{c|}{\multirow{2}{*}{{\bf Ocena}}} \\
%		\multicolumn{1}{|c|}{}                                  & \multicolumn{1}{c|}{{\bf naturalnej}}          & \multicolumn{1}{c|}{{\bf płciowej}}        & \multicolumn{2}{c|}{}                             \\ \hline \hline
%		Alg. ewolucyjny                    & opNatSel(RS)                                          & stdGenSel($\bot$, R, R)                                 & 37830,7834      & 1865,7587      \\ \hline
%		GGA                                    & opNatSel(RS)                                          & stdGenSel($\top$, R, R)                                 & 68582,5919      & 2240,5888      \\ \hline
%		\multirow{2}{*}{SexualGA}              & \multirow{2}{*}{opNatSel(R)}                          & stdGenSel($\bot$, RS, RS)                               & 65454,5598      & 4541,7191      \\ \cline{3-5} 
%		&                                                       & stdGenSel($\bot$, RS, R)                                & 72023,5817      & 5135,7596      \\ \hline
%		\multirow{6}{*}{DSEA}                  & \multirow{6}{*}{opNatSel(RS)}                         & stdGenSel($\bot$, RS, RS)                               & 40199,6226      & 3708,8267      \\ \cline{3-5} 
%		&                                                       & stdGenSel($\bot$, RS, R)                                & 45328,6013      & 967,5844       \\ \cline{3-5} 
%		&                                                       & stdGenSel($\top$, RS, RS)                               & 41555,6377      & 3146,2442      \\ \cline{3-5} 
%		&                                                       & stdGenSel($\top$, RS, R)                                & 46760,4675      & 4193,2035      \\ \cline{3-5} 
%		&                                                       & harem(5, 0,1, RS, RS, R)                                & 49180,7887      & 4249,2504      \\ \cline{3-5} 
%		&                                                       & harem(5, 0,1, RS, R, R)                                 & 48915,0643      & 799,6703       \\ \hline
%	\end{tabular}
%\end{table}

\begin{figure}[H]
	\centering
	\graph{tsp_results_summary.tex}
	\caption{Najlepsze wyniki różnych heurystyk dla problemu komiwojażera, przedstawione w tabeli \ref{figure:tsp_results_summary}. Niższe słupki oznaczają lepsze wyniki. \label{figure:tsp_results_summary}}
\end{figure}

Pełne zestawienie wyników można znaleźć w dodatku \ref{appendix:full_results}.


Najefektywniejszy okazał się być standardowy algorytm ewolucyjny.
Niewiele gorsze wyniki osiągnął algorytm DSEA z uogólnionym operatorem selekcji płciowej, korzystający z tego samego operatora wyboru dla obu płci, co operator selekcji naturalnej, tzn. operatora ruletkowego.
Nieco gorzej zadziałała heurystyka DSEA korzystająca z uogólnionego operatora selekcji płciowej używającego operatorów ruletkowego i losowego. Jeszcze niższe wyniki osiągnęła heurystyka DSEA z haremowym operatorem selekcji płciowej. Najgorzej poradziły sobie heurystyki SexualGA i GGA.

Żadna ze zbadanych heurystyk nie przyniosła lepszych wyników niż standardowy algorytm ewolucyjny. 
Heurystyka DSEA okazała się lepsza od istniejących heurystyk GGA i SexualGA. 
Druga z nich dała lepsze wyniki niż GGA jeżeli używała dwóch takich samych operatorów wyboru, jednak po zastąpieniu jednego z nich operatorem losowym otrzymano gorsze wyniki.

\begin{table}[H]
	\caption{Wyniki metaheurystyki DSEA z operatorem haremowym \label{table:tsp_results_dsea_harem}}
	\centering
	\begin{tabular}{|l|l|l|l|r@{$\pm$}l|}
		\hline
		\multicolumn{1}{|c|}{{\bf a}} & \multicolumn{1}{|c|}{{\bf b}} & \multicolumn{1}{|c|}{{\bf WA}} & \multicolumn{1}{c|}{{\bf WB}} & \multicolumn{2}{c|}{{\bf Ocena}} \\ \hline \hline
		\insertData{tsp_d_top}
	\end{tabular}	
\end{table}

Można zaobserwować, że heurystyka działała najskuteczniej, kiedy jeden lub oba operatory wyboru były takie same jak użyty w operatorze selekcji naturalnej.
Ponadto, najlepsze wyniki uzyskano dla większych liczb osobników alfa i beta.

Warto zauważyć, że najlepszy i najbardziej stabilny wynik uzyskano wybierając osobniki alfa operatorem ruletkowym, a osobniki beta i partnerów losowym.
Może to oznaczać, że drastyczna selekcja rodziców nie jest nieuzasadniona, a osobniki alfa mają najbardziej korzystne cechy.

Wykres \ref{figure:tsp_male_avg} przedstawia średnią funkcję oceny dla różnych liczb samców.
Liczba ta została obliczona jako $\lceil \param{liczbaAlfa} + \param{wspBeta} \times 50 \rceil$, ponieważ $\param{rozmiarPopulacji}$ wynosił 50.
Nie udało się zaobserwować żadnej jednoznacznej zależności między tą liczbą, a wynikami, z czego można wnioskować, że to operatory wyboru (losowy lub ruletkowy w większości przypadków) decydowały o skuteczności.

\newpage

\begin{figure}[H]
	\caption{Wykres średniej oceny od łącznej liczby osobników alfa i beta \label{figure:tsp_male_avg}}
	\centering
	\graph{tsp_male_avg.tex}
\end{figure}

\paragraph{Problem plecakowy}

W tabeli \ref{table:knapsack_results_summary} i na wykresie \ref{figure:knapsack_results_summary} przedstawiono podumowanie wyników różnych heurystyk. 
W wierszach znajdują się wyniki dla odpowiednich parametrow, pokrywające się znaczeniem z tymi z tabeli \ref{table:tsp_results_summary} przedstawiającej wyniki dla problemu komiwojażera.
Co ciekawe, w niektórych sytuacjach (wiersz 4. i wiersz 10.) tam, gdzie dla problemu plecakowego użyto losowego operatora wyboru zastosowano tym razem operator turniejowy (o rozmiarze turnieju równym 2).

Najlepsze wyniki udało się uzyskać przy pomocy heurystyki DSEA. 
Najniższy z nich znaleziono korzystając z uogólnionego operatora wyboru nie wymuszającego różnych płci rodziców i dwukrotnie korzystającego z ruletkowego operatora wyboru, używanego też przez operator selekcji naturalnej.
Różnice między wynikami DSEA są jednak zbyt małe, żeby uważać to konkretne rozwiązanie za znacząco lepsze.

Pozostałe metody sprawdziły się w ogólności gorzej.
Warto zauważyć, że jedynie SexualGA korzystające z tego samego operatora selekcji płciowej co DSEA w najlepszym przypadku osiągnęło lepsze rezultaty niż standardowy algorytm ewolucyjny.
Mimo to, w tym przypadku działanie heurystyki było o wiele mniej stabilne, ponieważ odchylenie standardowe było ponad 6,6 raza większe niż dla standardowego algorytmu ewolucyjnego.

Można podejrzewać, że w tym przypadku selekcja naturalna była ,,za słaba'' i za wolno odrzucała mało przydatne rozwiązania.
Po dodaniu dodatkowego członu, dodatkowo zmniejszającego szanse słabo dopasowanych osobników na wydanie potomstwa, efektywność wydawała się wzrosnąć.
Takiego efektu nie zaobserwowaliśmy dla SexualGA ponieważ tam eksplorację przestrzeni rozwiązań zakłócał losowy operator selekcji naturalnej.

Przy pomocy haremowego operatora wyboru uzyskano drugi i czwarty najlepszy wynik.
Jak było wcześniej wspomniane, różnice w tej części rankingu są zbyt mało znaczące, aby z nich wnioskować.
Warto zauważyć, że oceny tych wyników mają największe odchylenia standardowe wśrod wyników DSEA.
Są one jednak od 3 do ponad 8 razy mniejsze niż w przypadku innych heurystyk.

\begin{table}[H]
	\centering
	\caption{Najlepsze wyniki różnych heurystyk dla problemu plecakowego, pokazane na wykresie \ref{figure:knapsack_results_summary} \label{table:knapsack_results_summary}}
	\begin{tabular}{|l|l|l|l|r@{$\pm$}l|}
		\hline
		\multicolumn{1}{|c|}{\multirow{2}{*}{{\bf Nr.}}} & \multicolumn{1}{c|}{\multirow{2}{*}{{\bf Heurystyka}}} & \multicolumn{1}{c|}{{\bf Op.selekcji}} & \multicolumn{1}{c|}{{\bf Op.selekcji}} & \multicolumn{2}{c|}{\multirow{2}{*}{{\bf Ocena}}} \\
		& \multicolumn{1}{c|}{}                                  & \multicolumn{1}{c|}{{\bf naturalnej}}    & \multicolumn{1}{c|}{{\bf płciowej}}  & \multicolumn{2}{c|}{}                             \\ \hline \hline
		1 & Alg. ewolucyjny                                         & opNatSel(RS)                           & stdGenSel($\bot$, R, R)                & -1068                   & 10,9727                 \\ \hline
		2 & GGA                                                     & opNatSel(RS)                           & stdGenSel($\top$, R, R)                & -1020                   & 115,6045                \\ \hline
		3 & \multirow{2}{*}{SexualGA}                               & \multirow{2}{*}{opNatSel(R)}           & stdGenSel($\bot$, RS, RS)              & -1102                   & 66,7383                 \\ \cline{1-1}\cline{4-6} 
		4 & &                                        & stdGenSel($\bot$, RS, TS(2))           & -909,2                  & 169,7308                \\ \hline
		5 & \multirow{6}{*}{DSEA}                                   & \multirow{6}{*}{opNatSel(RS)}          & stdGenSel($\bot$, RS, RS)              & -1152,8                 & 14,8647                 \\ \cline{1-1}\cline{4-6} 
		6 & &                                        & stdGenSel($\bot$, RS, R)               & -1143,2                 & 12,3839                 \\ \cline{1-1}\cline{4-6} 
		7 & &                                        & stdGenSel($\top$, RS, RS)              & -1138,6                 & 20,5777                 \\ \cline{1-1}\cline{4-6} 
		8 & &                                        & stdGenSel($\top$, RS, R)               & -1136,2                 & 20,4783                 \\ \cline{1-1}\cline{4-6} 
		9 & &                                        & harem(5, 0,25, RS, RS, R)              & -1144,2                 & 20,4294                 \\ \cline{1-1}\cline{4-6} 
		10 & &                                        & harem(3, 0,0, RS, TS(2), R)            & -1142                   & 19,308                  \\ \hline
	\end{tabular}
\end{table}

%\begin{table}[h]
%	\centering
%	\caption{My caption}
%	\label{my-label2}
%	\begin{tabular}{|l|l|l|r@{$\pm$}l|}
%		\hline
%		\multicolumn{1}{|c|}{\multirow{2}{*}{{\bf Heurystyka}}} & \multicolumn{1}{c|}{{\bf Op.selekcji}} & \multicolumn{1}{c|}{{\bf Op.selekcji}} & \multicolumn{2}{c|}{\multirow{2}{*}{{\bf Ocena}}} \\
%		\multicolumn{1}{|c|}{}                                  & \multicolumn{1}{c|}{{\bf płciowej}}    & \multicolumn{1}{c|}{{\bf naturalnej}}  & \multicolumn{2}{c|}{}                             \\ \hline \hline
%		Alg. ewolucyjny                                         & opNatSel(RS)                           & stdGenSel($\bot$, R, R)                & -1068                   & 10,9727                 \\ \hline
%		GGA                                                     & opNatSel(RS)                           & stdGenSel($\top$, R, R)                & -1020                   & 115,6045                \\ \hline
%		\multirow{2}{*}{SexualGA}                               & \multirow{2}{*}{opNatSel(R)}           & stdGenSel($\bot$, RS, RS)              & -1102                   & 66,7383                 \\ \cline{3-5} 
%		&                                        & stdGenSel($\bot$, RS, TS(2))           & -909,2                  & 169,7308                \\ \hline
%		\multirow{6}{*}{DSEA}                                   & \multirow{6}{*}{opNatSel(RS)}          & stdGenSel($\bot$, RS, RS)              & -1152,8                 & 14,8647                 \\ \cline{3-5} 
%		&                                        & stdGenSel($\bot$, RS, R)               & -1143,2                 & 12,3839                 \\ \cline{3-5} 
%		&                                        & stdGenSel($\top$, RS, RS)              & -1138,6                 & 20,5777                 \\ \cline{3-5} 
%		&                                        & stdGenSel($\top$, RS, R)               & -1136,2                 & 20,4783                 \\ \cline{3-5} 
%		&                                        & harem(5, 0,25, RS, RS, R)              & -1144,2                 & 20,4294                 \\ \cline{3-5} 
%		&                                        & harem(3, 0,0, RS, TS(2), R)            & -1142                   & 19,308                  \\ \hline
%	\end{tabular}
%\end{table}

\begin{figure}[H]
	\centering
	\graph{knapsack_results_summary.tex}
%	\customImg{graphs/knapsack_results_summary.png}
	\caption{Najlepsze wyniki różnych heurystyk dla problemu plecakowego, przedstawione w tabeli \ref{figure:knapsack_results_summary}. Wyższe słupki oznaczają lepsze wyniki. \label{figure:knapsack_results_summary}}
\end{figure}

\newpage

\begin{table}[H]
	\caption{Wyniki metaheurystyki DSEA z operatorem haremowym \label{table:knapsack_results_dsea_harem}}
	\centering
	\begin{tabular}{|l|l|l|l|r@{$\pm$}l|}
		\hline
		\multicolumn{1}{|c|}{{\bf a}} & \multicolumn{1}{|c|}{{\bf b}} & \multicolumn{1}{|c|}{{\bf WA}} & \multicolumn{1}{c|}{{\bf WB}} & \multicolumn{2}{c|}{{\bf Ocena}} \\ \hline \hline
		\insertData{knapsack_d_top}
	\end{tabular}	
\end{table}

W tabeli \ref{table:knapsack_results_dsea_harem} przedstawiono 12 najlepszych wyników heurystyki DSEA korzystającej z haremowego operatora selekcji płciowej.
Można zauważyć, że wyniki są na dość niskim poziomiem a pierwsze trzy do pięciu z najlepszych jest podobnych do pozostałych wyników heurystyki DSEA.

Ponadto, rozwiązując problem komiwojażera najlepsze wyniki osiągano korzystając jedynie z operatora ruletkowego i losowego, w różnych kombinacjach.
Przy rozwiązywaniu tego problemu dość często korzysta się również z operatora turniejowego.
Cztery z tych wyników korzystają z rozmiaru turnieju równego 2, a pozostałe dwa z rozmiaru równego 3, jednak jest to zbyt mała próbka danych, żeby wnioskować o jakichś proporcjach efektywności działania i wartości tego parametru.
Jeśli jednak zauważy się, że operator turniejowy używany był głównie w sytuacjach w których współczynnik beta był równo 0, a więc operator ten nie był w ogóle wykorzystywany, liczby te zmaleją do dwóch wyników z których każdy miał inny rozmiar turnieju (2 lub 3).
Są to dość niskie wyniki stojące daleko w rankingu, gorsze niż wyniki DSEA z uogólnionym operatorem selekcji płciowej, jednak wciąż lepsze niż wyniki pozostałych heurystyk.

\newpage

\begin{figure}[H]
	\caption{Wykres średniej oceny od łącznej liczby osobników alfa i beta \label{figure:knapsack_male_avg}}
	\centering
	\graph{knapsack_male_avg.tex}
\end{figure}

Na rysunku \ref{chapter:research} znajduje się wykres średniej oceny uruchomień w zależności od łącznej liczby samców (tj. osobników alfa i beta) w nich wykorzystanej.
Jest ona obliczana podobnie jak dla problemu komiwojażera, jednak tym razem $\param{rozmiarPopulacji}=60$.
Podobnie jak dla poprzedniego problemu, tu również nie da się zauważyć znaczącej zależności między tymi wartościami.

\paragraph{Podsumowanie}

Na tym etapie eksperymentów nie udało się znaleźć optimum globalnego.
Dla problemu komiwojażera znaleziono punkty dające zadowalające, jednak nie optymalne wyniki.
Tymczasem dla problemu plecakowego udało się znaleźć punkty o wartości kryterium bardzo bliskich wartości optimum.

Porównano cztery wybrane metody optymalizacji.
Dla problemu komiwojażera najlepszy okazał się klasyczny algorytm ewolucyjny, następnie DSEA, a na końcu heurystyki GGA i SexualGA.
Dla problemu plecakowego najlepsza była heurystyka DSEA, a standardowy algorytm ewolucyjny, SexualGA i GGA były na zbliżonym poziomie.

Dla obu problemów lepsze wyniki osiągnięto przy pomocy uogólnionego operatora selekcji płciowej, niż operatora haremowego.
W obu przypadakch najlepszy wynik spośród nich dawało użycie dwóch operatorów ruletkowych oraz ignorowanie cechy płci.
W ogólności operator sparametryzowany tak aby tą cechę ignorował daje nieznacznie lepsze wyniki niż operator sparametryzowany tymi samymi operatorami, ale wymuszający różną płeć rodziców.

Przy rozwiązywaniu problemu komiwojażera żadne z alternatywnych podejść nie było lepsze niż klasyczny algorytm ewolucyjny.
Jednakże dla problemu plecakowego udało się znacząco poprawić wyniki.
W tej sytuacji dodanie drugiego operatora selekcji wykorzystywanego do wyboru rodziców powoduje poprawę wyników.
Dzieje się tak jednak tylko wtedy, kiedy co najmniej jeden zastosowany operator wyboru jest taki sam jak zastosowany w operatorze selekcji naturalnej.



\end{document}