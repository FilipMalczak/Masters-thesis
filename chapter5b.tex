\documentclass[./FM_mgr.tex]{subfiles}

\begin{document}
\section{Badania} \label{chapter:research}

W dalszej części pracy konkretny zestaw parametrów będzie nazywany konfiguracją.
Wyrażenie ,,wynik eksperymentu dla konfiguracji wyniósł \emph{X}'' oznacza, że heurystyka została uruchomiona z parametrami (rozmiarem populacji, prawdopodobieństwami, itd.) określonymi przez konfigurację, a jej wynikiem było \emph{X}.
Należy pamiętać, że algorytm ewolucyjny to heurystyka losowa, więc wynik dla danej konfiguracji za każdym uruchomieniem może być inny.

\todo{2 etapy - jakie, każdy problem osobno}

\subsection{Określenie bazowej konfiguracji}
\label{subsection:init_params}

\todo{krótko o eksploracji}

\begin{config}
	\caption{Wartości wykorzystane podczas poszukiwania parametrów początkowych \label{config:tsp_init_params}}
	\centering
	\begin{tabular}{|c|l|}
		\hline
		\textbf{Parametr} & \multicolumn{1}{c|}{\textbf{Zbiór wartości}} \\
		\hline
		\hline
		$\param{rozmiarPopulacji}$ & 10, 20, 50 \\
		\hline
		$\param{wspDomieszek}$ & 0, 0,1, 0,25, 0,5\\
		\hline
		$(\param{prawdKrzyzowania}, \param{prawdMutacji})$ & [0,6, 0,7, 0,8] $\times$ [0,1, 0,2, 0,3]  \\
		\hline
		$\param{max}$ & 25, 50, 100 \\
		\hline		
		$\param{opSelNat}$ & \opName{natSel}(RS), \opName{natSel}(TS(2)), \opName{natSel}(TS(3))\\
		\hline
		\multicolumn{2}{c}{}\\
		\hline
		\textbf{Parametr} & \multicolumn{1}{c|}{\textbf{Wartość}} \\
		\hline
		\hline
		$\param{opSelPlciowej}$ & \opName{stdGenSel}($\bot$, R, R)\\
		\hline
%		$R$ & 3\\
%		\hline
%		$I$ & 5\\
%		\hline
%		$statystyka$ & najlepszy wynik \\
%		\hline
		Zbiór płci  & $\important{G}=\{ \odot \}$ \\
		\hline
	\end{tabular}
\end{config}

\todo{zakres, jak przy drugim powtórzeniu}

\todo{wynikowe konfigi}

\begin{config}
	\caption{Parametry używane w dalszych badaniach problemu komiwojażera \label{config:tsp_base}}
	\centering
	\begin{tabular}{|l|l|}
		\hline
		\textbf{Parametr} & \multicolumn{1}{c|}{\textbf{Wartość}} \\
		\hline
		\hline
		$\param{opSelNat}$ & \opName{natSel}(RS)\\
		\hline
		$\param{opSelPlciowej}$ & \opName{stdGenSel}($\bot$, R, R)\\
		\hline
		$\param{rozmiarPopulacji}$                         & 50                      \\ \hline 
		$\param{wspDomieszek}$                             & 0,0                     \\ \hline
		$\param{prawdKrzyzowania}$ & 0,6 \\ \hline 
		$\param{prawdMutacji}$ & 0,15             \\ \hline
		$\param{max}$                                      & 110                     \\ \hline
	\end{tabular}
\end{config}

\begin{config}
	\caption{Parametry używane w dalszych badaniach problemu plecakowego \label{config:knapsack_base}}
	\centering
	\begin{tabular}{|l|l|}
		\hline
		\textbf{Parametr} & \multicolumn{1}{c|}{\textbf{Wartość}} \\
		\hline
		\hline
		$\param{opSelNat}$ & \opName{natSel}(RS)\\
		\hline
		$\param{rozmiarPopulacji}$                         & 60                      \\ \hline 
		$\param{wspDomieszek}$                             & 0,1                    \\ \hline
		$\param{prawdKrzyzowania}$ & 0,75 \\ \hline 
		$\param{prawdMutacji}$ & 0,15             \\ \hline
		$\param{max}$                                      & 60                     \\ \hline
	\end{tabular}
\end{config}

\subsection{Badania porównawcze} \label{subsection:compare}
\todo{co zbadano}
\todo{fallback do poprzednich tabel}

\subsubsection{Badane konfiguracje}

\todo{plain; gga; sexual ga; uogólniony ; harem}

\begin{config}
	\caption{Konfiguracja standardowego algorytmu ewolucyjnego \label{config:ea}}
	\centering
	\begin{tabu} to \textwidth {lX}
		\hline
		\multicolumn{1}{|l|}{\textbf{Operator selekcji płciowej}} &
		\multicolumn{1}{l|}{\opName{stdGenSel}($\bot$, R, R)} \\ 
		\hline
		\multicolumn{1}{|l|}{\textbf{Pozostałe parametry}} & 
		\multicolumn{1}{l|}{Zgodne z konfiguracją \ref{config:tsp_base} lub \ref{config:knapsack_base}} \\ 
		\hline
		\multicolumn{1}{|l|}{\textbf{Ilość powtórzeń}} &
		\multicolumn{1}{l|}{5} \\ 
		\hline
		\multicolumn{1}{|l|}{\textbf{Zbiór płci}} & 
		\multicolumn{1}{l|}{$\important{G} = \{ \odot \}$} \\ 
		\hline
	\end{tabu}
\end{config}

\begin{config}
	\caption{Konfiguracja heurystyki GGA \label{config:gga}}
	\begin{tabularx}{\linewidth}{lX}
		\hline
		\multicolumn{1}{|l|}{\textbf{Operator selekcji płciowej}} & \multicolumn{1}{l|}{\opName{stdGenSel}($\top$, R, R)} \\ 
		\hline
		\multicolumn{1}{|l|}{\textbf{Operator selekcji naturalnej}} & \multicolumn{1}{l|}{\opName{natSel}($W$)} \\ 
		\hline
		\multicolumn{1}{|l|}{\textbf{Pozostałe parametry}} & 
		\multicolumn{1}{l|}{Zgodne z konfiguracją \ref{config:tsp_base} lub \ref{config:knapsack_base}} \\ 
		\hline
		\multicolumn{1}{|l|}{\textbf{Ilość powtórzeń}} & \multicolumn{1}{l|}{5} \\ 
		\hline
		\multicolumn{1}{|l|}{\textbf{Zbiór płci}} & 
		\multicolumn{1}{l|}{$\important{G} = \{ \male, \female \}$} \\ 
		\hline
		& \\
		\hline
		\multicolumn{1}{|c|}{\textbf{Parametr}} & 
		\multicolumn{1}{c|}{\textbf{Zbiór wartości}} \\ 
		\hline \hline
		\multicolumn{1}{|l|}{$W$} & 
		\multicolumn{1}{l|}{RS, TS(2), TS(3)} \\ 
		\hline
	\end{tabularx}
\end{config}

\begin{config}
	\caption{Konfiguracja heurystyki SexualGA \label{config:sexual_ga}}
	\centering
	\begin{tabularx}{\linewidth}{lX}
		\hline
		\multicolumn{1}{|l|}{\textbf{Operator selekcji płciowej}} &
		\multicolumn{1}{l|}{\opName{stdGenSel}($\bot$, $\param{opWyboru1}$, $\param{opWyboru2}$)} \\ 
		\hline
		\multicolumn{1}{|l|}{\textbf{Operator selekcji naturalnej}} &
		\multicolumn{1}{l|}{\opName{natSel}(R)} \\ 
		\hline
		\multicolumn{1}{|l|}{\textbf{Pozostałe parametry}} & 
		\multicolumn{1}{l|}{Zgodne z konfiguracją \ref{config:tsp_base} lub \ref{config:knapsack_base}} \\ 
		\hline
		\multicolumn{1}{|l|}{\textbf{Ilość powtórzeń}} &
		\multicolumn{1}{l|}{5} \\ 
		\hline
		\multicolumn{1}{|l|}{\textbf{Zbiór płci}} & 
		\multicolumn{1}{l|}{$\important{G} = \{ \odot \}$} \\ 
		\hline
		& \\ 
		\hline
		\multicolumn{1}{|c|}{\textbf{Parametr}} & 
		\multicolumn{1}{c|}{\textbf{Zbiór wartości}} \\ 
		\hline \hline
		\multicolumn{1}{|l|}{$\param{opWyboru1}$} & 
		\multicolumn{1}{l|}{RS, TS(2), TS(3)} \\ 
		\hline
		\multicolumn{1}{|l|}{$\param{opWyboru2}$} & 
		\multicolumn{1}{l|}{R, RS, TS(2), TS(3)} \\ 
		\hline
	\end{tabularx}
\end{config}

\begin{config}
	%gender, true
	\caption{Konfiguracja heurystyki DSEA z uogólnionym operatorem selekcji płciowej \label{config:dsea_general}}
	\centering
	\begin{tabularx}{\linewidth}{lX}
		\hline
		\multicolumn{1}{|l|}{\textbf{Operator selekcji płciowej}} &
		\multicolumn{1}{l|}{\opName{stdGenSel}($p$, $\param{opWyboru1}$, $\param{opWyboru2}$)} \\ 
		\hline
		\multicolumn{1}{|l|}{\textbf{Pozostałe parametry}} &
		\multicolumn{1}{l|}{Zgodne z konfiguracją \ref{config:tsp_base} lub \ref{config:knapsack_base}} \\ 
		\hline
		\multicolumn{1}{|l|}{\textbf{Ilość powtórzeń}} &
		\multicolumn{1}{l|}{5} \\ 
		\hline
		& \\ 
		\hline
		\multicolumn{1}{|c|}{\textbf{Parametr}} & 
		\multicolumn{1}{c|}{\textbf{Zbiór wartości}} \\ 
		\hline \hline
		\multicolumn{1}{|l|}{$p = \param{plecMaZnaczenie}$} & 
		\multicolumn{1}{l|}{$\top, \bot$} \\
		\hline
		\multicolumn{1}{|l|}{$\param{opWyboru1}$} & 
		\multicolumn{1}{l|}{RS, TS(2), TS(3)} \\ 
		\hline
		\multicolumn{1}{|l|}{$\param{opWyboru2}$} & 
		\multicolumn{1}{l|}{R, RS, TS(2), TS(3)} \\
		\hline
		\multicolumn{1}{|l|}{\textbf{Zbiór płci}} & 
		\multicolumn{1}{l|}{ $|\important{G}| \in \{1, 2\}$, zależnie od $\param{plecMaZnaczenie}$ } \\ 
		\hline
	\end{tabularx}
\end{config}

\begin{config}
	%harem
	\caption{Konfiguracja heurystyki DSEA z haremowym operatorem selekcji płciowej \label{config:dsea_harem}}
	\begin{tabularx}{\linewidth}{lX}
		\hline
		\multicolumn{1}{|l|}{\textbf{Operator selekcji płciowej}} &
		\multicolumn{1}{l|}{\opName{harem}(a, b, WA, WB, WP)} \\ 
		\hline
		\multicolumn{1}{|l|}{\textbf{Pozostałe parametry}} &
		\multicolumn{1}{l|}{Zgodne z konfiguracją \ref{config:tsp_base} lub \ref{config:knapsack_base}} \\ 
		\hline
		\multicolumn{1}{|l|}{\textbf{Ilość powtórzeń}} &
		\multicolumn{1}{l|}{5} \\ 
		\hline
		\multicolumn{1}{|l|}{\textbf{Zbiór płci}} & 
		\multicolumn{1}{l|}{$\important{G} = \{ \male, \female \}$} \\ 
		\hline
		& \\ 
		\hline
		\multicolumn{1}{|c|}{\textbf{Parametr}} & 
		\multicolumn{1}{c|}{\textbf{Zbiór wartości}} \\ 
		\hline \hline
		\multicolumn{1}{|l|}{a} & 
		\multicolumn{1}{l|}{1, 3, 5} \\ 
		\hline
		\multicolumn{1}{|l|}{b} & 
		\multicolumn{1}{l|}{0, 0,1, 0,25} \\ 
		\hline
		\multicolumn{1}{|l|}{WA} & 
		\multicolumn{1}{l|}{R, RS, TS(2), TS(3)} \\ 
		\hline
		\multicolumn{1}{|l|}{WB} & 
		\multicolumn{1}{l|}{R, RS, TS(2), TS(3)} \\ 
		\hline
		& \\ 
		\hline
		\multicolumn{1}{|c|}{\textbf{Parametr}} & 
		\multicolumn{1}{c|}{\textbf{Wartość}} \\ 
		\hline \hline
		\multicolumn{1}{|l|}{WP} &
		\multicolumn{1}{l|}{R} \\ 
		\hline
	\end{tabularx}
\end{config}

\subsubsection{Wyniki}

\paragraph{Problem komiwojażera}



\begin{table}[h]
	\centering
	\caption{My caption}
	\label{my-label}
	\begin{tabular}{|l|l|l|l|r@{$\pm$}l|}
		\hline
		\multicolumn{1}{|c|}{\multirow{2}{*}{{\bf Nr.}}} &\multicolumn{1}{c|}{\multirow{2}{*}{{\bf Heurystyka}}} & \multicolumn{1}{c|}{{\bf Op. selekcji}} & \multicolumn{1}{c|}{{\bf Op. selekcji}} & \multicolumn{2}{c|}{\multirow{2}{*}{{\bf Ocena}}} \\
		& \multicolumn{1}{c|}{}                                  & \multicolumn{1}{c|}{{\bf naturalnej}}          & \multicolumn{1}{c|}{{\bf płciowej}}        & \multicolumn{2}{c|}{}                             \\ \hline \hline
		1 & Alg. ewolucyjny                    & opNatSel(RS)                                          & stdGenSel($\bot$, R, R)                                 & 37830,7834      & 1865,7587      \\ \hline
		2 & GGA                                    & opNatSel(RS)                                          & stdGenSel($\top$, R, R)                                 & 68582,5919      & 2240,5888      \\ \hline
		3 & \multirow{2}{*}{SexualGA}              & \multirow{2}{*}{opNatSel(R)}                          & stdGenSel($\bot$, RS, RS)                               & 65454,5598      & 4541,7191      \\ \cline{1-1}\cline{4-6} 
		4 & &                                                       & stdGenSel($\bot$, RS, R)                                & 72023,5817      & 5135,7596      \\ \hline
		5 & \multirow{6}{*}{DSEA}                  & \multirow{6}{*}{opNatSel(RS)}                         & stdGenSel($\bot$, RS, RS)                               & 40199,6226      & 3708,8267      \\ \cline{1-1}\cline{4-6} 
		6 & &                                                       & stdGenSel($\bot$, RS, R)                                & 45328,6013      & 967,5844       \\ \cline{1-1}\cline{4-6} 
		7 & &                                                       & stdGenSel($\top$, RS, RS)                               & 41555,6377      & 3146,2442      \\ \cline{1-1}\cline{4-6} 
		8 & &                                                       & stdGenSel($\top$, RS, R)                                & 46760,4675      & 4193,2035      \\ \cline{1-1}\cline{4-6} 
		9 & &                                                       & harem(5, 0,1, RS, RS, R)                                & 49180,7887      & 4249,2504      \\ \cline{1-1}\cline{4-6} 
		10 & &                                                       & harem(5, 0,1, RS, R, R)                                 & 48915,0643      & 799,6703       \\ \hline
	\end{tabular}
\end{table}

%\begin{table}[h]
%	\centering
%	\caption{My caption}
%	\label{my-label}
%	\begin{tabular}{|l|l|l|r@{$\pm$}l|}
%		\hline
%		\multicolumn{1}{|c|}{\multirow{2}{*}{{\bf Heurystyka}}} & \multicolumn{1}{c|}{{\bf Op. selekcji}} & \multicolumn{1}{c|}{{\bf Op. selekcji}} & \multicolumn{2}{c|}{\multirow{2}{*}{{\bf Ocena}}} \\
%		\multicolumn{1}{|c|}{}                                  & \multicolumn{1}{c|}{{\bf naturalnej}}          & \multicolumn{1}{c|}{{\bf płciowej}}        & \multicolumn{2}{c|}{}                             \\ \hline \hline
%		Alg. ewolucyjny                    & opNatSel(RS)                                          & stdGenSel($\bot$, R, R)                                 & 37830,7834      & 1865,7587      \\ \hline
%		GGA                                    & opNatSel(RS)                                          & stdGenSel($\top$, R, R)                                 & 68582,5919      & 2240,5888      \\ \hline
%		\multirow{2}{*}{SexualGA}              & \multirow{2}{*}{opNatSel(R)}                          & stdGenSel($\bot$, RS, RS)                               & 65454,5598      & 4541,7191      \\ \cline{3-5} 
%		&                                                       & stdGenSel($\bot$, RS, R)                                & 72023,5817      & 5135,7596      \\ \hline
%		\multirow{6}{*}{DSEA}                  & \multirow{6}{*}{opNatSel(RS)}                         & stdGenSel($\bot$, RS, RS)                               & 40199,6226      & 3708,8267      \\ \cline{3-5} 
%		&                                                       & stdGenSel($\bot$, RS, R)                                & 45328,6013      & 967,5844       \\ \cline{3-5} 
%		&                                                       & stdGenSel($\top$, RS, RS)                               & 41555,6377      & 3146,2442      \\ \cline{3-5} 
%		&                                                       & stdGenSel($\top$, RS, R)                                & 46760,4675      & 4193,2035      \\ \cline{3-5} 
%		&                                                       & harem(5, 0,1, RS, RS, R)                                & 49180,7887      & 4249,2504      \\ \cline{3-5} 
%		&                                                       & harem(5, 0,1, RS, R, R)                                 & 48915,0643      & 799,6703       \\ \hline
%	\end{tabular}
%\end{table}

\begin{figure}
	\centering
	\graph{tsp_results_summary.tex}
\end{figure}

\todo{słupki i tabela z porównaniem plain/gga/sga/general true/false/harem}

W tabeli \ref{table:tsp_results_dsea_harem} przedstawiono 12 najlepszych wyników uzyskanych przy użyciu konfiguracji \ref{config:dsea_harem}.
Zestawienie pełnych wyników można znaleźć w dodatku \ref{appendix:full_results}.

Można zaobserwować, że heurystyka działała najskuteczniej, kiedy jeden lub oba operatory wyboru były takie same jak użyty w operatorze selekcji naturalnej.
Ponadto, najlepsze wyniki uzyskano dla większych liczb osobników alfa i beta.

Warto zauważyć, że najlepszy i najbardziej stabilny wynik uzyskano wybierając osobniki alfa operatorem ruletkowym, a osobniki beta i partnerów losowym.
Może to oznaczać, że drastyczna selekcja rodziców nie jest nieuzasadniona, a osobniki alfa mają najbardziej korzystne cechy.

\begin{table}[h]
	\caption{Wyniki heurystyki DSEA z operatorem haremowym \label{table:tsp_results_dsea_harem}}
	\centering
	\begin{tabular}{|l|l|l|l|r@{$\pm$}l|}
		\hline
		\multicolumn{1}{|c|}{{\bf a}} & \multicolumn{1}{|c|}{{\bf b}} & \multicolumn{1}{|c|}{{\bf WA}} & \multicolumn{1}{c|}{{\bf WB}} & \multicolumn{2}{c|}{{\bf Ocena}} \\ \hline \hline
		\insertData{tsp_d_top}
	\end{tabular}	
\end{table}

Wykres \ref{figure:tsp_male_avg} przedstawia średnią funkcję oceny dla różnych liczb samców.
Liczba ta została obliczona jako $\lceil \param{liczbaAlfa} + \param{wspBeta} \times 50 \rceil$, ponieważ $\param{rozmiarPopulacji}$ wynosił 50.
Jak widać, nie ma żadnej jednoznacznej zależności między tą liczbą, a wynikami, z czego można wnioskować, że to operatory wyboru (losowy lub ruletkowy w większości przypadków) decydowały o skuteczności.

\begin{figure}
	\caption{Wykres średniej oceny od łącznej liczby osobników alfa i beta \label{figure:tsp_male_avg}}
	\centering
	\graph{tsp_male_avg.tex}
\end{figure}

\paragraph{Problem plecakowy}

\begin{figure}
	\centering
	\graph{knapsack_results_summary.tex}
%	\customImg{graphs/knapsack_results_summary.png}
\end{figure}


\begin{figure}
	\caption{Wykres średniej oceny od łącznej liczby osobników alfa i beta \label{figure:knapsack_male_avg}}
	\centering
	\graph{knapsack_male_avg.tex}
\end{figure}

\begin{table}[h]
	\centering
	\caption{My caption}
	\label{my-label2}
	\begin{tabular}{|l|l|l|l|r@{$\pm$}l|}
		\hline
		\multicolumn{1}{|c|}{\multirow{2}{*}{{\bf Nr.}}} & \multicolumn{1}{c|}{\multirow{2}{*}{{\bf Heurystyka}}} & \multicolumn{1}{c|}{{\bf Op.selekcji}} & \multicolumn{1}{c|}{{\bf Op.selekcji}} & \multicolumn{2}{c|}{\multirow{2}{*}{{\bf Ocena}}} \\
		& \multicolumn{1}{c|}{}                                  & \multicolumn{1}{c|}{{\bf naturalnej}}    & \multicolumn{1}{c|}{{\bf płciowej}}  & \multicolumn{2}{c|}{}                             \\ \hline \hline
		1 & Alg. ewolucyjny                                         & opNatSel(RS)                           & stdGenSel($\bot$, R, R)                & -1068                   & 10,9727                 \\ \hline
		2 & GGA                                                     & opNatSel(RS)                           & stdGenSel($\top$, R, R)                & -1020                   & 115,6045                \\ \hline
		3 & \multirow{2}{*}{SexualGA}                               & \multirow{2}{*}{opNatSel(R)}           & stdGenSel($\bot$, RS, RS)              & -1102                   & 66,7383                 \\ \cline{1-1}\cline{4-6} 
		4 & &                                        & stdGenSel($\bot$, RS, TS(2))           & -909,2                  & 169,7308                \\ \hline
		5 & \multirow{6}{*}{DSEA}                                   & \multirow{6}{*}{opNatSel(RS)}          & stdGenSel($\bot$, RS, RS)              & -1152,8                 & 14,8647                 \\ \cline{1-1}\cline{4-6} 
		6 & &                                        & stdGenSel($\bot$, RS, R)               & -1143,2                 & 12,3839                 \\ \cline{1-1}\cline{4-6} 
		7 & &                                        & stdGenSel($\top$, RS, RS)              & -1138,6                 & 20,5777                 \\ \cline{1-1}\cline{4-6} 
		8 & &                                        & stdGenSel($\top$, RS, R)               & -1136,2                 & 20,4783                 \\ \cline{1-1}\cline{4-6} 
		9 & &                                        & harem(5, 0,25, RS, RS, R)              & -1144,2                 & 20,4294                 \\ \cline{1-1}\cline{4-6} 
		10 & &                                        & harem(3, 0,0, RS, TS(2), R)            & -1142                   & 19,308                  \\ \hline
	\end{tabular}
\end{table}

%\begin{table}[h]
%	\centering
%	\caption{My caption}
%	\label{my-label2}
%	\begin{tabular}{|l|l|l|r@{$\pm$}l|}
%		\hline
%		\multicolumn{1}{|c|}{\multirow{2}{*}{{\bf Heurystyka}}} & \multicolumn{1}{c|}{{\bf Op.selekcji}} & \multicolumn{1}{c|}{{\bf Op.selekcji}} & \multicolumn{2}{c|}{\multirow{2}{*}{{\bf Ocena}}} \\
%		\multicolumn{1}{|c|}{}                                  & \multicolumn{1}{c|}{{\bf płciowej}}    & \multicolumn{1}{c|}{{\bf naturalnej}}  & \multicolumn{2}{c|}{}                             \\ \hline \hline
%		Alg. ewolucyjny                                         & opNatSel(RS)                           & stdGenSel($\bot$, R, R)                & -1068                   & 10,9727                 \\ \hline
%		GGA                                                     & opNatSel(RS)                           & stdGenSel($\top$, R, R)                & -1020                   & 115,6045                \\ \hline
%		\multirow{2}{*}{SexualGA}                               & \multirow{2}{*}{opNatSel(R)}           & stdGenSel($\bot$, RS, RS)              & -1102                   & 66,7383                 \\ \cline{3-5} 
%		&                                        & stdGenSel($\bot$, RS, TS(2))           & -909,2                  & 169,7308                \\ \hline
%		\multirow{6}{*}{DSEA}                                   & \multirow{6}{*}{opNatSel(RS)}          & stdGenSel($\bot$, RS, RS)              & -1152,8                 & 14,8647                 \\ \cline{3-5} 
%		&                                        & stdGenSel($\bot$, RS, R)               & -1143,2                 & 12,3839                 \\ \cline{3-5} 
%		&                                        & stdGenSel($\top$, RS, RS)              & -1138,6                 & 20,5777                 \\ \cline{3-5} 
%		&                                        & stdGenSel($\top$, RS, R)               & -1136,2                 & 20,4783                 \\ \cline{3-5} 
%		&                                        & harem(5, 0,25, RS, RS, R)              & -1144,2                 & 20,4294                 \\ \cline{3-5} 
%		&                                        & harem(3, 0,0, RS, TS(2), R)            & -1142                   & 19,308                  \\ \hline
%	\end{tabular}
%\end{table}

\begin{table}[h]
	\caption{Wyniki heurystyki DSEA z operatorem haremowym \label{table:knapsack_results_dsea_harem}}
	\centering
	\begin{tabular}{|l|l|l|l|r@{$\pm$}l|}
		\hline
		\multicolumn{1}{|c|}{{\bf a}} & \multicolumn{1}{|c|}{{\bf b}} & \multicolumn{1}{|c|}{{\bf WA}} & \multicolumn{1}{c|}{{\bf WB}} & \multicolumn{2}{c|}{{\bf Ocena}} \\ \hline \hline
		\insertData{knapsack_d_top}
	\end{tabular}	
\end{table}


\end{document}