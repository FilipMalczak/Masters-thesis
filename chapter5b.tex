\documentclass[./FM_mgr.tex]{subfiles}

\begin{document}
\section{Przeprowadzone eksperymenty}

Badania były przeprowadzane oddzielnie dla każdego problemu.
Ich pierwszą częścią było znalezienie początkowej konfiguracji, wykorzystywanej w dalszych etapach.
Odbywało się to w dwóch krokach.
Pierwszy z nich polegał na znalezieniu parametrów dla standardowego algoryytmu ewolucyjnego przy dużej ziarnistości przeszukiwanych wartości, a drugi - na przeszukaniu lokalnego sąsiedztwa konfiguracji znalezionej w kroku pierwszym.
W tej części zdarzały się sytuacje w których znaleziono globalne optimum dla problemu, jednak takie zestawy parametrów zostawały odrzucone.
Miało to na celu uzyskanie wyników dobrych, ale nie najlepszych, aby sprawdzić, czy w dalszych eksperymentach uda się osiągnąć lepsze rezultaty.
Gdyby już na tym etapie znajdowano optimum globalne, nie dałoby się określić, czy nowe rozwiązania poprawiają wyniki.

Drugą częścią badań było zebranie wyników dla DSEA z wykorzystaniem parametrów znalezionych w poprzednim kroku, ale z różnymi implementacjami operatorów selekcji płciowej. 
Najpierw odtworzono działanie heurystyk GGA i DSEA sparametryzowanej w podobny sposób. Następnie powtórzono to dla heurystyki SexualGA. 
Na końcu zbadano działanie heurystyki DSEA z operatorem haremowym.

Każdy z tych kroków zostanie opisany oddzielnie, z określeniem rozwiązywanego problemu, zakresem zmienianych parametrów, opisem przebiegu eksperymentu oraz jego najważniejszymi wynikami.

Pełne wyniki eksperymentów są zamieszczone w dodatkach do niniejszego dokumentu.

Kolejne podsekcje opisują przebieg eksperymentów dla problemu komiwojażera i problemu plecakowego, w tej kolejności.

Dla każdego z eksperymentów należało określić (poza rozwiązywanym problemem, a więc reprezentacją osobnika, operatorami genetycznymi i funkcją oceny):

\begin{itemize}
	\item rozmiar populacji ($\param{rozmiarPopulacji}$),
	\item współczynnik domieszek ($\param{wspDomieszek}$),
	\item liczbę pokoleń (parametr kryterium stopu, $\param{max}$),
	\item prawdopodobieństwa mutacji i krzyżowania ($\param{prawdMutacji}$, $\param{prawdKrzyzowania}$),
	\item operator selekcji naturalnej ($\param{opSelNat}$), definiowany przez wykorzystany operator wyboru,
	\item operator selekcji płciowej ($\param{opSelPlciowej}$), wraz z jego parametrami (zależnymi od implementacji).
\end{itemize}

Ponadto, korzystając z procedury eksploracji poza powyższymi zbiorami wartości parametrów trzeba było wskazać wartości:

\begin{itemize}
	\item ilości nawrotów ($R$),
	\item ilości powtórzeń eksperymentu ($I$), oraz
	\item statystyki używanej do wyboru dalszej ścieżki eksploracji ($statystyka$).
\end{itemize}

\subsection{Badania problemu komiwojażera}

W następnych podsekcjach zostaną opisane kolejne kroki badań działania DSEA dla problemu komiwojażera, zdefiniowanego i zaimplementowanego zgodnie z podsekcją \ref{subsection:tsp_impl}.

\subsubsection{Poszukiwanie parametrów początkowych}

\paragraph{Konfiguracja}

W tym kroku wykorzystano procedurę eksploracji w celu znalezienia konfiguracji początkowej. 
Konfiguracja \ref{config:tsp_init_params} przedstawia zbiory wartości parametrów heurystyki oraz wartości stałe w ramach tego etapu - zarówno dla algorytmu ewolucyjnego jak i procedury eksploracji użyte w tym kroku.

Zastosowano losowy operator selekcji płciowej, bez uwzględniania płci, aby zasymulować działanie standardowego algorytmu ewolucyjnego. 
W związku z tym użyto również jednoelementowego zbioru płci. 
Statystyką użytą do porównywania konfiguracji była ocena najlepszego przebiegu.
Miało to na celu silna eliminację parametrów które mogły pogarszać wyniki.

\begin{config}
	\caption{Wartości wykorzystane podczas poszukiwania parametrów początkowych \label{config:tsp_init_params}}
	\centering
	\begin{tabular}{|c|l|}
		\hline
		\textbf{Parametr} & \multicolumn{1}{c|}{\textbf{Zbiór wartości}} \\
		\hline
		\hline
		$\param{rozmiarPopulacji}$ & [10, 20, 50] \\
		\hline
		$\param{wspDomieszek}$ & [0, 0,1, 0,25, 0,5]\\
		\hline
		$(\param{prawdKrzyzowania}, \param{prawdMutacji})$ & [0,6, 0,7, 0,8] $\times$ [0,1, 0,2, 0,3]  \\
		\hline
		$\param{max}$ & [25, 50, 100] \\
		\hline		
		$\param{opSelNat}$ & [\opName{natSel}(RS), \opName{natSel}(TS(2)), \opName{natSel}(TS(3))]\\
		\hline
		\multicolumn{2}{c}{}\\
		\hline
		\textbf{Parametr} & \multicolumn{1}{c|}{\textbf{Wartość}} \\
		\hline
		\hline
		$\param{opSelPlciowej}$ & \opName{stdGenSel}($\bot$, R, R)\\
		\hline
		$R$ & 3\\
		\hline
		$I$ & 5\\
		\hline
		$statystyka$ & najlepszy wynik \\
		\hline
		Zbiór płci  & $\important{G}=\{ \odot \}$ \\
		\hline
	\end{tabular}
\end{config}

\paragraph{Przebieg}

Tabela \ref{table:tsp_init_flow} przedstawia przebieg wykorzystanej procedury. 
Lewa kolumna grupuje znalezione parametry w ramach jednego nawrotu eksploracji. 
Kolejne dwie, czytane od góry, określają jakie wartości zostały kolejno znalezione.

Jak widać, w pierwszym nawrocie znaleziono prawie najlepszą konfigurację. 
W drugim, udało się odnaleźć prawdopodobieństwa mutacji i krzyżowania które dawały lepsze wyniki.
Trzeci nawrót potwierdził, że znaleziona konfiguracja jest najlepsza z przeszukiwanych.

\begin{table}[H]
	\caption{Przebieg procedury eksploracji poszukującej parametrów początkowych \label{table:tsp_init_flow}}
	\centering
	\begin{tabular}{c|c|c|}
		\cline{2-3}
		\multicolumn{1}{l|}{} &
		{\bf Parametr} & 
		{\bf Określona wartość} \\ 
		\hline
		\multicolumn{1}{|c|}{\multirow{5}{*}{{\bf Nawrót 1}}} &
		$\param{rozmiarPopulacji}$ & 
		50 \\ 
		\cline{2-3} 
		\multicolumn{1}{|c|}{} & 
		$\param{wspDomieszek}$ & 
		0,0 \\ 
		\cline{2-3}
		\multicolumn{1}{|c|}{} & 
		$(\param{prawdKrzyzowania}, \param{prawdMutacji})$ & (0,8, 0,3) \\ \cline{2-3} 
		\multicolumn{1}{|c|}{} & 
		$\param{max}$ & 
		100 \\ 
		\cline{2-3} 
		\multicolumn{1}{|c|}{} & 
		$\param{opSelNat}$ & 
		\opName{natSel}(RS) \\ 
		\hline \hline
		\multicolumn{1}{|c|}{\multirow{5}{*}{{\bf Nawrót 2}}} &
		$\param{rozmiarPopulacji}$ & 
		50 \\ 
		\cline{2-3} 
		\multicolumn{1}{|c|}{} & 
		$\param{wspDomieszek}$ & 
		0,0 \\ 
		\cline{2-3} 
		\multicolumn{1}{|c|}{} & 
		$(\param{prawdMutacji}, \param{prawdKrzyzowania})$ 
		& (0,6, 0,1) \\ 
		\cline{2-3} 
		\multicolumn{1}{|c|}{} & 
		$\param{max}$ & 
		100 \\ 
		\cline{2-3} 
		\multicolumn{1}{|c|}{} & 
		$\param{opSelNat}$ & 
		\opName{natSel}(RS) \\ 
		\hline \hline
		\multicolumn{1}{|c|}{\multirow{5}{*}{{\bf Nawrót 3}}} &
		$\param{rozmiarPopulacji}$ & 
		50 \\ 
		\cline{2-3} 
		\multicolumn{1}{|c|}{} & 
		$\param{wspDomieszek}$ & 
		0,0 \\ 
		\cline{2-3} 
		\multicolumn{1}{|c|}{} & 
		$(\param{prawdMutacji}, \param{prawdKrzyzowania})$ & 
		(0,6, 0,1) \\ 
		\cline{2-3} 
		\multicolumn{1}{|c|}{} & 
		$\param{max}$ & 
		100 \\ 
		\cline{2-3} 
		\multicolumn{1}{|c|}{} & 
		$\param{opSelNat}$ & 
		\opName{natSel}(RS) \\ 
		\hline
	\end{tabular}
\end{table}

\paragraph{Wyniki}

Na tym etapie eksperymentów nie udało się znaleźć globalnego optimum.
W tabeli \ref{table:tsp_init_results} przedstawiono wyniki wszystkich powtórzeń dla konfiguracji znalezionej w procesie eksploracji.
Średnia różnica między znalezionym, a najlepszym możliwym rozwiązanie wynosiła około 13 tysięcy, podczas gdy optimum miało wartość około 27 i pół tysiąca.
Oznacza to, że znajdowane rozwiązania były około półtora raza gorsze niż poszukiwane.

Heurystyka zachowywała się dość stabilnie, tzn. za każdym uruchomieniem zwracała podobny wynik.
Odchylenie standardowe wyniosło około 2 tysiące, czyli mniej więcej 5\% średniej.

Na rysunku \ref{figure:tsp_init_example} przedstawiono wykres trzeciego z pięciu wykonanych przebiegów.
Można zaobserwować, że zarówno średnia, jak i najlepsze i najgorsze rozwiązanie spadają na przestrzeni wszystkich populacji, co oznacza, że wyniki sukcesywnie się polepszały.

Jednocześnie da się zauważyć, że wariancja wariancja utrzymywała się na dość stałym poziomie, z czego można wnioskować, że różnorodność genetyczna była zachowana.

\begin{table}[H]
	\caption{Wyniki heurystyki parametryzowanej znalezioną konfiguracją \label{table:tsp_init_results}}
	\centering
	\begin{tabular}{|l|r|r}
		\hline
		\multicolumn{1}{|c|}{{\bf Numer powtórzenia}} & \multicolumn{1}{c|}{{\bf Wynik}} & \multicolumn{1}{c|}{{\bf Różnica względem optimum}} \\ \hline \hline
		1                                             & 42129,167                        & \multicolumn{1}{r|}{14526,167}                      \\ \hline
		2                                             & 42808,0792                       & \multicolumn{1}{r|}{15205,0792}                     \\ \hline
		3                                             & 36656,3316                       & \multicolumn{1}{r|}{9053,3316}                      \\ \hline
		4                                             & 42174,1093                       & \multicolumn{1}{r|}{14571,1093}                     \\ \hline
		5                                             & 41143,6598                       & \multicolumn{1}{r|}{13540,6598}                     \\ \hline \hline
		{\bf Średnia}                                 & 40982,2694                      & \multicolumn{1}{r|}{13379,2694}                    \\ \hline
		{\bf Odchylenie standardowe}                  & 2227,5198                        &                       \\  \hhline{==~}
		{\bf Optimum}                                 & \multicolumn{1}{r|}{{\bf 27603}} & \multicolumn{1}{l}{}                                \\ \cline{1-2}
	\end{tabular}
\end{table}

\begin{figure}[H]
	\caption{Wykres przebiegu trzeciego powtórzenia znalezionej konfiguracji \label{figure:tsp_init_example}}
	\centering
	\graph{tsp_init_example.tex}
\end{figure}

\subsubsection{Poprawa konfiguracji początkowej}

\paragraph{Konfiguracja} Kolejny etap eksperymentów miał na celu zbadanie przestrzeni parametrów z większą ziarnistością.
Ponownie wykorzystano procedurę eksploracji.
Zestawy wartości parametrów zostały określone na podstawie konfiguracji znalezionej w poprzednim kroku.
Do wartości całkowitoliczbowych (rozmiaru populacji i liczby pokoleń) dodano i odjęto wartości 5 i 10 w celu przeszukania jej lokalnego sąsiedztwa.
Analogicznie, do wartości rzeczywistych (jak współcznnik domieszek i prawdopodbieństwa zastosowania operatorów genetycznych) dodano i odjęto wartości 0,02 i 0,05.
Rzecz jasna wykorzystano jedynie te wyniki, które miały sens, tj. dodatnie, a w przypadku parametrów ciągłych - mniejsze od 1.
Operator selekcji został niezmieniony, ponieważ jest to poniekąd parametr wyliczeniowy i nie da się określić dla niego sąsiedztwa.

Na tym etapie wykonano jedynie 2 nawroty, trzykrotnie uruchamiając heurystykę z każdą konfiguracją.

\begin{config}
	\caption{Wartości wykorzystane podczas poprawy parametrów początkowych \label{config:tsp_tweak_params}}
	\centering
	\begin{tabular}{|c|l|}
		\hline
		\textbf{Parametr} & \multicolumn{1}{c|}{\textbf{Zbiór wartości}} \\
		\hline
		\hline
		$\param{rozmiarPopulacji}$ & [40, 45, 50, 55, 60] \\
		\hline
		$\param{wspDomieszek}$ & [0, 0,05, 0,1]\\
		\hline
		$(\param{prawdKrzyzowania}, \param{prawdMutacji})$ & [0,55, 0,58, 0,6, 0,62, 0,65] $\times$ [0,05, 0,08, 0,1 0,12, 0,15]  \\
		\hline
		$\param{max}$ & [90, 95, 100, 105, 110] \\
		\hline		
		\multicolumn{2}{c}{}\\
		\hline
		\textbf{Parametr} & \multicolumn{1}{c|}{\textbf{Wartość}} \\
		\hline
		\hline
		$\param{opSelNat}$ & \opName{natSel}(RS)\\
		\hline
		$\param{opSelPlciowej}$ & \opName{stdGenSel}($\bot$, R, R)\\
		\hline
		$R$ & 2\\
		\hline
		$I$ & 3\\
		\hline
		$statystyka$ & najlepsza średnia wyników $I$ powtórzeń \\
		\hline
		Zbiór płci  & $\important{G}=\{ \odot \}$ \\
		\hline
	\end{tabular}
\end{config}

\paragraph{Przebieg}

W tabeli \ref{table:tsp_tweak_flow} przedstawiono przebieg procedury eksploracji, korzystając z tej samej reprezentacji co na poprzednim etapie.

Rozmiar populacji i współczynnik domieszek nie zmienił się względem konfiguracji z pierwszego etapu.
W pierwszym nawrocie określono, że nieznacznie mniejsze (0,58) prawdopodobieństwo krzyżowania i większa liczba pokoleń dają lepsze wyniki.
Drugi nawrót wykazał, że pierwotne prawdopodobieństwo krzyżowania daje lepsze wyniki, za to zwiększone zostało prawdopodobieństwo mutacji. 

\begin{table}[H]
	\caption{Przebieg procedury eksploracji poszukującej parametrów początkowych \label{table:tsp_tweak_flow}}
	\centering
	\begin{tabular}{c|c|c|}
		\cline{2-3}
		& {\bf Parametr}                                     & {\bf Określona wartość} \\ \hline
		\multicolumn{1}{|c|}{\multirow{4}{*}{{\bf Nawrót 1}}} & $\param{rozmiarPopulacji}$                         & 50                      \\ \cline{2-3} 
		\multicolumn{1}{|c|}{}                                & $\param{wspDomieszek}$                             & 0,0                     \\ \cline{2-3} 
		\multicolumn{1}{|c|}{}                                & $(\param{prawdKrzyzowania}, \param{prawdMutacji})$ & (0,58, 0,1)             \\ \cline{2-3} 
		\multicolumn{1}{|c|}{}                                & $\param{max}$                                      & 110                     \\ \hline \hline
		\multicolumn{1}{|c|}{\multirow{4}{*}{{\bf Nawrót 2}}} & $\param{rozmiarPopulacji}$                         & 50                      \\ \cline{2-3} 
		\multicolumn{1}{|c|}{}                                & $\param{wspDomieszek}$                             & 0,0                     \\ \cline{2-3} 
		\multicolumn{1}{|c|}{}                                & $(\param{prawdKrzyzowania}, \param{prawdMutacji})$ & (0,6, 0,15)             \\ \cline{2-3} 
		\multicolumn{1}{|c|}{}                                & $\param{max}$                                      & 110                     \\ \hline
	\end{tabular}
\end{table}	

\paragraph{Wyniki}

W tabeli \ref{table:tsp_tweak_results} przedstawiono wyniki działania heurystyki dla konfiguracji \ref{config:tsp_base} znalezionej w tym etapie.

Średnia ocena jest niższa (a więc lepsza) od średniej z poprzedniego etapu.
Przeciętna różnica między znalezionym rozwiązaniem, a optimum spadła do około 9,5 tysiąca, a odchylenie standardowe nieznacznie (o około pół tysiąca) wzrosło.

Oznacza to, że udało się uzyskać lepsze wyniki kosztem nieznacznego pogorszenia stabilności heurystyki.

\begin{table}[H]
	\caption{Wyniki heurystyki parametryzowanej poprawioną konfiguracją \label{table:tsp_tweak_results}}
	\centering
	\begin{tabular}{|l|r|r}
		\hline
		{\bf Numer powtórzenia}      & \multicolumn{1}{l|}{{\bf Wynik}} & \multicolumn{1}{l|}{{\bf Różnica względem optimum}} \\ \hline \hline
		1                            & 39592,0623                       & \multicolumn{1}{r|}{11989,0623}                     \\ \hline
		2                            & 34292,9277                       & \multicolumn{1}{r|}{6689,9277}                      \\ \hline
		3                            & 37904,1809                       & \multicolumn{1}{r|}{10301,1809}                     \\ \hline \hline
		{\bf Średnia}                & 37263,0570                       & \multicolumn{1}{r|}{9660,0570}                      \\ \hline
		{\bf Odchylenie standardowe} & 2707,1178                        &                                                     \\ \hhline{==~}
		{\bf Optimum}                & {\bf 27603}                      &                                                     \\ \cline{1-2}
	\end{tabular}
\end{table}

\begin{config}
	\caption{Parametry używane w dalszych badaniach \label{config:tsp_base}}
	\centering
	\begin{tabular}{|l|l|}
		\hline
		\textbf{Parametr} & \multicolumn{1}{c|}{\textbf{Wartość}} \\
		\hline
		\hline
		$\param{opSelNat}$ & \opName{natSel}(RS)\\
		\hline
		$\param{opSelPlciowej}$ & \opName{stdGenSel}($\bot$, R, R)\\
		\hline
		$\param{rozmiarPopulacji}$                         & 50                      \\ \hline 
		$\param{wspDomieszek}$                             & 0,0                     \\ \hline
		$\param{prawdKrzyzowania}$ & 0,6 \\ \hline 
		$\param{prawdMutacji}$ & 0,15             \\ \hline
		$\param{max}$                                      & 110                     \\ \hline
\end{tabular}
\end{config}

\subsubsection{Badania skuteczności heurystyki GGA oraz heurystyki DSEA z podobnym operatorem selekcji płciowej}

Ten etap eksperymentów miał na celu odtworzenie wyników heurystyki GGA i porównanie ich z heurystyką DSEA.

\paragraph{Konfiguracja} W tym celu wykorzystano konfigurację \ref{config:tsp_gga} opisującą GGA oraz \ref{table:tsp_dsea_gga} opisująca operator zbliżony do GGA w ten sposób, że korzysta z tego samego operatora wyboru dla obu płci (mimo, że nie wymusza różnych płci rodziców, przez co jest zbliżony do SexualGA).
Każdy z eksperymentów został powtórzony pięciokrotnie.

\begin{config}
	\caption{Konfiguracja heurystyki GGA \label{config:tsp_gga}}
	\begin{tabularx}{\linewidth}{lX}
		\hline
		\multicolumn{1}{|l|}{{\bf Operator selekcji płciowej}} & \multicolumn{1}{l|}{\opName{stdGenSel}($\top$, R, R)} \\ 
		\hline
		\multicolumn{1}{|l|}{{\bf Operator selekcji naturalnej}} & \multicolumn{1}{l|}{\opName{natSel}($W$)} \\ 
		\hline
		\multicolumn{1}{|l|}{{\bf Pozostałe parametry}} & 
		\multicolumn{1}{l|}{Zgodne z konfiguracją \ref{config:tsp_base}} \\ 
		\hline
		\multicolumn{1}{|l|}{{\bf Ilość powtórzeń}} & \multicolumn{1}{l|}{5} \\ 
		\hline
		\multicolumn{1}{|l|}{{\bf Zbiór płci}} & 
		\multicolumn{1}{l|}{$\important{G} = \{ \male, \female \}$} \\ 
		\hline
		& \\
		\hline
		\multicolumn{1}{|c|}{{\bf Parametr}} & 
		\multicolumn{1}{c|}{{\bf Zbiór wartości}} \\ 
		\hline \hline
		\multicolumn{1}{|l|}{$W$} & 
		\multicolumn{1}{l|}{[RS, TS(2), TS(3)]} \\ 
		\hline
	\end{tabularx}
\end{config}

\begin{config}
	%nogender
	\caption{Konfiguracja heurystyki DSEA z operatorem podobnym do GGA \label{config:tsp_dsea_gga}}
	\begin{tabularx}{\linewidth}{lX}
		\hline
		\multicolumn{1}{|l|}{{\bf Operator selekcji płciowej}} &
		\multicolumn{1}{l|}{\opName{stdGenSel}($\bot$, $W$, $W$)} \\ 
		\hline
		\multicolumn{1}{|l|}{{\bf Pozostałe parametry}} & 
		\multicolumn{1}{l|}{Zgodne z konfiguracją \ref{config:tsp_base}} \\ 
		\hline
		\multicolumn{1}{|l|}{{\bf Ilość powtórzeń}} &
		\multicolumn{1}{l|}{5} \\ 
		\hline
		\multicolumn{1}{|l|}{{\bf Zbiór płci}} & 
		\multicolumn{1}{l|}{$\important{G} = \{ \odot \}$} \\ 
		\hline
		\hline
		& \\ 
		\hline
		\multicolumn{1}{|c|}{{\bf Parametr}} & 
		\multicolumn{1}{c|}{{\bf Zbiór wartości}} \\ 
		\hline \hline
		\multicolumn{1}{|l|}{$W$} & 
		\multicolumn{1}{l|}{[R, RS, TS(2), TS(3)]} \\ 
		\hline
	\end{tabularx}
\end{config}

\paragraph{Wyniki} Tabela \ref{table:tsp_results_gga} przedstawia średnią i odchylenie standardowe wyników heurystyki GGA, a tabela \ref{table:tsp_dsea_gga} analogiczne wartości dla heurystyki DSEA.
Drugi wiersz drugiej z tej tabel przedstawia wyniki dla standardowego algorytmu ewolucyjnego.
Jak widać, jedynie użycie operatora ruletkowego poprawiło względem niego wyniki. 
Może to wynikać z tego, że wykorzystano również operator selekcji naturalnej korzystający z ruletki.

Pozostałe wyniki są znacznie gorsze niż dwa wymienione wyżej, chociaż w przypadku heurystyki DSEA udało się w ogólności poprawić stabilność.
W ogólności, heurystyka GGA dawała gorsze wyniki.

\begin{table}[h]
	\caption{Wyniki heurystyki GGA \label{table:tsp_results_gga}}
	\centering
	\begin{tabular}{|l|r@{$\pm$}l|}
		\hline
		\multicolumn{1}{|c|}{{\bf W}} & \multicolumn{2}{c|}{{\bf Ocena}} \\ \hline \hline
		\insertData{tsp_gga}
	\end{tabular}
\end{table}

\begin{table}[h]
	\caption{Wyniki heurystyki DSEA z operatorem podobnym do GGA \label{table:tsp_dsea_gga}}
	\centering
	\begin{tabular}{|l|r@{$\pm$}l|}
		\hline
		\multicolumn{1}{|c|}{{\bf W}} & 
		\multicolumn{2}{c|}{{\bf Ocena}} \\ 
		\hline \hline
		\insertData{tsp_a}
	\end{tabular}
\end{table}

\subsubsection{Badania skuteczności heurystyki SexualGA oraz heurystyki DSEA z podobnym operatorem selekcji płciowej}

Celem kolejnego etapu eksperymentów było odtworzenie wyników działania heurystyki SexualGA i zbadanie działania operatora selekcji płciowej zbliżonego do niej.

\paragraph{Konfiguracja} W tym celu określono trzy konfiguracje: \ref{config:tsp_sexual_ga} opisującą SexualGA, \ref{config:tsp_dsea_sexual_ga_false} zbliżonej do poprzedniej, jednak wykorzystującej ruletkowy operator selekcji płciowej, oraz \ref{config:tsp_dsea_sexual_ga_true} różniącej się od poprzedniej tym, że operator selekcji płciowej wymagał od rodziców tego, żeby byli różnej płci (przez co był zbliżony do GGA).

\begin{config}
	\caption{Konfiguracja heurystyki SexualGA \label{config:tsp_sexual_ga}}
	\centering
	\begin{tabularx}{\linewidth}{lX}
		\hline
		\multicolumn{1}{|l|}{{\bf Operator selekcji płciowej}} &
		\multicolumn{1}{l|}{\opName{stdGenSel}($\bot$, $\param{opWyboru1}$, $\param{opWyboru2}$)} \\ 
		\hline
		\multicolumn{1}{|l|}{{\bf Operator selekcji naturalnej}} &
		\multicolumn{1}{l|}{\opName{natSel}(R)} \\ 
		\hline
		\multicolumn{1}{|l|}{{\bf Pozostałe parametry}} & 
		\multicolumn{1}{l|}{Zgodne z konfiguracją \ref{config:tsp_base}} \\ 
		\hline
		\multicolumn{1}{|l|}{{\bf Ilość powtórzeń}} &
		\multicolumn{1}{l|}{5} \\ 
		\hline
		\multicolumn{1}{|l|}{{\bf Zbiór płci}} & 
		\multicolumn{1}{l|}{$\important{G} = \{ \odot \}$} \\ 
		\hline
		& \\ 
		\hline
		\multicolumn{1}{|c|}{{\bf Parametr}} & 
		\multicolumn{1}{c|}{{\bf Zbiór wartości}} \\ 
		\hline \hline
		\multicolumn{1}{|l|}{$\param{opWyboru1}$} & 
		\multicolumn{1}{l|}{[TS(2), TS(3)]} \\ 
		\hline
		\multicolumn{1}{|l|}{$\param{opWyboru2}$} & 
		\multicolumn{1}{l|}{[RS, TS(2), TS(3)]} \\ 
		\hline
	\end{tabularx}
\end{config}

\begin{config}
	%gender, false
	\caption{Konfiguracja heurystyki DSEA z jedną płcią z operatorem podobnym do SexualGA \label{config:tsp_dsea_sexual_ga_false}}
	\centering
	\begin{tabularx}{\linewidth}{lX}
		\hline
		\multicolumn{1}{|l|}{{\bf Operator selekcji płciowej}} &
		\multicolumn{1}{l|}{\opName{stdGenSel}($\bot$, $\param{opWyboru1}$, $\param{opWyboru2}$)} \\ 
		\hline
		\multicolumn{1}{|l|}{{\bf Pozostałe parametry}} &
		\multicolumn{1}{l|}{Zgodne z konfiguracją \ref{config:tsp_base}} \\ 
		\hline
		\multicolumn{1}{|l|}{{\bf Ilość powtórzeń}} &
		\multicolumn{1}{l|}{5} \\ 
		\hline
		\multicolumn{1}{|l|}{{\bf Zbiór płci}} & 
		\multicolumn{1}{l|}{$\important{G} = \{ \odot \}$} \\ 
		\hline
		& \\ 
		\hline
		\multicolumn{1}{|c|}{{\bf Parametr}} & 
		\multicolumn{1}{c|}{{\bf Zbiór wartości}} \\ 
		\hline \hline
		\multicolumn{1}{|l|}{$\param{opWyboru1}$} & 
		\multicolumn{1}{l|}{[RS, TS(2), TS(3)]} \\ 
		\hline
		\multicolumn{1}{|l|}{$\param{opWyboru2}$} & 
		\multicolumn{1}{l|}{[R, RS, TS(2), TS(3)]} \\
		\hline
	\end{tabularx}
\end{config}

\begin{config}
	%gender, true
	\caption{Konfiguracja heurystyki DSEA z dwoma płciami z operatorem podobnym do SexualGA \label{config:tsp_dsea_sexual_ga_true}}
	\centering
	\begin{tabularx}{\linewidth}{lX}
		\hline
		\multicolumn{1}{|l|}{{\bf Operator selekcji płciowej}} &
		\multicolumn{1}{l|}{\opName{stdGenSel}($\top$, $\param{opWyboru1}$, $\param{opWyboru2}$)} \\ 
		\hline
		\multicolumn{1}{|l|}{{\bf Pozostałe parametry}} &
		\multicolumn{1}{l|}{Zgodne z konfiguracją \ref{config:tsp_base}} \\ 
		\hline
		\multicolumn{1}{|l|}{{\bf Ilość powtórzeń}} &
		\multicolumn{1}{l|}{5} \\ 
		\hline
		\multicolumn{1}{|l|}{{\bf Zbiór płci}} & 
		\multicolumn{1}{l|}{$\important{G} = \{ \male, \female \}$} \\ 
		\hline
		& \\ 
		\hline
		\multicolumn{1}{|c|}{{\bf Parametr}} & 
		\multicolumn{1}{c|}{{\bf Zbiór wartości}} \\ 
		\hline \hline
		\multicolumn{1}{|l|}{$\param{opWyboru1}$} & 
		\multicolumn{1}{l|}{[RS, TS(2), TS(3)]} \\ 
		\hline
		\multicolumn{1}{|l|}{$\param{opWyboru2}$} & 
		\multicolumn{1}{l|}{[R, RS, TS(2), TS(3)]} \\
		\hline
	\end{tabularx}
\end{config}

\paragraph{Wyniki} \todo{SexualGA + natsel rox, bez natsel do niczego, dwie płcie nieznacznie gorsze, ogólnie gorsze niż tabela 6, lepsze niż GGA}

\begin{table}[H]
	\caption{Wyniki heurystyki SexualGA \label{table:tsp_results_sexual_ga}}
	\centering
	\begin{tabular}{|l|l|r@{$\pm$}l|}
		\hline
		\multicolumn{1}{|c|}{{\bf $\param{opWyboru1}$}} &
		\multicolumn{1}{c|}{{\bf $\param{opWyboru2}$}} & 
		\multicolumn{2}{c|}{{\bf Ocena}} \\ 
		\hline \hline
		\insertData{tsp_sexual_ga}
	\end{tabular}
\end{table}

\begin{table}[H]
	\caption{Wyniki heurystyki DSEA z jedną płcią, z operatorem podobnym do SexualGA \label{table:tsp_results_dsea_sexual_ga_false}}
	\centering
	\begin{tabular}{|l|l|r@{$\pm$}l|}
		\hline
		\multicolumn{1}{|c|}{{\bf $\param{opWyboru1}$}} &
		\multicolumn{1}{c|}{{\bf $\param{opWyboru2}$}} & 
		\multicolumn{2}{c|}{{\bf Ocena}} \\ 
		\hline \hline
		\insertData{tsp_c}
	\end{tabular}
\end{table}

\begin{table}[H]
	\caption{Wyniki heurystyki DSEA z dwoma płciami, z operatorem podobnym do SexualGA \label{table:tsp_results_dsea_sexual_ga_true}}
	\centering
	\begin{tabular}{|l|l|r@{$\pm$}l|}
		\hline
		\multicolumn{1}{|c|}{{\bf $\param{opWyboru1}$}} & 
		\multicolumn{1}{c|}{{\bf $\param{opWyboru2}$}} & 
		\multicolumn{2}{c|}{{\bf Ocena}} \\ 
		\hline \hline
		\insertData{tsp_b}
	\end{tabular}
\end{table}

\subsubsection{Badania skuteczności heurystyki DSEA z haremowym operatorem selekcji płciowej}

\paragraph{Konfiguracja} \todo{blah}

\begin{config}
	%harem
	\caption{Konfiguracja heurystyki DSEA z operatorem haremowym \label{table:tsp_dsea_harem}}
	\begin{tabularx}{\linewidth}{lX}
		\hline
		\multicolumn{1}{|l|}{{\bf Operator selekcji płciowej}} &
		\multicolumn{1}{l|}{\opName{harem}(a, b, WA, WB, WP)} \\ 
		\hline
		\multicolumn{1}{|l|}{{\bf Pozostałe parametry}} &
		\multicolumn{1}{l|}{Zgodne z konfiguracją \ref{config:tsp_base}} \\ 
		\hline
		\multicolumn{1}{|l|}{{\bf Ilość powtórzeń}} &
		\multicolumn{1}{l|}{5} \\ 
		\hline
		\multicolumn{1}{|l|}{{\bf Zbiór płci}} & 
		\multicolumn{1}{l|}{$\important{G} = \{ \male, \female \}$} \\ 
		\hline
		& \\ 
		\hline
		\multicolumn{1}{|c|}{{\bf Parametr}} & 
		\multicolumn{1}{c|}{{\bf Zbiór wartości}} \\ 
		\hline \hline
		\multicolumn{1}{|l|}{a} & 
		\multicolumn{1}{l|}{{[}1, 3, 5{]}} \\ 
		\hline
		\multicolumn{1}{|l|}{b} & 
		\multicolumn{1}{l|}{[0, 0,1, 0,25]} \\ 
		\hline
		\multicolumn{1}{|l|}{WA} & 
		\multicolumn{1}{l|}{[R, RS, TS(2), TS(3)]} \\ 
		\hline
		\multicolumn{1}{|l|}{WB} & 
		\multicolumn{1}{l|}{[R, RS, TS(2), TS(3)]} \\ 
		\hline
		& \\ 
		\hline
		\multicolumn{1}{|c|}{{\bf Parametr}} & 
		\multicolumn{1}{c|}{{\bf Wartość}} \\ 
		\hline \hline
		\multicolumn{1}{|l|}{WP} &
		\multicolumn{1}{l|}{R} \\ 
		\hline
	\end{tabularx}
\end{config}

\begin{table}[h]
	\caption{Wyniki heurystyki DSEA z operatorem haremowym \label{table:tsp_results_dsea_harem}}
	\centering
	\begin{tabular}{|l|l|l|l|r@{$\pm$}l|}
		\hline
		\multicolumn{1}{|c|}{{\bf a}} & \multicolumn{1}{|c|}{{\bf b}} & \multicolumn{1}{|c|}{{\bf WA}} & \multicolumn{1}{c|}{{\bf WB}} & \multicolumn{2}{c|}{{\bf Ocena}} \\ \hline \hline
		\insertData{tsp_d_top}
	\end{tabular}	
\end{table}

\subsubsection{Podsumowanie}

\todo{proza, co wyszło, a co nie}

\subsection{Badania problemu plecakowego}


W następnych podsekcjach zostaną opisane kolejne kroki badań działania DSEA dla problemu komiwojażera, zdefiniowanego i zaimplementowanego zgodnie z podsekcją \ref{subsection:knapsack_impl}.

\subsubsection{Poszukiwanie parametrów początkowych}

\paragraph{Konfiguracja}

W tym kroku wykorzystano procedurę eksploracji w celu znalezienia konfiguracji wyjściowej. W tabeli \ref{config:knapsack_init_params} przedstawiono zbiory wartości parametrów heurystyki i procedury eksploracji użyte w tym kroku.

\begin{config}
	\caption{Wartości wykorzystane podczas poszukiwania parametrów początkowych \label{config:knapsack_init_params}}
	\centering
	\begin{tabular}{|c|l|}
		\hline
		\textbf{Parametr} & \multicolumn{1}{c|}{\textbf{Zbiór wartości}} \\
		\hline
		\hline
		$\param{rozmiarPopulacji}$ & [10, 20, 50] \\
		\hline
		$\param{wspDomieszek}$ & [0, 0,1, 0,25, 0,5]\\
		\hline
		$(\param{prawdKrzyzowania}, \param{prawdMutacji})$ & [0,6, 0,7, 0,8] $\times$ [0,1, 0,2, 0,3]  \\
		\hline
		$\param{max}$ & [25, 50, 100] \\
		\hline		
		$\param{opSelNat}$ & [\opName{natSel}(RS), \opName{natSel}(TS(2)), \opName{natSel}(TS(3))]\\
		\hline
		\multicolumn{2}{c}{}\\
		\hline
		\textbf{Parametr} & \multicolumn{1}{c|}{\textbf{Wartość}} \\
		\hline
		\hline
		$\param{opSelPlciowej}$ & \opName{stdGenSel}($\bot$, R, R)\\
		\hline
		$R$ & 3\\
		\hline
		$I$ & 5\\
		\hline
		$statystyka$ & najlepszy wynik \\
		\hline
		Zbiór płci  & $\important{G}=\{ \odot \}$ \\
		\hline
	\end{tabular}
\end{config}

\paragraph{Przebieg}

Tabela \ref{table:knapsack_init_flow} przedstawia przebieg wykorzystanej procedury. 
Lewa kolumna grupuje znalezione parametry w ramach jednego nawrotu eksploracji. 
Kolejne dwie, czytane od góry, określają jakie wartości zostały kolejno znalezione.

Jak widać, w pierwszym nawrocie znaleziono prawie najlepszą konfigurację. 
W drugim, udało się odnaleźć prawdopodobieństwa mutacji i krzyżowania które dawały lepsze wyniki.
Trzeci nawrót potwierdził, że znaleziona konfiguracja jest najlepsza z przeszukiwanych.

\begin{table}[H]
	\caption{Przebieg procedury eksploracji poszukującej parametrów początkowych \label{table:knapsack_init_flow}}
	\centering
	\begin{tabular}{c|c|c|}
		\cline{2-3}
		\multicolumn{1}{l|}{}                                 & {\bf Parametr}                                     & {\bf Określona wartość} \\ \hline
		\multicolumn{1}{|c|}{\multirow{5}{*}{{\bf Nawrót 1}}} & $\param{rozmiarPopulacji}$                         & 50                      \\ \cline{2-3} 
		\multicolumn{1}{|c|}{}                                & $\param{wspDomieszek}$                             & 0,25                     \\ \cline{2-3} 
		\multicolumn{1}{|c|}{}                                & $(\param{prawdKrzyzowania}, \param{prawdMutacji})$ & (0,6, 0,1)              \\ \cline{2-3} 
		\multicolumn{1}{|c|}{}                                & $\param{max}$                                      & 50                     \\ \cline{2-3} 
		\multicolumn{1}{|c|}{}                                & $\param{opSelNat}$                                 & \opName{natSel}(RS)                \\ \hline
		\hline
		\multicolumn{1}{|c|}{\multirow{5}{*}{{\bf Nawrót 2}}} & $\param{rozmiarPopulacji}$                         & 50                      \\ \cline{2-3} 
		\multicolumn{1}{|c|}{}                                & $\param{wspDomieszek}$                             & 0,1                     \\ \cline{2-3} 
		\multicolumn{1}{|c|}{}                                & $(\param{prawdMutacji}, \param{prawdKrzyzowania})$ & (0,8, 0,1)              \\ \cline{2-3} 
		\multicolumn{1}{|c|}{}                                & $\param{max}$                                      & 50                     \\ \cline{2-3} 
		\multicolumn{1}{|c|}{}                                & $\param{opSelNat}$                                 & \opName{natSel}(RS)                \\ \hline
		\hline
		\multicolumn{1}{|c|}{\multirow{5}{*}{{\bf Nawrót 3}}} & $\param{rozmiarPopulacji}$                         & 50                      \\ \cline{2-3} 
		\multicolumn{1}{|c|}{}                                & $\param{wspDomieszek}$                             & 0,1                     \\ \cline{2-3} 
		\multicolumn{1}{|c|}{}                                & $(\param{prawdMutacji}, \param{prawdKrzyzowania})$ & (0,8, 0,1)              \\ \cline{2-3} 
		\multicolumn{1}{|c|}{}                                & $\param{max}$                                      & 50                     \\ \cline{2-3} 
		\multicolumn{1}{|c|}{}                                & $\param{opSelNat}$                                 & \opName{natSel}(RS)                \\ \hline
		
	\end{tabular}
\end{table}

\paragraph{Wyniki}

\todo{Blah, blah, here go results}

\begin{table}[H]
	\caption{Wyniki heurystyki parametryzowanej znalezioną konfiguracją \label{knapsack_init_results}}
	\centering
	\begin{tabular}{|l|r|r}
		\hline
		\multicolumn{1}{|c|}{{\bf Numer powtórzenia}} & \multicolumn{1}{c|}{{\bf Wynik}} & \multicolumn{1}{c|}{{\bf Różnica względem optimum}} \\ \hline \hline
		1                                             & -1107                            & \multicolumn{1}{r|}{66}                             \\ \hline
		2                                             & -1156                            & \multicolumn{1}{r|}{17}                             \\ \hline
		3                                             & -1115                            & \multicolumn{1}{r|}{58}                             \\ \hline
		4                                             & -1120                            & \multicolumn{1}{r|}{53}                             \\ \hline
		5                                             & -1173                            & \multicolumn{1}{r|}{0}                              \\ \hline \hline
		{\bf Średnia}                                 & -1134,2000                          & \multicolumn{1}{r|}{38,8000}                           \\ \hline
		{\bf Odchylenie standardowe}                  & 28,6827                          & \multicolumn{1}{l}{}                                \\ \hhline{==~}
		{\bf Optimum}                                 & {\bf -1173}                      & \multicolumn{1}{l}{}                                \\ \cline{1-2}
	\end{tabular}
\end{table}

\begin{figure}[H]
	\caption{Wykres przebiegu piątego powtórzenia znalezionej konfiguracji \label{figure:knapsack_init_example}}
	\centering
	\graph{knapsack_init_example.tex}
\end{figure}

\subsubsection{Poprawa konfiguracji początkowej}

\paragraph{Konfiguracja}

\todo{opisać skąd sie wzięły wartości}

\begin{config}
	\caption{Wartości wykorzystane podczas poprawy parametrów początkowych \label{config:knapsack_tweak_params}}
	\centering
	\begin{tabular}{|c|l|}
		\hline
		\textbf{Parametr} & \multicolumn{1}{c|}{\textbf{Zbiór wartości}} \\
		\hline
		\hline
		$\param{rozmiarPopulacji}$ & [40, 45, 50, 55, 60] \\
		\hline
		$\param{wspDomieszek}$ & [0, 0,05, 0,1, 0,15, 0,2]\\
		\hline
		$(\param{prawdKrzyzowania}, \param{prawdMutacji})$ & [0,75, 0,78, 0,8, 0,82, 0,85] $\times$ [0,05, 0,08, 0,1 0,12, 0,15]  \\
		\hline
		$\param{max}$ & [40, 45, 50, 55, 60] \\
		\hline		
		\multicolumn{2}{c}{}\\
		\hline
		\textbf{Parametr} & \multicolumn{1}{c|}{\textbf{Wartość}} \\
		\hline
		\hline
		$\param{opSelNat}$ & \opName{natSel}(RS)\\
		\hline
		$\param{opSelPlciowej}$ & \opName{stdGenSel}($\bot$, R, R)\\
		\hline
		$R$ & 2\\
		\hline
		$I$ & 3\\
		\hline
		$statystyka$ & najlepsza średnia wyników $I$ powtórzeń \\
		\hline
		Zbiór płci  & $\important{G}=\{ \odot \}$ \\
		\hline
	\end{tabular}
\end{config}

\paragraph{Przebieg}

\todo{skomentuj}

\begin{table}[H]
	\caption{Przebieg procedury eksploracji poszukującej parametrów początkowych \label{table:knapsack_tweak_flow}}
	\centering
	\begin{tabular}{c|c|c|}
		\cline{2-3}
		& {\bf Parametr}                                     & {\bf Określona wartość} \\ \hline
		\multicolumn{1}{|c|}{\multirow{4}{*}{{\bf Nawrót 1}}} & $\param{rozmiarPopulacji}$                         & 60                      \\ \cline{2-3} 
		\multicolumn{1}{|c|}{}                                & $\param{wspDomieszek}$                             & 0,1                     \\ \cline{2-3} 
		\multicolumn{1}{|c|}{}                                & $(\param{prawdKrzyzowania}, \param{prawdMutacji})$ & (0,75, 0,1)             \\ \cline{2-3} 
		\multicolumn{1}{|c|}{}                                & $\param{max}$                                      & 50                     \\ \hline \hline
		\multicolumn{1}{|c|}{\multirow{4}{*}{{\bf Nawrót 2}}} & $\param{rozmiarPopulacji}$                         & 60                      \\ \cline{2-3} 
		\multicolumn{1}{|c|}{}                                & $\param{wspDomieszek}$                             & 0,1                     \\ \cline{2-3} 
		\multicolumn{1}{|c|}{}                                & $(\param{prawdKrzyzowania}, \param{prawdMutacji})$ & (0,75, 0,15)             \\ \cline{2-3} 
		\multicolumn{1}{|c|}{}                                & $\param{max}$                                      & 60                     \\ \hline
	\end{tabular}
\end{table}	

\paragraph{Wyniki}
\todo{Blah, blah, here go results}

\begin{table}[H]
	\caption{Wyniki heurystyki parametryzowanej poprawioną konfiguracją \label{knapsack_tweak_results}}
	\centering
	\begin{tabular}{|l|r|r}
		\hline
		{\bf Numer powtórzenia}      & \multicolumn{1}{l|}{{\bf Wynik}} & \multicolumn{1}{l|}{{\bf Różnica względem optimum}} \\ \hline \hline
		1                                             & -1127                            & \multicolumn{1}{r|}{46}                             \\ \hline
		2                                             & -1161                            & \multicolumn{1}{r|}{12}                             \\ \hline
		3                                             & -1131                            & \multicolumn{1}{r|}{42}                             \\ \hline \hline
		{\bf Średnia}                                 & -1139,6667                       & \multicolumn{1}{r|}{33,3334}                        \\ \hline
		{\bf Odchylenie standardowe}                  & 18,5831                          &                                                     \\ \hhline{==~}
		{\bf Optimum}                                 & {\bf -1173}                      &                                                     \\ \cline{1-2}
	\end{tabular}
\end{table}

\todo{poszło lepiej niż init}

\begin{config}
	\caption{Parametry używane w dalszych badaniach \label{config:knapsack_base}}
	\centering
	\begin{tabular}{|l|l|}
		\hline
		\textbf{Parametr} & \multicolumn{1}{c|}{\textbf{Wartość}} \\
		\hline
		\hline
		$\param{opSelNat}$ & \opName{natSel}(RS)\\
		\hline
		$\param{rozmiarPopulacji}$                         & 60                      \\ \hline 
		$\param{wspDomieszek}$                             & 0,1                    \\ \hline
		$\param{prawdKrzyzowania}$ & 0,75 \\ \hline 
		$\param{prawdMutacji}$ & 0,15             \\ \hline
		$\param{max}$                                      & 60                     \\ \hline
	\end{tabular}
\end{config}

\subsubsection{Badania skuteczności heurystyki GGA oraz heurystyki DSEA z podobnym operatorem selekcji płciowej}

\paragraph{Konfiguracja} \todo{blah}

\begin{config}
	\caption{Konfiguracja heurystyki GGA \label{config:knapsack_gga}}
	\begin{tabularx}{\linewidth}{lX}
		\hline
		\multicolumn{1}{|l|}{{\bf Operator selekcji płciowej}} & \multicolumn{1}{l|}{\opName{stdGenSel}($\top$, R, R)} \\ 
		\hline
		\multicolumn{1}{|l|}{{\bf Operator selekcji naturalnej}} & \multicolumn{1}{l|}{\opName{natSel}($W$)} \\ 
		\hline
		\multicolumn{1}{|l|}{{\bf Pozostałe parametry}} & 
		\multicolumn{1}{l|}{Zgodne z konfiguracją \ref{config:knapsack_base}} \\ 
		\hline
		\multicolumn{1}{|l|}{{\bf Ilość powtórzeń}} & \multicolumn{1}{l|}{5} \\ 
		\hline
		\multicolumn{1}{|l|}{{\bf Zbiór płci}} & 
		\multicolumn{1}{l|}{$\important{G} = \{ \male, \female \}$} \\ 
		\hline
		& \\
		\hline
		\multicolumn{1}{|c|}{{\bf Parametr}} & 
		\multicolumn{1}{c|}{{\bf Zbiór wartości}} \\ 
		\hline \hline
		\multicolumn{1}{|l|}{$W$} & 
		\multicolumn{1}{l|}{[RS, TS(2), TS(3)]} \\ 
		\hline
	\end{tabularx}
\end{config}

\begin{config}
	%nogender
	\caption{Konfiguracja heurystyki DSEA z operatorem podobnym do GGA \label{config:knapsack_dsea_gga}}
	\begin{tabularx}{\linewidth}{lX}
		\hline
		\multicolumn{1}{|l|}{{\bf Operator selekcji płciowej}} &
		\multicolumn{1}{l|}{\opName{stdGenSel}($\bot$, $W$, $W$)} \\ 
		\hline
		\multicolumn{1}{|l|}{{\bf Pozostałe parametry}} & 
		\multicolumn{1}{l|}{Zgodne z konfiguracją \ref{config:knapsack_base}} \\ 
		\hline
		\multicolumn{1}{|l|}{{\bf Ilość powtórzeń}} &
		\multicolumn{1}{l|}{5} \\ 
		\hline
		\multicolumn{1}{|l|}{{\bf Zbiór płci}} & 
		\multicolumn{1}{l|}{$\important{G} = \{ \odot \}$} \\ 
		\hline
		\hline
		& \\ 
		\hline
		\multicolumn{1}{|c|}{{\bf Parametr}} & 
		\multicolumn{1}{c|}{{\bf Zbiór wartości}} \\ 
		\hline \hline
		\multicolumn{1}{|l|}{$W$} & 
		\multicolumn{1}{l|}{[R, RS, TS(2), TS(3)]} \\ 
		\hline
	\end{tabularx}
\end{config}

\paragraph{Wyniki} \todo{bum bada bim}

\begin{table}[h]
	\caption{Wyniki heurystyki GGA \label{table:knapsack_results_gga}}
	\centering
	\begin{tabular}{|l|r@{$\pm$}l|}
		\hline
		\multicolumn{1}{|c|}{{\bf W}} & \multicolumn{2}{c|}{{\bf Ocena}} \\ \hline \hline
		\insertData{knapsack_gga}
	\end{tabular}
\end{table}

\begin{table}[h]
	\caption{Wyniki heurystyki DSEA z operatorem podobnym do GGA \label{table:knapsack_dsea_gga}}
	\centering
	\begin{tabular}{|l|r@{$\pm$}l|}
		\hline
		\multicolumn{1}{|c|}{{\bf W}} & 
		\multicolumn{2}{c|}{{\bf Ocena}} \\ 
		\hline \hline
		\insertData{knapsack_a}
	\end{tabular}
\end{table}

\subsubsection{Badania skuteczności heurystyki SexualGA oraz heurystyki DSEA z podobnym operatorem selekcji płciowej}

\paragraph{Konfiguracja} \todo{blah}

\begin{config}
	\caption{Konfiguracja heurystyki SexualGA \label{config:knapsack_sexual_ga}}
	\centering
	\begin{tabularx}{\linewidth}{lX}
		\hline
		\multicolumn{1}{|l|}{{\bf Operator selekcji płciowej}} &
		\multicolumn{1}{l|}{\opName{stdGenSel}($\bot$, $\param{opWyboru1}$, $\param{opWyboru2}$)} \\ 
		\hline
		\multicolumn{1}{|l|}{{\bf Operator selekcji naturalnej}} &
		\multicolumn{1}{l|}{\opName{natSel}(R)} \\ 
		\hline
		\multicolumn{1}{|l|}{{\bf Pozostałe parametry}} & 
		\multicolumn{1}{l|}{Zgodne z konfiguracją \ref{config:knapsack_base}} \\ 
		\hline
		\multicolumn{1}{|l|}{{\bf Ilość powtórzeń}} &
		\multicolumn{1}{l|}{5} \\ 
		\hline
		\multicolumn{1}{|l|}{{\bf Zbiór płci}} & 
		\multicolumn{1}{l|}{$\important{G} = \{ \odot \}$} \\ 
		\hline
		& \\ 
		\hline
		\multicolumn{1}{|c|}{{\bf Parametr}} & 
		\multicolumn{1}{c|}{{\bf Zbiór wartości}} \\ 
		\hline \hline
		\multicolumn{1}{|l|}{$\param{opWyboru1}$} & 
		\multicolumn{1}{l|}{[TS(2), TS(3)]} \\ 
		\hline
		\multicolumn{1}{|l|}{$\param{opWyboru2}$} & 
		\multicolumn{1}{l|}{[RS, TS(2), TS(3)]} \\ 
		\hline
	\end{tabularx}
\end{config}

\begin{config}
	%gender, false
	\caption{Konfiguracja heurystyki DSEA z jedną płcią z operatorem podobnym do SexualGA \label{config:knapsack_dsea_sexual_ga_false}}
	\centering
	\begin{tabularx}{\linewidth}{lX}
		\hline
		\multicolumn{1}{|l|}{{\bf Operator selekcji płciowej}} &
		\multicolumn{1}{l|}{\opName{stdGenSel}($\bot$, $\param{opWyboru1}$, $\param{opWyboru2}$)} \\ 
		\hline
		\multicolumn{1}{|l|}{{\bf Pozostałe parametry}} &
		\multicolumn{1}{l|}{Zgodne z konfiguracją \ref{config:knapsack_base}} \\ 
		\hline
		\multicolumn{1}{|l|}{{\bf Ilość powtórzeń}} &
		\multicolumn{1}{l|}{5} \\ 
		\hline
		\multicolumn{1}{|l|}{{\bf Zbiór płci}} & 
		\multicolumn{1}{l|}{$\important{G} = \{ \odot \}$} \\ 
		\hline
		& \\ 
		\hline
		\multicolumn{1}{|c|}{{\bf Parametr}} & 
		\multicolumn{1}{c|}{{\bf Zbiór wartości}} \\ 
		\hline \hline
		\multicolumn{1}{|l|}{$\param{opWyboru1}$} & 
		\multicolumn{1}{l|}{[RS, TS(2), TS(3)]} \\ 
		\hline
		\multicolumn{1}{|l|}{$\param{opWyboru2}$} & 
		\multicolumn{1}{l|}{[R, RS, TS(2), TS(3)]} \\
		\hline
	\end{tabularx}
\end{config}

\begin{config}
	%gender, false
	\caption{Konfiguracja heurystyki DSEA z dwoma płciami z operatorem podobnym do SexualGA \label{config:knapsack_dsea_sexual_ga_true}}
	\centering
	\begin{tabularx}{\linewidth}{lX}
		\hline
		\multicolumn{1}{|l|}{{\bf Operator selekcji płciowej}} &
		\multicolumn{1}{l|}{\opName{stdGenSel}($\top$, $\param{opWyboru1}$, $\param{opWyboru2}$)} \\ 
		\hline
		\multicolumn{1}{|l|}{{\bf Pozostałe parametry}} &
		\multicolumn{1}{l|}{Zgodne z konfiguracją \ref{config:knapsack_base}} \\ 
		\hline
		\multicolumn{1}{|l|}{{\bf Ilość powtórzeń}} &
		\multicolumn{1}{l|}{5} \\ 
		\hline
		\multicolumn{1}{|l|}{{\bf Zbiór płci}} & 
		\multicolumn{1}{l|}{$\important{G} = \{ \male, \female \}$} \\ 
		\hline
		& \\ 
		\hline
		\multicolumn{1}{|c|}{{\bf Parametr}} & 
		\multicolumn{1}{c|}{{\bf Zbiór wartości}} \\ 
		\hline \hline
		\multicolumn{1}{|l|}{$\param{opWyboru1}$} & 
		\multicolumn{1}{l|}{[RS, TS(2), TS(3)]} \\ 
		\hline
		\multicolumn{1}{|l|}{$\param{opWyboru2}$} & 
		\multicolumn{1}{l|}{[R, RS, TS(2), TS(3)]} \\
		\hline
	\end{tabularx}
\end{config}

\paragraph{Wyniki} \todo{tam da dam}

\begin{table}[h]
	\caption{Wyniki heurystyki DSEA z jedną płcią, z operatorem podobnym do SexualGA \label{table:knapsack_results_dsea_sexual_ga_false}}
	\centering
	\begin{tabular}{|l|l|r@{$\pm$}l|}
		\hline
		\multicolumn{1}{|c|}{{\bf $\param{opWyboru1}$}} &
		\multicolumn{1}{c|}{{\bf $\param{opWyboru2}$}} & 
		\multicolumn{2}{c|}{{\bf Ocena}} \\ 
		\hline \hline
		\insertData{knapsack_c}
	\end{tabular}
\end{table}

\begin{table}[h]
	\caption{Wyniki heurystyki DSEA z dwoma płciami, z operatorem podobnym do SexualGA \label{table:knapsack_results_dsea_sexual_ga_true}}
	\centering
	\begin{tabular}{|l|l|r@{$\pm$}l|}
		\hline
		\multicolumn{1}{|c|}{{\bf $\param{opWyboru1}$}} & 
		\multicolumn{1}{c|}{{\bf $\param{opWyboru2}$}} & 
		\multicolumn{2}{c|}{{\bf Ocena}} \\ 
		\hline \hline
		\insertData{knapsack_b}
	\end{tabular}
\end{table}

\subsubsection{Badania skuteczności heurystyki DSEA z haremowym operatorem selekcji płciowej}

\paragraph{Konfiguracja} \todo{blah}

\begin{config}
	%harem
	\caption{Konfiguracja heurystyki DSEA z operatorem haremowym \label{table:knapsack_dsea_harem}}
	\begin{tabularx}{\linewidth}{lX}
		\hline
		\multicolumn{1}{|l|}{{\bf Operator selekcji płciowej}} &
		\multicolumn{1}{l|}{\opName{harem}(a, b, WA, WB, WP)} \\ 
		\hline
		\multicolumn{1}{|l|}{{\bf Pozostałe parametry}} &
		\multicolumn{1}{l|}{Zgodne z konfiguracją \ref{config:knapsack_base}} \\ 
		\hline
		\multicolumn{1}{|l|}{{\bf Ilość powtórzeń}} &
		\multicolumn{1}{l|}{5} \\ 
		\hline
		\multicolumn{1}{|l|}{{\bf Zbiór płci}} & 
		\multicolumn{1}{l|}{$\important{G} = \{ \male, \female \}$} \\ 
		\hline
		& \\ 
		\hline
		\multicolumn{1}{|c|}{{\bf Parametr}} & 
		\multicolumn{1}{c|}{{\bf Zbiór wartości}} \\ 
		\hline \hline
		\multicolumn{1}{|l|}{a} & 
		\multicolumn{1}{l|}{{[}1, 3, 5{]}} \\ 
		\hline
		\multicolumn{1}{|l|}{b} & 
		\multicolumn{1}{l|}{[0, 0,1, 0,25]} \\ 
		\hline
		\multicolumn{1}{|l|}{WA} & 
		\multicolumn{1}{l|}{[R, RS, TS(2), TS(3)]} \\ 
		\hline
		\multicolumn{1}{|l|}{WB} & 
		\multicolumn{1}{l|}{[R, RS, TS(2), TS(3)]} \\ 
		\hline
		& \\ 
		\hline
		\multicolumn{1}{|c|}{{\bf Parametr}} & 
		\multicolumn{1}{c|}{{\bf Wartość}} \\ 
		\hline \hline
		\multicolumn{1}{|l|}{WP} &
		\multicolumn{1}{l|}{R} \\ 
		\hline
	\end{tabularx}
\end{config}

\begin{table}[h]
	\caption{Wyniki heurystyki DSEA z operatorem haremowym \label{table:knapsack_results_dsea_harem}}
	\centering
	\begin{tabular}{|l|l|l|l|r@{$\pm$}l|}
		\hline
		\multicolumn{1}{|c|}{{\bf a}} & \multicolumn{1}{|c|}{{\bf b}} & \multicolumn{1}{|c|}{{\bf WA}} & \multicolumn{1}{c|}{{\bf WB}} & \multicolumn{2}{c|}{{\bf Ocena}} \\ \hline \hline
		\insertData{knapsack_d_top}
	\end{tabular}	
\end{table}

\subsubsection{Podsumowanie}

\todo{proza, co wyszło, a co nie}
\end{document}