\documentclass[./FM_mgr.tex]{subfiles}

\begin{document}
\chapter{Algorytmy ewolucyjne} \label{chapter:eaDesc}

Ten rozdział poświęcony jest omówieniu algorytmów ewolucyjnych.
W pierwszej jego części opisana jest ogólna idea tych heurystyk. 
Kolejna sekcja skupia się na formalnym opisie poszczególnych pojęć.
Po niej znajduje się część, w której znajdziemy szczegóły idei opisanej na początku, zapisane z użyciem pojęć przedstawionych w sekcji drugiej.
W następnym podrozdziale opisane zostały sposoby oceny i analizy działania algorytmów ewolucyjnych.
Całość zamyka przegląd prac naukowych poświęconych zagadnieniu płci w dziedzinie algorytmów ewolucyjnych.

\section{Idea algorytmów ewolucyjnych} \label{section:general_idea}

Podstawowym pojęciem używanym w opisywanej heurystyce jest \emph{osobnik}. 
Jest to abstrakcja pojedynczego rozwiązania, nawiązująca do pojedynczego żywego stworzenia charakteryzującego się pewnymi cechami, które wpływają na prawdopodobieństwo jego przeżycia i wydania potomstwa. 
Pojęciem symulującym wpływ cech na prawdopodobieństwo przeżycia jest \emph{funkcja oceny}. 
Dodatkowo wprowadzamy pojęcie \emph{populacji}, czyli zbioru osobników istniejących w danym momencie w procesie ewolucji.

Osobnik reprezentuje rozwiązanie problemu, a populacja - zbiór rozwiązań, czyli podzbiór przestrzeni rozwiązań.
Populację możemy też rozumieć jako zbiór próbek z pewnej podprzestrzeni zbioru rozwiązań.

Upraszczając nieco, przez pojęcie funkcji oceny rozumiemy pewne przyporządkowanie (w idealnej sytuacji - funkcję, jednak nie zawsze jest to możliwe) każdemu osobnikowi, czyli rozwiązaniu problemu, jakiejś wartości, na której możemy określić porządek.
Taka relacja porządku powinna określać czy dane rozwiązanie jest równie dobre lub lepsze niż inne, czy nie.

Osobnik najczęściej reprezentowany jest jako wektor binarny.
Taka postać sprawdza się w wielu zastosowaniach i jest prosty w realizacji programowej lub sprzętowej. 
Algorytmy ewolucyjne korzystające z takiej reprezentacji nazywane są algorytmami genetycznymi. 
Reprezentacja jest jednak uzależniona od rozwiązywanego problemu. 
Często wykorzystuje się bardziej skomplikowane formy, jak na przykład krotka wartości z pewnego zbioru (skończonego, lub nieskończonego, jak liczby), czy drzewo \cite{gen_prog} (np. dla problemów aproksymacji funkcji, przedstawionych jako zadanie minimalizacji błędu przybliżenia \cite{gen_prog_foo_approx}).

Warto dodać, że istnieją implementacje i narzędzia do obliczeń ewolucyjnych które rozróżniają pojęcie genotypu osobnika (czyli wewnętrznej reprezentacji rozwiązania, na której stosujemy operatory genetyczne takie jak krzyżowanie i mutacja) i jego fenotypu (czyli reprezentacji zewnętrznej, na podstawie której oceniamy osobnika), jednak jest to jedynie abstrakcja umożliwiająca czytelniejsze zapisanie realizacji odpowiednich pojęć w języku programowania. Niniejsza praca nie wprowadza takiego rozróżnienia, ponieważ nie zmienia ono w żadnym stopniu jakości działania heurystyki i nie jest związane z tematem badań.

Na rysunku \ref{figure:ea_general} zobrazowany został ogólny schemat działania algorytmów ewolucyjnych. Formalny opis używanych parametrów i operatorów znajduje się w kolejnej sekcji, wraz z definicją takich pojęć jak np. operator.

\begin{figure}
	
	\img{ea_general.png}
	\caption{Ogólny schemat działania algorytmów ewolucyjnych
		\label{figure:ea_general}
	Oznaczenia wykorzystane do rysowania schematów blokowych zostały wyjaśnione w tabeli \ref{table:flowchart_conventions}.}
\end{figure}
%
%\subfile{ea_general_tikz.tex}

Współczesna nauka dalej nie jest w stanie dać nam odpowiedzi na pytanie ,,skąd wzięło się życie'', a co za tym idzie, nie jesteśmy w stanie zasymulować momentu rozpoczęcia ewolucji.
Dlatego w heurystykach naśladujących ten proces działanie zaczyna się od wygenerowania losowej populacji. 
Symuluje to w uproszczony sposób analizę ewolucji od przypadkowego momentu w czasie istnienia gatunku.

Po utworzeniu zbioru początkowych osobników (czyli rozwiązań naszego problemu) rozpoczynamy proces przeszukiwania ich przestrzeni.
Dzieje się to iteracyjnie, co znaczy, że pewien zestaw kroków jest wielokrotnie powtarzany.
Jedno takie powtórzenie jest nazywane \emph{generacją} lub \emph{pokoleniem}, a składa się na nie kolejno krzyżowanie, mutacja i selekcja naturalna.
Moment w którym należy przerwać proces optymalizacji jest określany przez kryterium zatrzymania. 
Trudno tu znaleźć analogię do natury.
Kryterium stopu zapewnia, że heurystyka nie będzie działać wiecznie, tylko w skończonym czasie zwróci interesujące nas wyniki.

Na pojedyncze pokolenie składają się 3 główne kroki: krzyżowanie, mutacja i selekcja. 
Mają one symulować odpowiednie zjawiska występujące w przyrodzie - kolejno dziedziczenie, mutację i przetrwanie lepiej dopasowanych osobników.
Pierwsze dwa z nich kontrolują tempo przeszukiwania przestrzeni rozwiązań, a ostatnie steruje przeszukiwaniem tak, aby odnajdywać jak najlepsze rozwiązania. 

Krzyżowanie odbywa się przy użyciu dedykowanego operatora, z częstością określoną odpowiednim prawdopodobieństwem.
Polega ono na wybraniu losowych par osobników nazywanych rodzicami.
Stosuje się na nich operator krzyżowania, którego wyniki są dołączane do populacji.
Ma to symulować dziedziczenie cech rodziców przez ich potomków.
W algorytmach ewolucyjnych zadaniem krzyżowania jest eksploatacja przestrzeni rozwiązań, poprzez zbadanie punktów, które znajdują się pomiędzy znanymi już rozwiązaniami.
Dzięki temu zagęszcza się próbkowanie podprzestrzeni, co ma na celu znalezienie potencjalnych optimów, które mogły zostać przeoczone przez zbyt rzadkie próbkowanie.

Jedną z popularnych realizacji operatora krzyżowania dla osobnika reprezentowanego jako wektor jest wybranie losowego punktu przecięcia w wektorach opisujących rodziców i zwrócenie 2 osobników powstałych przez zamianę podwektorów od wylosowanej pozycji wzwyż. 
Na rysunku \ref{vector:example_crossover} zobrazowano działanie takiego podejścia. 
Wektory $(x_0, \ldots, x_9)$ i $(y_0, \ldots, y_9)$ to przykładowi rodzice. 
Grubą linią zaznaczono wylosowany punkt przecięcia, wyznaczający zamieniane części wektorów. 
Wektory $(x_0, \ldots, x_3, y_4, \ldots y_9)$ i $(y_0, \ldots, y_3, x_4, \ldots x_9)$ są wtedy traktowane jako potomstwo.

\begin{figure}[h]
	\caption{Działanie przykładowej realizacji operatora krzyżowania \label{vector:example_crossover}}
	\centering
	\begin{tikzpicture}[>=triangle 60]
	\tikzstyle{cellStyle}=[draw, rectangle, minimum width=1cm, minimum height=1cm, text centered, draw=black]
	\tikzstyle{first}=[cellStyle, fill=gray!30]
	\tikzstyle{second}=[cellStyle, fill=gray!70]
	
	%first input
	\foreach \x in {0,1,...,9}
	\node[first] at (\x, 0) {$x_\x$};
	
	%second input	
	\foreach \x in {0,1,...,9}
	\node[second] at (\x, -1.5) {$y_\x$};
	
	%swap arrow
	\draw[<->, >=latex', shorten >=2pt, shorten <=2pt, bend left=45, ultra thick] 
	(10, 0) to (10, -1.5); 
	
	%input cutpoint
	\draw[ultra thick] (3.5, 1) -- (3.5, -2.5);
	
	%operator arrow
	\draw [->, thick] (4.5, -2.5)  -- (4.5, -4.5); 
	\node[right] at (4.5, -3.5) {operator krzyżowania};
	
	%first output
	\foreach \x in {0,1,...,3}
	\node[first] at (\x, -5.5) {$x_\x$};
	\foreach \x in {4,5,...,9}
	\node[second] at (\x, -5.5) {$y_\x$};
	
	%second output
	\foreach \x in {0,1,...,3}
	\node[second] at (\x, -7) {$y_\x$};
	\foreach \x in {4,5,...,9}
	\node[first] at (\x, -7) {$x_\x$};
	
	%output cutpoint
	\draw[ultra thick] (3.5, -4.5) -- (3.5, -8);
	\end{tikzpicture} 
\end{figure}

Innym operatorem jest mutacja, czyli wprowadzenie losowych zmian w losowo wybranych osobnikach. 
Operacja ta jest wykonywana z pewnym prawdopodobieństwem określającym szansę osobnika na zmutowanie.
Zmodyfikowane osobniki w zależności od implementacji są dołączane do populacji, albo zastępują niezmodyfikowane rozwiązania.
W tej pracy wykorzystano pierwszą z tych opcji, aby nie doprowadzić do sytuacji, w której mutacja pogorszy znalezione wcześniej rozwiązanie.
Zadaniem tego kroku jest eksploracja przestrzeni rozwiązań, czyli zbadanie rozwiązań spoza dotychczas spróbkowanej podprzestrzeni.
W ten sposób heurystyka przeszukuje nowe obszary. 

Jedną z najczęstszych realizacji operatora mutacji dla osobnika reprezentowanego jako wektor bitów jest negacja losowych wartości, w wyniku czego otrzymujemy osobniki różniące się od argumentu operatora tylko kilkoma elementami. Na rysunku \ref{vector:example_mutation} zobrazowano działanie takiego podejścia. Wektor $(1, 0, 0, 1, 1, 1, 0, 0, 1, 0)$ to argument operatora. Kolorem zaznaczono losowe pozycje w wektorze, które zostaną zanegowane, w wyniku czego powstanie wektor wyjściowy $(1, \textbf{1}, 0, 1, \textbf{0}, 1, 0, 0, 1, \textbf{1})$.

\begin{figure}[h]
	\caption{Działanie przykładowej realizacji operatora mutacji \label{vector:example_mutation}}
	\centering
	\begin{tikzpicture}[>=triangle 60]
	\tikzstyle{cellStyle}=[draw, rectangle, minimum width=1cm, minimum height=1cm, text centered, draw=black]
	\tikzstyle{mutated}=[cellStyle, fill=gray!50]
	
	%original
	\node[cellStyle] at(0, 0) {$1$};
	\node[mutated] at(1, 0) {$0$};
	\node[cellStyle] at(2, 0) {$0$};
	\node[cellStyle] at(3, 0) {$1$};
	\node[mutated] at(4, 0) {$1$};
	\node[cellStyle] at(5, 0) {$1$};
	\node[cellStyle] at(6, 0) {$0$};
	\node[cellStyle] at(7, 0) {$0$};
	\node[cellStyle] at(8, 0) {$1$};
	\node[mutated] at(9, 0) {$0$};
	
	%operator arrow
	\draw [->, thick] (4.5, -1)  -- (4.5, -3); 
	\node[right] at (4.5, -2) {operator mutacji};
	
	%mutated
	\node[cellStyle] at(0, -4) {$1$};
	\node[mutated] at(1, -4) {$1$};
	\node[cellStyle] at(2, -4) {$0$};
	\node[cellStyle] at(3, -4) {$1$};
	\node[mutated] at(4, -4) {$0$};
	\node[cellStyle] at(5, -4) {$1$};
	\node[cellStyle] at(6, -4) {$0$};
	\node[cellStyle] at(7, -4) {$0$};
	\node[cellStyle] at(8, -4) {$1$};
	\node[mutated] at(9, -4) {$1$};				
	
	\end{tikzpicture} 
\end{figure}

Zmiany zachodzące w jednym osobniku nie powinny być zbyt drastyczne, ale też nie mogą być zbyt mało znaczące, ponieważ znacząco spowolniłoby to proces eksploracji.
Gdyby operator mutacji pokazany na rysunku \ref{vector:example_mutation} operował na wektorach o długości rzędu kilku tysięcy, to zamiana pojedynczego bitu byłaby zbyt mało znacząca, ponieważ rozwiązania przed i po jego zastosowaniu praktycznie by się od siebie nie różniły.
Jeśli jednak operator zmieniałby około połowy bitów, to nowe rozwiązanie byłoby zbyt dalekie od pierwotnego, przez co przeszukiwanie przestrzeni rozwiązań odbywałoby się kompletnie losowo.

W zależności od implementacji oba operatory opisane powyżej mogą być stosowane w różnej kolejności.
Ponadto, od realizacji zależy to, czy mutacja jest stosowana jedynie na populacji z poprzedniego pokolenia, czy również na potomstwie.
W tej pracy przyjmuje się, że krzyżowanie zostaje wykonane przed mutacją, a jego wyniki również mogą być przez nią modyfikowane.
Dzięki temu próbkowanie przestrzeni rozwiązań jest gęstsze, niż gdybyśmy postępowali w inny sposób.

Zgodnie z prawami rządzącymi naturą, większe prawdopodobieństwo przeżycia powinny mieć jednostki lepiej dopasowane do swojego środowiska, co w heurystyce przekłada się na lepszą (mniejszą, lub większą, w zależności od rozwiązywanego zadania) funkcję oceny.

Najprostszym operatorem selekcji jest tzw. operator elitarny, którego działanie polega na zwróceniu określonej liczby bezwzględnie najlepszych osobników z dotychczasowej populacji.
Mimo, że wydaje się to intuicyjne, to nie jest to rozwiązanie idealne.
Praktyka pokazuje, że rozwiązania gorsze często mają cechy, które w dalszych pokoleniach, po kolejnych modyfikacjach, dają rozwiązania lepsze od dotychczasowych.
Jest to powodem istnienia takich realizacji jak np. selekcja turniejowa, czy ruletkowa.

Proces selekcji służy do efektywnego przeszukiwania przestrzeni rozwiązań.
Zadanie optymalizacji ma na celu znalezienie jak najlepszego z nich.
Oznacza to, że jedno z rozwiązań, które zostanie zbadane w trakcie działania heurystyki musi być zapamiętane i zwrócone jako wynik. 
Naiwnym podejściem jest analiza populacji po ostatnim pokoleniu i zwrócenie najlepszego z osobników. 
Jako, że operator selekcji nie gwarantuje tego, że w zbiorze wynikowym znajdzie się najlepszy osobnik ze zbioru wejściowego, to istnieje szansa, że najlepsze rozwiązanie z ostatniej populacji nie jest najlepszym rozwiązaniem znalezionym w trakcie działania heurystyki. 
Zamiast tego w trakcie symulacji zapamiętujemy globalnie najlepsze rozwiązanie, na jakie trafiliśmy i to rozwiązanie traktujemy jako rozwiązanie problemu optymalizacji.
Zazwyczaj realizuje się to w ten sposób, że definiuje się zmienną przechowującą globalne optimum\footnote{
	W tym kontekście optima globalne i lokalne oznaczają najlepsze dotychczas znalezione rozwiązanie i najlepsze rozwiązanie w danej populacji.
}, która początkowa ma wartość pustą.
Po zakończeniu pierwszego pokolenia, ale przed zastosowaniem operatora selekcji, zapisuje się w niej najlepsze rozwiązanie w tym pokoleniu.
W tym samym momencie w dalszych pokoleniach porównuje się ocenę obecnego globalnego optimum i najlepszego osobnika z danego pokolenia (optimum lokalnego). 
Jeśli optimum globalne jest gorsze niż lokalne, to zostaje nim zastąpione.

\section{Formalny opis operatorów i parametrów}

Niniejszy podrozdział przedstawia formalny opis algorytmu ewolucyjnego.
Z założenia ma on być na tyle ogólny, aby możliwe było jego użycie niezależnie od implementacji.

W pierwszej kolejności przedstawiony zostanie zbiór zasad oznaczania pojęć i diagramów używany w tej pracy.

\subsection{Przyjęte zasady oznaczania} \label{subsection:conventions}

W tabeli \ref{table:signature_conventions} zostały pokazane konwencje dotyczące znaczenia czcionek używane w tej pracy.

\begin{table}[h]
	\caption{Przyjęte zasady oznaczania czcionek. W lewej kolumnie znajduje się przykładowy tekst zapisany odpowiednią czcionką, której znaczenie jest opisane w prawej kolumnie. \label{table:signature_conventions}}
	\centering
	\begin{tabular}{cl}
		$\param{operator}, \param{n}$ & operatory i parametry heurystyki \\
		$\variable{k}$ & parametry zależne od realizacji operatorów \\
		$T, p$ & zbiory i parametry pomocnicze \\
		$\important{S}$ & zbiory używane w definicjach \\
		$\R$ & relacje \\
		$\numberSet{N}$ & zbiory liczbowe
	\end{tabular}
\end{table}

W tabeli \ref{table:flowchart_conventions} pokazane i opisane zostały różne elementy schematów blokowych, używane w tej pracy.
%
%Na rysunku \ref{figure:flowchart_conventions} zobrazowane zostały różne elementy schematów blokowych używane w tej pracy, opisane w tabeli \ref{table:flowchart_conventions}.

%\begin{figure}
%	\caption{Konwencje dotyczące elementów schematów blokowych \label{figure:flowchart_conventions}}
%	\img{conventions.png}
%\end{figure}

\begin{table}
	\caption{Opis konwencji dotyczących elementów schematów blokowych \label{table:flowchart_conventions}}
	\begin{tabularx}{\linewidth}{|c|X|}
%	\begin{tabularx}{\linewidth}{|>{\centering\arraybackslash}p{\diagramPartColumnWidth}|X|}
		\hline 
		\inlineImg{parts/start.png} & Symbol rozpoczęcia heurystyki. \\
		\hline
		\inlineImg{parts/stop.png} & Symbol rozpoczęcia heurystyki. \\
		\hline
		\inlineImg{parts/cond.png} & Symbol oznaczający ewaluację warunku C i kontynuację procesu zgodnie z jego wynikiem. \\
		\hline
		\inlineImg{parts/statement.png} & Symbol oznaczający wykonanie kroku opisanego przez S. \\
		\hline
		\inlineImg{parts/changed.png} & Symbol tożsamy z powyższym, jednak stosowany w opisie proponowanego podejścia dla zaznaczenia różnic ze standardowym schematem (używane w rozdziale \ref{chapter:proposed}).\\
		\hline
		\inlineImg{parts/operator.png} & Kombinacja symboli oznaczająca, że podczas wykonywania kroku S używany jest operator O. \\
		\hline
		\inlineImg{parts/param.png} & Kombinacja symboli oznaczająca, że podczas wykonywania kroku S używany jest parametr P. \\
		\hline
		\inlineImg{parts/desc.png} & Symbol oznaczający, że dany schemat wyjaśnia szczegółowe działanie kroku D. \\
		\hline
		\inlineImg{parts/substep.png} & Symbol oznaczający rozpoczęcie lub zakończenie szczegółowo opisywanego kroku. \\
		\hline
		\inlineImg{parts/iterate.png} & Symbol oznaczający wykonanie kolejnego symbolu dla każdego z elementów opisanych przez E. \\
		\hline
	\end{tabularx}
\end{table}

Przyjęty w pracy sposób rysowania wykresów pokazano na rysunku \ref{plot:conventions}.

\begin{figure}
	\centering
	\graph{example_whiskers.tex}
	\caption{Konwencje dotyczące rysowania wykresów \label{plot:conventions}}
\end{figure}

W celu formalnego opisu operatorów i funkcji wykorzystuje się pojęcie klasy zbiorów o konkretnym rozmiarze. W niniejszej pracy przyjęto, że zapis $\withSize{A}{n}$ oznacza klasę zbiorów $n$-elementowych, w której każdy element należy do zbioru $A$. Zapis $A^n$, zgodnie ze standardową konwencją oznacza $n$krotny iloczyn kartezjański zbioru $A$, czyli klasę wektorów o długości $n$, których składowe pochodzą ze zbioru $A$. Oznaczenia $A^{n}$ i $\withSize{A}{n}$ mają różne znaczenie.

\begin{align}
	\withSize{A}{n} \equiv &\{ A': A' \subset A, |A'|=n  \} \\
	\withSize{A}{0} \equiv &\emptyset \\
	A^n \equiv &\underbrace{A \times A \times \ldots \times A}_{n} \\
	A^0 \equiv &\emptyset \\
	A \times B \equiv &\{ (a, b): a \in A, b \in B \}
\end{align}

$A$ i $B$ to oznaczenia na dowolne przykładowe zbiory. 
$A'$ to oznaczenie na podzbiór zbioru $A$.
$n$ to pewna liczba naturalna.

$\withSize{A}{n}$ oznacza zbiór wszystkich podzbiorów $A'$ zbioru $A$, których rozmiar wynosi $n$.
$A^n$ natomiast oznacza zbiór wszystkich krotek o długości $n$ których wszystkie elementy pochodzą ze zbioru $A$.

\subsection{Pojęcie operatora} \label{subsection:operator}

W dalszych rozdziałach będziemy używać pojęcia ,,operator'', które jest zbliżone do pojęcia funkcji. 
Różnica między tymi dwoma terminami jest taka, że funkcja musi zawsze zwrócić tę samą wartość dla tego samego argumentu, podczas gdy takie ograniczenie nie musi być spełnione dla operatora. Innymi słowy pojęcie operatora pokrywa się z pojęciem funkcji z imperatywnych języków programowania, ale nie jest tożsame z funkcją w rozumieniu matematycznym.

Operator możemy też rozumieć jako zmienną losową opisaną rozkładem parametryzowanym jego argumentami.

Warto pamiętać, że każda funkcja (w rozumieniu matematycznym) jest operatorem.

Do operatorów możemy stosować niektóre pojęcia używane w stosunku do funkcji. 
Przez \emph{dziedzinę} operatora rozumiemy przestrzeń dozwolonych argumentów operatora, a przez \emph{przeciwdziedzinę} - zbiór wartości, które może zwrócić.
\emph{Wynik} operatora (czyli wartość przez niego zwracaną), otrzymuje się przez jego \emph{zastosowanie} \emph{argumentach}.

Dodatkowo, operator możemy \emph{sparametryzować}, aby uzyskać inny operator. 
Przez \emph{parametry} rozumiemy argumenty, które w ramach danego ciągu obliczeń są stałe. 
Na potrzeby przykładu można zdefiniować pewien operator.
Nie będzie on wykorzystywany w dalszej części pracy i służy jedynie do zobrazowania pewnego założenia.
Może to być operator sąsiedztwa o określonym rozmiarze \emph{d} dla punktu \emph{(x, y)}, określony jako:

\begin{displaymath}
	sasiedztwo((x, y), d) \in [x-d, x+d] \times [y-d, y+d]
\end{displaymath}

Operator ten możemy sparametryzować rozmiarem sąsiedztwa, aby uzyskać operator sąsiedztwa o konkretnym rozmiarze, czyli:

\begin{displaymath}
	sasiedztwo_5 (P) = sasiedztwo (P, 5)
\end{displaymath}

\Defacto każdy operator opisywany w tej pracy może być parametryzowany. 
W kolejnych rozdziałach przyjęto konwencję, według której definiując nowy operator zapisywana jest jego minimalna dziedzina. 
Nie oznacza, że podczas realizacji nie może on być parametryzowany. 
Analogicznie stwierdzenie, że operator powinien przyjmować daną liczbę argumentów nie oznacza, że nie może on przyjmować ich więcej. 
Innymi słowy definicja postaci:

\begin{displaymath}
	operator: D \rightarrow C
\end{displaymath}

gdzie $D$ oznacza dziedzinę, a $C$ przeciwdziedzinę, jest równoważna z:

\begin{displaymath}
	operator: D \times P \rightarrow C
\end{displaymath}

gdzie $P$ oznacza przestrzeń parametrów, zależnych od realizacji operatora.

Ponadto, w algorytmach znajdujących się w tej pracy używany jest operator $random(X)$ opisany definicją \ref{signature:random}.
Zwraca on losowy element zbioru $X$ (z rozkładem równomiernym).

\begin{signature}
	\caption{Operator $random(X)$ \label{signature:random}}
	\begin{align}
		random: &\withSize{X}{n} \rightarrow X \\
		X' \gets &\left\{ x_0, x_1, \dots x_{n-1} \right\} \\
		\forall_{x_i, x_j \in X'} P(random(X') = x_i) = &P(random(X') = x_j)
	\end{align}
	$X$ to dowolny zbiór, a $n$ jego rozmiar.
	Zbiór $X'$ jest zbiorem pomocniczym, zawierającym dowolne elementy $x_i$.
	Za jego pomocą możemy zapisać z jakim rozkładem losowane są wyniki operatora.
	Funkcja $P()$ to funkcja prawdopodobieństwa.
\end{signature}

\subsection{Osobnik, populacja i ocena} \label{subsection:specimen_and_foo}

Podstawowym pojęciem w dziedzinie algorytmów ewolucyjnych jest osobnik, czyli rozwiązanie. 
Definiujemy go jako element przestrzeni rozwiązań, którą również musimy określić. 
Na te pojęcia nie ma nałożonych żadnych ograniczeń.

Istotnym pojęciem w algorytmach ewolucyjnych jest funkcja oceny. 
Wbrew nazwie nie musi być to funkcja w rozumieniu matematycznym, a operator\footnote{
	Stochastyczną funkcję oceny możemy zastosować na przykład w sytuacji, w której za pomocą algorytmu ewolucyjnego szukamy konfiguracji innej heurystyki. 
	Osobnikiem w takiej sytuacji może być zestaw parametrów konfigurowanej heurystyki, a oceną - średnia jakość działania dla kilku uruchomień.
}. 
Jego dziedziną powinna być przestrzeń możliwych osobników, a przeciwdziedziną dowolny zbiór, na którym możemy określić relację porządku. 
Relacja ta odpowiada relacji bycia lepiej przystosowanym do środowiska w rzeczywistym procesie ewolucji.


\begin{signature}
	\caption{Osobnik \label{signature:specimen}}
	\begin{align}
		osobnik \in &\important{S}
	\end{align}
	Zbiór $\important{S}$ to przestrzeń rozwiązań.
\end{signature}

\begin{signature}
	\caption{Funkcja oceny \label{signature:eval}}
	\begin{align}
		\param{funkcjaOceny}: &\important{S} \rightarrow \important{E} \\
		\exists \minorityEvalRel \subset &\important{E} \times \important{E} \\
		\minoritySpecimenRel \gets &\left\{ (x, y): (\param{funkcjaOceny}(x), \param{funkcjaOceny}(y)) \in \minorityEvalRel \right\}
	\end{align}
	Zbiór $\important{E}$ to zbiór możliwych ocen osobnika, 
	a $\minorityEvalRel$ to relacja porządku na nim określona. Ponadto określamy relację lepszego dopasowania $\minoritySpecimenRel$, porządkującą przestrzeń osobników według porządku określonego na ich ocenach. 
\end{signature}

\subsection{Operator mutacji}

Operator mutacji przyjmuje i zwraca pojedynczego osobnika.
Wynik działania operatora powinien być podobny, ale nie identyczny do argumentu.
Ma to na celu symulację mutacji występującej w naturze.

Formalniej, dziedziną i przeciwdziedziną operatora mutacji jest przestrzeń rozwiązań.

Z operatorem mutacji ściśle związany jest jeden z parametrów algorytmu ewolucyjnego - prawdopodobieństwo mutacji. 
Jest to wartość określająca prawdopodobieństwo zastosowania operatora mutacji do osobnika w danym pokoleniu. 
Jeśli wartość ta jest równa 0, to operator mutacji nie jest w ogóle używany, a jeśli jest równa 1, to operator zostanie zastosowany do każdego osobnika.

\begin{signature}
	\caption{Operator mutacji \label{signature:mutation}}
	\begin{align}
		\param{opMutacji}: & \important{S} \rightarrow \important{S} \\
		\param{prawdMutacji} \in & \range{0}{1}
	\end{align}
	$\param{opMutacji}$ to operator mutacji, a $\param{prawdMutacji}$ to prawdopodobieństwo jego zastosowania.
\end{signature}	

\subsection{Operator krzyżowania} \label{subsection:crossover}

Dziedziną tego operatora jest klasa $\inputVar{c}$-elementowych podzbiorów przestrzeni rozwiązań, a przeciwdziedziną - klasa $\outputVar{c}$-elementowych podzbiorów tej przestrzeni. 
Oznacza to, że operator ten przyjmuje $\inputVar{c}$ osobników (nazywanych \emph{rodzicami}) jako argument, a zwraca $\outputVar{c}$ osobników (nazywanych \emph{potomstwem}). 
Zazwyczaj przyjmuje się $ \inputVar{c} = 2 $ i $\outputVar{c} \in \left\{1, 2 \right\}$. 
Zwracane osobniki powinny być podobne (w takim sensie, że mają podobną reprezentację) do osobników wejściowych (argumentów operatora). 

Z operatorem krzyżowania ściśle związany jest jeden z parametrów algorytmu ewolucyjnego - prawdopodobieństwo krzyżowania. Jest to wartość określająca prawdopodobieństwo zastosowania operatora krzyżowania do osobnika w danym pokoleniu. Jeśli wartość ta jest równa 0, to operator krzyżowania nie jest w ogóle używany, a jeśli jest równa 1, to operator zostanie zastosowany do każdego osobnika.

\begin{signature}
	\caption{Operator krzyżowania \label{signature:crossover}}
	\begin{align}
		\param{opKrzyzowania}: & \withSize{\important{S}}{\inputVar{c}} \rightarrow \withSize{\important{S}}{\outputVar{c}} \\
		\param{prawdKrzyzowania} \in & \range{0}{1}
	\end{align}
	$\inputVar{c}$ to liczba rodziców, a $\outputVar{c}$ to liczba potomków.
\end{signature}	

\subsection{Operator selekcji} \label{subsection:natSel}

Zadaniem operatora selekcji jest symulacja zjawiska przeżycia silniejszych (czyli lepiej dopasowanych) osobników. Jest on stosowany pod koniec każdego pokolenia w celu usunięcia z niej osobników gorzej dopasowanych, poprzez wykorzystanie populacji wyjściowej jako używanej w kolejnym pokoleniu. Zmniejsza on różnorodność genetyczną w obrębie populacji, co w ogólności jest negatywnym efektem, ponieważ zahamowuje  eksplorację przestrzeni rozwiązań. Operator mutacji powinien eksplorować ją na tyle efektywnie, żeby operator selekcji odrzucał mało obiecujące kierunki eksploracji zamiast całkiem zatrzymywać jej proces.

Dziedziną tego operatora jest klasa $p$-elementowych podzbiorów przestrzeni rozwiązań, co oznacza, że przyjmuje on populację $p$ osobników. 
Wartość $p$ to liczba osobników w populacji pod koniec danej generacji, więc można ją przybliżać za pomocą równań \ref{line:p_start}-\ref{line:p_stop} definicji \ref{signature:selection}, ponieważ w populacji znajdą się osobniki z populacji początkowej tego pokolenia oraz wyniki operatorów mutacji i krzyżowania. 

Przeciwdziedziną operatora selekcji jest klasa podzbiorów zbioru wejściowego, o rozmiarze $\param{rozmiarPopulacji}$.
Wartość $\param{rozmiarPopulacji}$ to parametr całej heurystyki, określająca jak liczna powinna być populacja na początku każdego pokolenia. 
Im większa jest jego wartość, tym lepsze przeszukiwanie przestrzeni rozwiązań, ale też tym dłużej trwają obliczenia (ponieważ trzeba ewaluować więcej osobników i do większej liczby z nich zastosować operatory genetyczne). 

Dodatkowo na operator nakłada się ograniczenie opisane wzorami \ref{line:constraint_sel_start}-\ref{line:constraint_sel_stop} definicji \ref{signature:selection}. 
Mówią one o tym, że prawdopodobieństwo tego, że osobnik z populacji wejściowej znajdzie się w populacji wyjściowej powinno być tym większe im lepiej dopasowany jest osobnik. 
Ma to symulować zasadę przetrwania osobników najlepiej przystosowanych do środowiska.

\begin{signature}
	\caption{Operator selekcji \label{signature:selection}}
	\begin{align}
		\param{opSelekcji}: &\withSize{\important{S}}{p} \rightarrow \withSize{\important{S}}{\param{rozmiarPopulacji}} \\
		\param{rozmiarPopulacji} \in &\numberSet{N}_{+} \\
		\label{line:p_start} 
		p \approx &(1 + \param{prawdMutacji} \\
		&+ \param{prawdKrzyzowania} \times \outputVar{c}) \\
		\label{line:p_stop} 
		&\times \param{rozmiarPopulacji} \\
		\label{line:constraint_sel_start}          
		S' \gets &\left\{ t_0, t_1, \ldots t_{\param{rozmiarPopulacji}-1} \right\} \\
%		s \in \param{opSelekcji}(T) &\Rightarrow s \in S'\\
		\param{opSelekcji}(S') \subset &S'\\
		P(i) \gets &P(s_i \in \param{opSelekcji}(S')) \\
		\label{line:constraint_sel_stop} 
		P(i) < P(j) &\Leftrightarrow (s_i, s_j) \in \minoritySpecimenRel
	\end{align}
	$\param{opSelekcji}$ to operator selekcji, a $\param{rozmiarPopulacji}$ to rozmiar populacji na początku każdego pokolenia.
	Wejściem operatora jest populacja na końcu każdego pokolenia, a wyjściem - populacja używana w kolejnym pokoleniu. \\
	$p$ to przybliżenie rozmiaru zbioru wejściowego za pomocą prawdopodobieństw zastosowania operatorów oraz rozmiaru populacji.\\
	$S'$ to pomocniczy zbiór przedstawiający przykładową populację na końcu pokolenia, zawierający osobniki $s_i$. \\
	$P()$ to funkcja prawdopodobieństwa. 
	W szczególności $P(i)$ oznacza prawdopodobieństwo tego, że osobnik $s_i$ znajdzie się w zbiorze będącym wynikiem zastosowania opisywanego operatora do populacji $S'$.\\
	$\minoritySpecimenRel$ to relacja porządku określona na osobnikach na podstawie ich ocen. Zapis $(s_i, s_j) \in \minoritySpecimenRel$ oznacza, że osobnik $s_i$ jest gorzej dopasowany niż $s_j$.
	
\end{signature}

\subsection{Warunek stopu} \label{subsection:stop}

Warunek stopu, nazywany też warunkiem zatrzymania lub przerwania to pojęcie określające warunek zakończenia heurystyki. Jest to odpowiedź na pytanie ,,kiedy należy przestać przetwarzanie kolejnych pokoleń?''. Często stosuje się takie kryteria jak przekroczenie jakiejś arbitralnej liczby pokoleń lub zaobserwowanie tzw. stagnacji. Zjawisko takie polega na znacznym zmniejszeniu różnorodności ocen w populacji, co oznacza, że znaleźliśmy optimum lokalne. Jeśli wariancja ocen przez kilka pokoleń się nie zmienia (lub zmienia, ale nieznacząco), to można uznać, że operatory genetyczne nie pozwolą na wyjście z tego optimum i przerwać działanie algorytmu ewolucyjnego. 
Warunek zatrzymania możemy traktować jako dodatkowy operator, jednak trudno jest określić jego jednoznaczną definicję, ze względu na dowolność realizacji. W definicji \ref{signature:stop} przedstawiono ogólny zapis operatora (z dowolną dziedziną, linia \ref{line:stop_general}) i jego przykładowej realizacji (opartej o stałą liczbę pokoleń, linie \ref{line:stop_maxGen_begin}-\ref{line:stop_maxGen_end}).

\begin{signature}
	\caption{Warunek zatrzymania i jego przykładowa realizacja \label{signature:stop}} 
	\begin{align}
		\label{line:stop_general}
		\param{warunekStopu}: &\ldots \rightarrow \left\{ 0, 1 \right\}
	\end{align}
	\begin{align}
		\label{line:stop_maxGen_begin}
		\param{warunekStopu}: &\numberSet{N} \rightarrow \left\{ 0, 1 \right\} \\
		\label{line:stop_maxGen_end}
		\param{warunekStopu}(g) = &1 \Leftrightarrow g \leq \overline{\variable{g}}
	\end{align}
	Zwrócenie wartości $1$ oznacza, że należy zakończyć działanie heurystyki, a $0$ - sytuację odwrotną. \\ 
	Wartość $g$ oznacza numer obecnego pokolenia (zaczynając od 1), a $\overline{\variable{g}}$ - maksymalną dopuszczalną liczbę pokoleń w ramach jednego przebiegu algorytmu ewolucyjnego.
\end{signature}	

\section{Szczegóły działania algorytmu ewolucyjnego} \label{section:ea_details}

Ogólny schemat działania algorytmu ewolucyjnego został przedstawiony w sekcji \ref{section:general_idea}. 
W następnych akapitach szczegółowo omówiony zostanie diagram tam zawarty.

Cały proces rozpoczyna się od inicjalizacji populacji za pomocą losowych osobników. Następnie, póki warunek zatrzymania nie jest spełniony, powtarzane są kroki wykonywane w ramach jednego pokolenia. Są to kolejno: krzyżowanie, mutacja i selekcja naturalna.

\begin{algorithm}[h]
	\caption{Szczegółowy schemat działania kroku "Wykonaj krzyżowanie"	\label{algorithm:crossOver_std}}
	\begin{algorithmic}[1]
		\Context{
			$populacja$ 
			\Comment{Obecna populacja, tj. zbiór osobników}
		}
		\Params{
			$\param{prawdKrzyzowania}$
			\Comment{\raggedleft Prawdopodobieństwo krzyżowania}\\ 
			$\param{opKrzyzowania}$  
			\Comment{Operator krzyżowania}
		}
		\Start
		\Procedure{wykonajKrzyzowanie}{}
		\Var $noweOsobniki \gets \emptyset$
		\label{line:cross_init} 
		\ForAll {$osobnik \in populacja$}
		\label{line:cross_each} 
		\If {$random([0,1]) \leq \nicefrac{\param{prawdKrzyzowania}}{2}$}
		\label{line:cross_check} 
		\Var $partner \gets random(populacja \setminus \left\{ osobnik \right\})$ 
		\label{line:cross_choose}
		\Var $potomek \gets \param{opKrzyzowania}(\{ osobnik, partner \})$
		\label{line:cross_apply} 
		\State $noweOsobniki \gets noweOsobniki \cup \left\{ potomek \right\}$
		\label{line:cross_add}
		\EndIf 
		\EndFor
		\State $populacja \gets populacja \cup noweOsobniki$ 
		\label{line:cross_merge}
		\EndProcedure
	\end{algorithmic}
\end{algorithm}

Algorytm \ref{algorithm:crossOver_std} opisuje szczegóły kroku "Wykonaj krzyżowanie". Krok ten rozpoczyna się od inicjalizacji zmiennej $noweOsobniki$ (w linii \ref{line:cross_init}).
Jej wartość to pusty zbiór $\emptyset$.
Po wykonaniu większości kroków, w linii \ref{line:cross_merge}. wartość tej zmiennej jest dołączona do ogólnej populacji. 
Dla każdego osobnika w populacji (linia \ref{line:cross_each}) sprawdzane jest, czy należy wykonać krzyżowanie z jego udziałem (linia \ref{line:cross_check}). 
Jeżeli tak, to losowo dobierany jest jego partner (linia \ref{line:cross_choose}), a następnie na tak określonej parze osobników stosowany jest operator krzyżowania (linia \ref{line:cross_apply}), którego wynik jest dołączany do zbioru $noweOsobniki$ (linia \ref{line:cross_add}). Schemat ten przewiduje, że operator krzyżowania przyjmuje 2 osobniki jako argument (tzn. $\inputVar{c} = 2$, patrz: rozdział \ref{subsection:crossover}), przez co wartość, z którą porównujemy losowo wybraną liczbę z przedziału $\range{0}{1}$ to $\nicefrac{\param{prawdKrzyzowania}}{2}$. W ogólności, dla dowolnego $\inputVar{c}$ wartość ta wynosi $\nicefrac{\param{prawdKrzyzowania}}{\inputVar{c}}$, a zamiast jednego partnera należy wybrać ich $\inputVar{c}-1$.

Nowe osobniki nie są dołączane od razu do populacji, tylko zbierane do zbioru $noweOsobniki$.
Ma to na celu uniknięcie sytuacji, w których potomek wydany w danym pokoleniu zostałby w tym samym pokoleniu rodzicem.

\begin{algorithm}[h]
	\caption{Szczegółowy schemat działania kroku "Wykonaj mutację"	\label{algorithm:mutation_std}}
	\begin{algorithmic}[1]
		\Context{
			$populacja$ 
			\Comment{Obecna populacja, tj. zbiór osobników}
		}
		\Params{
			$\param{prawdMutacji}$
			\Comment{\raggedleft Prawdopodobieństwo mutacji}\\ 
			$\param{opMutacji}$  
			\Comment{Operator mutacji}
		}
		\Start
		\Procedure{wykonajMutacje}{}
		\Var $noweOsobniki \gets \emptyset$
		\label{line:mut_init}
		\ForAll {$osobnik \in populacja$}
		\label{line:mut_each}
		\If {$random([0,1]) \leq \param{prawdMutacji}$}
		\label{line:mut_check}
		\Var $mutant \gets \param{opMutacji}(osobnik)$
		\label{line:mut_apply}
		\State $noweOsobniki \gets noweOsobniki \cup \left\{ mutant \right\}$
		\label{line:mut_add}
		\EndIf 
		\EndFor
		\State $populacja \gets populacja \cup noweOsobniki$
		\label{line:mut_merge}
		\EndProcedure
	\end{algorithmic}
\end{algorithm}

Algorytm \ref{algorithm:mutation_std} opisuje szczegóły kroku "Wykonaj mutację". Krok rozpoczyna się od inicjalizacji zmiennej $noweOsobniki$ na pusty zbiór (linia \ref{line:mut_init}), który (w linii \ref{line:mut_merge}) zostanie dołączony do ogólnej populacji. Dla każdego osobnika w populacji (linia \ref{line:mut_each}) sprawdzane jest, czy należy wykonać na nim mutację (linia \ref{line:mut_check}), a jeśli tak, to stosowany jest do niego operator mutacji (linia \ref{line:mut_apply}), którego wynik zostaje dołączony do zbioru $noweOsobniki$ (linia \ref{line:mut_add}).

\begin{algorithm}[h]
	\caption{Szczegółowy schemat działania kroku "Dokonaj selekcji naturalnej"	\label{algorithm:natSel_std}}
	\begin{algorithmic}[1]
		\Context{
			$populacja$ 
			\Comment{Obecna populacja, tj. zbiór osobników}
		}
		\Params{
			$\param{rozmiarPopulacji}$
			\Comment{Rozmiar populacji}\\ 
			$\param{opSelekcji}$  
			\Comment{Operator selekcji}
		}
		\Start
		\Procedure{dokonajSelekcjiNaturalnej}{}
		\State $populacja \gets \param{opSelekcji}_{\param{rozmiarPopulacji}}(populacja)$
		\label{line:sel_apply}
		\EndProcedure
	\end{algorithmic}
\end{algorithm}

Algorytm \ref{algorithm:natSel_std} opisuje szczegóły kroku "Dokonaj selekcji naturalnej", który sprowadza się do zastąpienia dotychczasowej populacji wynikiem zastosowania operatora selekcji na niej (linia \ref{line:sel_apply}).

\section{Analiza jakości działania}

Algorytmy ewolucyjne to rodzina losowych, iteracyjnych heurystyk populacyjnych. Określenie ,,losowych'' oznacza, że każde uruchomienie procesu optymalizacji może zwrócić inny wynik. Co za tym idzie, ocena działania heurystyki dla danego zestawu parametrów jest trudna i wymaga wielokrotnego zebrania wyników. W kolejnej części opisane zostaną sposoby analizy jakości działania pojedynczego przebiegu heurystyki, jak i wielu przebiegów o tych samych parametrach.

\subsection{Analiza jednego przebiegu heurystyki}

Jak wspomniano, algorytmy ewolucyjne to iteracyjne heurystyki populacyjne. Oznacza to, że w ramach jednego procesu optymalizacji wielokrotnie powtarzamy pewien krok (stąd określenie ,,iteracyjne''), w którym badamy wiele rozwiązań (stąd określenie ,,populacyjne'').

Aby ocenić pojedynczy przebieg, możemy analizować pewne statystyki oceny w skali populacji na przestrzeni wielu pokoleń.

Podstawowe statystyki używane w tym celu, to m.in.:
\begin{itemize}
	\item najlepsza i najgorsza ocena osobnika z populacji,
	\item średnia i odchylenie standardowe (lub wariancja \footnote{Często wariancji używa się do automatyzacji oceny, ponieważ obliczenie jej nie wymaga czasochłonnego pierwiastkowania. Znając wariancję w dowolnym momencie możemy obliczyć odchylenie standardowe, np. w celu prezentacji oceny.  \label{footer:variance}}) oceny osobników w populacji,
	\item mediana i kwantyle oceny osobników w populacji.
\end{itemize}

Wartości te można wygodnie zobrazować na wykresie pudełkowym, na osi odciętych umieszczając numery pokoleń, a na osi rzędnych - wartości odpowiednich statystyk. Takie zobrazowanie przebiegu heurystyki pozwala na wyciągnięcie wniosków i stosowne dopasowanie parametrów heurystyki (np. jeśli w problemie minimalizacji wszystkie wartości stopniowo maleją, to być może warto zwiększyć rozmiar populacji lub zmienić kryterium zatrzymania, aby cały proces trwał dłużej, ponieważ taka obserwacja wskazuje na efektywne działanie heurystyki).

W tej pracy do analizy wykorzystano tylko 4 z wyżej wymienionych statystyk: średnią, wariancję\footnote{Zdecydowano się na wykorzystanie wariancji zamiast odchylenia standardowego ponieważ wartości odchylenia okazały się być zbyt małe, aby były dostrzegalne na wykresach.}, minimum i maksimum. Wydaje się, że są one wystarczające do analizy porównawczej między uruchomieniami. 

%\clearpage

\begin{figure}[H]
	\caption{Przykładowy \emph{wykres przebiegu} pokazujący zależność między  najlepszą, najgorszą i średnią oceną oraz jej wariancją od liczby pokoleń które minęły. \label{plot:random_example}}
	\centering
	\graph{random_example.tex}
\end{figure}

Na rysunku \ref{plot:random_example} przedstawiony jest przykładowy wykres statystyk populacji na przestrzeni wielu pokoleń. Wykresy takie, nazywane w skrócie \emph{wykresami przebiegów}, będą wykorzystywane w tej pracy do analizy pojedynczych uruchomień heurystyki. 

\begin{figure}[H]
	\caption{Wykres przebiegu, w którym globalne optimum nie znajduje się w ostatniej populacji \label{plot:mid_opt}}
	\centering
	\graph{mid_optimum.tex}
\end{figure}

Na rysunku \ref{plot:mid_opt} przedstawiony jest wykres przebiegu na którym możemy zaobserwować sytuację opisaną na końcu rozdziału \ref{section:general_idea}, tzn. taką, w której optimum globalne jest znalezione w innym pokoleniu niż ostatnie.

\begin{figure}[H]
	\caption{Wykres przebiegu w którym obserwujemy stagnację \label{plot:stagnation}}
	\centering
	\graph{stagnation.tex}
\end{figure}

Na rysunku \ref{plot:stagnation} przedstawiono sytuację, w której obserwujemy tzw. stagnację (wspominaną już w rozdziale \ref{subsection:stop}). Możemy zaobserwować, że od pewnego momentu (około sześćdziesiątej generacji) kolejne pokolenia nie przynoszą znaczącej zmiany wyniku, co mogłoby być powodem do przerwania działania heurystyki.

\subsection{Analiza wielu przebiegów heurystyki}

Jak zostało wspomniane na początku tej sekcji, algorytmy ewolucyjne to heurystyki losowe, przez co za każdym uruchomieniem zwracają różne wartości. Aby ocenić wyniki procesu optymalizacji dla różnych zestawów parametrów należy kilkukrotnie powtórzyć proces i porównywać statystyki wyników. Jeśli jesteśmy w trakcie dostrajania heurystyki (czyli dobierania najlepszych parametrów), to taką statystyką może być najlepszy wynik z kilku powtórzeń, jednak jeśli chcemy przeprowadzić miarodajne badania, to najprostszym podejściem dającym wgląd w jakość działania jest obliczenie średniej i wariancji (lub odchylenia standardowego, patrz: przypis \ref{footer:variance}) wyników wielu przebiegów dla różnych konfiguracji i porównanie ich testem statystycznym, takim jak np. test t-studenta.
\end{document}