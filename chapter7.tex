\documentclass[./FM_mgr.tex]{subfiles}

\begin{document}

Proponowane operatory: haremowy oraz uogólniający heurystyki GGA i SexualGA osiągnęły lepsze wyniki niż uogólniane metody.
W jednym z przypadków nie udało się jednak osiągnąć lepszych wyników niż klasyczny algorytm ewolucyjny.
Ponadto, operator haremowy działał równie dobrze lub gorzej od uogólnionego.

W niniejszej pracy nie zbadano wpływu operatora wyboru partnerów na wyniki DSEA z operatorem haremowym.
Być może użycie operatora innego niż losowy spowodowałoby poprawę wyników.
Możliwe jest też, że określając liczbę osobników alfa jako procent rozmiaru populacji można osiągnąć lepsze wyniki dla wyższych wartości.
Warto również zbadać znacznie wyższe wartości współczynnika beta.
Nie jest wykluczone, że jeśli osobników alfa będzie znacznie mniej, to role się zamienią i to beta będą przyczyniać się do jakości wyników.

Operatory selekcji płciowej przedstawione w tej pracy nie wyczerpują wszystkich podejść występujących naturze.
Potencjalnie dobrym pomysłem mogłoby być dobieranie się osobników w pary tak, aby rodzice byli jak najbardziej odmienni genetycznie od siebie.
Wymagałoby to określenia pewnej miary odległości między rozwiązaniami, wykorzystywanej podczas wyboru.

Innym pomysłem jest rozszerzenie operatora haremowego o przypisanie roli osobnika alfa aż do wirtualnej ,,śmierci'' tego osobnika.
Do reprezentacji osobnika należałoby dodać bit oznaczający fakt bycia osobnikiem alfa.
Bit ten nie byłby dziedziczony i dopóki nie zostałby ręcznie zapalony miałby wartość 0 (fałsz).
Na etapie selekcji płciowej populacja byłaby przeszukiwana pod kątem osobników z zapalonym bitem alfa.
Jeżeli byłoby ich za mało, za pomocą odpowiedniego operatora wyboru losowano by odpowiednią ich liczbę i oznaczano jako osobniki alfa.
Osobniki beta i partnerzy byliby wybierani niezależnie w każdym pokoleniu.
Zapalone bity znikałyby stopniowo z populacji z powodu zastosowania selekcji naturalnej.

Istnieje wiele operatorów wyboru, które nie zostały zbadane.
Przykładem może być prosty operator elitystyczny lub operator turniejowy o większym rozmiarze turnieju.
W inny sposób można również ważyć prawdopodobieństwa wybrania każdego z osobników w przypadku operatora ruletkowego, np. przez kwadrat rangi, lub ocenę rzutowaną na wybrany dodatni przedział.

Ponadto, w pracy zbadano wpływ różnych kombinacji operatorów selekcji na działanie heurystyk korzystających z tych samych wartości pozostałych parametrów takich jak np. prawdopodobieństwa zastosowania operatorów genetycznych.
Być może dostrojenie każdej heurystyki z osobna poprawiłoby wyniki.

Zaproponowane operatory dały dość dobre rezultaty, w niektórych przypadkach nawet lepsze od podstawowej wersji algorytmu ewolucyjnego.
Istnieje możliwość, że wykorzystanie opisanych rozwiązań do problemów optymalizacji wielokryterialnej również przyniosłoby pozytywne skutki.

Warto pamiętać o tym, że algorytmy ewolucyjne to tylko jedna z wielu technik obliczeń ewolucyjnych.
Pomysł operatora selekcji płciowej można zastosować w dowolnej z tego typu technik, w której wykorzystywane jest krzyżowanie.
Ciekawym pomysłem byłoby zbadanie warunków środowiskowych, które zasymulowane w ramach sztucznego życia podtrzymałyby różne strategie selekcji płciowej.


\end{document}