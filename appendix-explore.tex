\documentclass[./FM_mgr.tex]{subfiles}

\begin{document}
\section{Procedura eksploracji} \label{appendix:explore}

\todo{poprawić}
W ramach pierwszej części badań wykorzystano procedurę eksperymentalną, nazywaną procedurą eksploracji (lub krócej, eksploracją).

Jest ona wykorzystywana w sytuacji, w której pojawia się potrzeba przeszukania dużej przestrzeni parametrów.
Jest to często niemożliwe, ponieważ pojedyncza ewaluacja zestawu parametrów trwa relatywnie długo, a ilość kombinacji wartości parametrów jest zbyt duża.

Aby obejść ten problem, w tej pracy zdecydowano się na iteracyjne podejście przeszukujące przestrzeń konfiguracji wzdłuż wymiarów.
Zostało ono wyjaśnione za pomocą algorytmu \ref{algorithm:explore}.

\clearpage

\begin{algorithm}[H]
	\caption{Procedura eksploracji \label{algorithm:explore}}
	\begin{algorithmic}[1]
		\Params{
			$X_1$, $X_2$, $\ldots$, $X_{p}$ \Comment{\raggedleft Zestaw $p$ uporządkowanych zbiorów wartości parametrów $X_i = [ x_{i, 1}, x_{i, 2}, \ldots, x_{i, |X_i|} ]$, gdzie $x_{i, j}$ to $j$ta wartość $i$tego parametru} \\
			$R$ \Comment{Ilość nawrotów procedury} \\
			$I$ \Comment{Ilość powtórzeń eksperymentu dla każdej konfiguracji} \\
			$statystyka$ \Comment{Operator, przyjmujący zbiór wyników, a zwracający jakąś ich statystykę, np. średnią, medianę wartość minimalną lub maksymalną}
		}
		\Input{
			$cialo$ \Comment{\raggedleft Ciało eksperymentu, tj. operator wykonania pojedynczego eksperymentu, przyjmujący zestaw parametrów $(P_1, P_2, \ldots)$ ($P_i \in X_i$), a zwracający jakąś miarę jakości działania; dla uproszczenia w algorytmie przyjęto, że zwracaną miarę należy zminimalizować.
			}
		}
		\Output{
			$konfiguracja$ \Comment{\raggedleft Najlepsza konfiguracja znaleziona w procesie eksploracji, tj. $p$ wartości $x_{i, j}$ o różnych indeksach $i$}
		}
		\Start
		\Operator{eksploruj}{$cialo$}
		\Var $konfiguracja = [x_{1, 1}, x_{2, 1}, \ldots, x_{p, 1}]$
		\For{$nawrot \gets 1$ \To $R$}
		\For{$parametr \gets 1$ \To $p$}
		\Var $statystyki = [NULL, NULL, \ldots]$ \Comment{Tablica o rozmiarze $|X_{parametr}|$}
		\For{$ i \gets 1$ \To $|X_{parametr}|$}
		\State $konfiguracja[parametr] \gets x_{parametr, i}$ 
		\Var $wyniki \gets \emptyset$
		\For{$powtorzenie \gets 1$ \To $I$}
		\State $wyniki \gets wyniki \cup \{ cialo(konfiguracja) \}$
		\EndFor
		\State $statystyki[i] \gets statystyka(wyniki)$
		\EndFor
		\Var $najlepszaWartosc \gets \argmin_{i} statystyki[i]$
		\State $konfiguracja[parametr] \gets x_{parametr, najlepszaWartosc}$
		\EndFor
		\EndFor
		\State \Return $konfiguracja$
		\EndOperator
	\end{algorithmic}
	Zapis $a[b]$ oznacza element tablicy $a$ o indeksie $b$
\end{algorithm}


Procedura eksploracji jest zapisana jako operator, ponieważ zwraca pewną wartość (więc nie jest procedurą), która za każdym razem może być inna (więc nie jest funkcją).

Jej działanie zaczyna się od określenia początkowej konfiguracji, czyli zestawu parametrów. 
W tym celu wykorzystywane są pierwsze elementy każdego z zestawów wartości.
Wybór konkretnie tych wartości jest zupełnie arbitralny.

Następnie wykonywane jest $R$ iteracji nazywanych \emph{nawrotami}.
Ma to na celu upewnienie się, że dla parametrów wybranych w dalszej części każdego nawrotu wartości wybrane wcześniej faktycznie są jak najlepsze.

W każdym nawrocie badane są wszystkie zestawy wartości parametrów, w kolejności określonej przez parametry procedury.
Dla każdego z nich sprawdzane są wszystkie elementy.
Dzieje się to przez zmianę odpowiedniej wartości w konfiguracji na element badany w danym momencie.

Po takim podstawieniu ciało eksperymentu wykonywane jest $I$ razy, a wyniki tych wykonań zbierane są do zbioru $wyniki$.
Po odpowiedniej liczbie powtórzeń obliczana jest statystyka tego zbioru.
W tym celu wykorzystywany jest parametr procedury $statystyka$.
Jej wartość jest zapisywana tak, żeby można było określić dla której wartości parametru została ona określona.

Kiedy wszystkie wartości parametru zostaną zbadane, wybiera się najlepszą z nich, oceniając je po wartości statystyk.
Wartość ta zostaje zapisana w konfiguracji używanej do badania kolejnego z parametrów.

\todo{referencje do linii}
\end{document}