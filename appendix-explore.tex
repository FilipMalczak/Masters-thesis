\documentclass[./FM_mgr.tex]{subfiles}

\begin{document}
\section{Procedura eksploracji} \label{appendix:explore}

W pierwszej części przeprowadzonych eksperymentów badano wrażliwość algorytmu ewolucyjnego na wartości parametrów nie związanych z tematem niniejszej pracy.
Wynikiem tych badań były domyślne zestawy parametrów dla obu problemów, używane w kolejnych etapach.

Sprawdzenie wszystkich kombinacji wybranych wartości byłoby zbyt czasochłonne.
Ich łączna liczba wynosiła 972, co przy 5 powtórzeniach dla każdego zestawu i dwóch badanych problemach dawało około 10 tysięcy powtórzeń które należy wykonać.
Ponadto, po wybraniu którejś z kombinacji jako najlepszej, badano jej lokalne sąsiedztwo (modyfikując wartości liczbowe o małe kroki, tj. $\pm0,02$ i $\pm0,05$ dla wartości z przedziału $\range{0}{1}$ i $\pm5$ oraz $\pm10$ dla wartości całkowitych).
Mimo, że w tej części nie zmieniano operatora selekcji naturalnej, to liczba kombinacji wynosiła około 320, co dla 3 powtórzeń i 2 problemów dawało prawie 2 tysiące dodatkowych uruchomień heurystyki.

Aby przyspieszyć ten etap eksperymentów określono iteracyjną procedurę przeszukiwania przestrzeni parametrów, nazywaną procedurą eksploracji (lub krócej, eksploracją).
Polega ona na zmienianiu tylko jednego parametru na raz, co zostało wyjaśnione za pomocą algorytmu \ref{algorithm:explore}.

%\clearpage

\begin{algorithm}[h]
	\caption{Procedura eksploracji \label{algorithm:explore}}
	\begin{algorithmic}[1]
		\Params{
			$X_1$, $X_2$, $\ldots$, $X_{p}$ \Comment{\raggedleft Zestaw $p$ uporządkowanych zbiorów wartości parametrów $X_i = [ x_{i, 1}, x_{i, 2}, \ldots, x_{i, |X_i|} ]$, gdzie $x_{i, j}$ to $j$ta wartość $i$tego parametru} \\
			$R$ \Comment{Ilość nawrotów procedury} \\
			$I$ \Comment{Ilość powtórzeń eksperymentu dla każdej konfiguracji} \\
			$statystyka$ \Comment{Operator, przyjmujący zbiór wyników, a zwracający jakąś ich statystykę, np. średnią, medianę wartość minimalną lub maksymalną}
		}
		\Input{
			$cialo$ \Comment{\raggedleft Ciało eksperymentu, tj. operator wykonania pojedynczego eksperymentu, przyjmujący zestaw parametrów $(P_1, P_2, \ldots)$ ($P_i \in X_i$), a zwracający jakąś miarę jakości działania; dla uproszczenia w algorytmie przyjęto, że zwracaną miarę należy zminimalizować.
			}
		}
		\Output{
			$konfiguracja$ \Comment{\raggedleft Najlepsza konfiguracja znaleziona w procesie eksploracji, tj. $p$ wartości $x_{i, j}$ o różnych indeksach $i$}
		}
		\Start
		\Operator{eksploruj}{$cialo$}
		\Var $konfiguracja = [x_{1, 1}, x_{2, 1}, \ldots, x_{p, 1}]$
		\label{line:explore-init}
		\For{$nawrot \gets 1$ \To $R$} 
		\label{line:nawroty}
			\For{$parametr \gets 1$ \To $p$}
			\label{line:parametry}
				\Var $statystyki = [NULL, NULL, \ldots]$ 
				\Comment{Tablica o rozmiarze $|X_{parametr}|$}
				\For{$ i \gets 1$ \To $|X_{parametr}|$}
					\label{line:wartosci}
					\State $konfiguracja[parametr] \gets x_{parametr, i}$ 
					\Var $wyniki \gets \emptyset$
					\For{$powtorzenie \gets 1$ \To $I$}
					\label{line:powtorzenia}
						\State $wyniki \gets wyniki \cup \{ cialo(konfiguracja) \}$
						\label{line:explore-append}
					\EndFor
					\State $statystyki[i] \gets statystyka(wyniki)$
					\label{line:save-stats}
				\EndFor
				\Var $najlepszaWartosc \gets \argmin_{i} statystyki[i]$
				\label{line:explore-choose}
				\State $konfiguracja[parametr] \gets x_{parametr, najlepszaWartosc}$
				\label{line:explore-use}
			\EndFor
		\EndFor
		\State \Return $konfiguracja$
		\label{line:explore-return}
		\EndOperator
	\end{algorithmic}
	Zapis $a[b]$ oznacza element tablicy $a$ o indeksie $b$
\end{algorithm}


Procedura eksploracji jest zapisana jako operator, ponieważ zwraca pewną wartość (więc nie jest procedurą sensu stricte), która za każdym razem może być inna (więc nie jest funkcją).

Jej działanie zaczyna się od określenia początkowej konfiguracji, czyli zestawu parametrów, w linii \ref{line:explore-init}. 
W tym celu wykorzystywane są pierwsze elementy każdego z zestawów wartości.
Wybór konkretnie tych wartości jest zupełnie arbitralny.

Następnie wykonywane jest $R$ iteracji nazywanych \emph{nawrotami} (linia \ref{line:nawroty}).
Ma to na celu upewnienie się, że dla parametrów wybranych w dalszej części każdego nawrotu wartości wybrane wcześniej faktycznie są jak najlepsze.

W każdym nawrocie badane są wszystkie zestawy wartości parametrów, w kolejności określonej przez parametry procedury (linia \ref{line:parametry}).
Dla każdego z nich sprawdzane są wszystkie elementy (linia \ref{line:wartosci}).
Dzieje się to przez zmianę odpowiedniej wartości w konfiguracji na element badany w danym momencie.

Po takim podstawieniu ciało eksperymentu wykonywane jest $I$ razy (linia \ref{line:powtorzenia}), a wyniki tych wykonań zbierane są do zbioru $wyniki$ (w linii \ref{line:explore-append}).
Po odpowiedniej liczbie powtórzeń, w linii \ref{line:save-stats}, obliczana jest statystyka tego zbioru.
W tym celu wykorzystywany jest parametr procedury $statystyka$.
Jej wartość jest zapisywana tak, żeby można było określić dla której wartości parametru została ona określona.

Kiedy wszystkie wartości parametru zostaną zbadane, wybiera się najlepszą z nich, oceniając je po wartości statystyk (linia \ref{line:explore-choose}).
Wartość ta zostaje zapisana w konfiguracji używanej do badania kolejnego z parametrów (w linii \ref{line:explore-use}).

Po zakończeniu wszystkich nawrotów, w linii \ref{line:explore-return} zwracana jest tak znaleziona konfiguracja.
\end{document}