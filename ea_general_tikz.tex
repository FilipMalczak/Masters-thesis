\documentclass[./FM_mgr.tex]{subfiles}



\begin{document}
	
	\begin{figure}
		\caption{Ogólny schemat działania algorytmów ewolucyjnych
			\label{figure:ea_general}
		}
		\begin{adjustbox}{max width=\textwidth,max height=\textheight}
		\begin{tikzpicture} [
		    auto,
		    partOfDiagram/.style  = {
				draw=black, thick, fill=gray!10, text centered, inner sep=0.25em,font=\bf\small\sffamily , text width=7em
		    	},
		    decision/.style = { diamond, partOfDiagram, aspect=1.5 },
		    block/.style    = { rectangle, partOfDiagram,
		                        rounded corners, minimum height=2em },
		    start/.style    = {
		    		ellipse, partOfDiagram
		    	},
		    stop/.style     = {
		    		, partOfDiagram
		    	},
		    param/.style     = {
			    	chamfered rectangle, partOfDiagram
		    },
		    nullstyle/.style = { scale=0.03 },
		    operator/.style     = {
		    		tape, partOfDiagram
		    },
		    baseline/.style     = { draw, thick, shorten >=2pt },
		    simpleline/.style     = { baseline },
		    arrowline/.style     = { baseline, -> },
		  ]
			 \matrix [column sep=15mm, row sep=10mm] {
			 	& & \node[start] (start) {START}; & &\\
			 	
			 	& &  \node[block] (init) {Zainicjalizuj losową populację}; & & \\
			 	
			 	\node[nullstyle](null0){};  & & 
			 	\node[nullstyle](null1){}; && \\
			 	
			 	
				\node[nullstyle](null2){};& 
			 	\node[stop] (stop) {STOP}; &
			 	\node[decision](cond) {Warunek zatrzymania jest spełniony?}; &
			 	\node[operator](stop_cond) {Warunek zatrzymania}; & \\
			 	
			 	& \node[operator] (crossOp) {Operator krzyżowania}; &
			 	\node[block] (crossover) {Wykonaj krzyżowanie}; &
			 	\node[param] (cp) {Prawd. krzyżowania}; \\
			 	
			 	& \node[operator] (mutOp) {Operator mutacji}; & 
			 	\node[block] (mutation) {Wykonaj mutację}; &
			 	\node[param] (mp) {Prawd. mutacji}; \\
			 	
			 	\node[nullstyle](null3){}; & & 
			 	\node[block] (sel) {Dokonaj selekcji}; &
			 	\node[operator] (selOp) {Operator selekcji}; & \\
			 };
			 
			 \begin{scope} [every path/.style=arrowline]
			   \path (start)      --    (init);
			   \path (init)      --    (cond);
			   \path (cond)      -- node [near start] {tak}   (stop);
			   \path (cond)      -- node [near start] {nie}   (crossover);
			   \path (crossover)      --    (mutation);
			   \path (mutation)      --    (sel);
			   \path (sel) -| (null2) |- (null1);
			   \end{scope}
			   
			   \begin{scope} [every path/.style=simpleline]
			   \path (stop_cond) -- (cond);
			   \path (crossOp) -- (crossover);
			   \path (cp) -- (crossover);
			   \path (mutOp) -- (mutation);
			   \path (mp) -- (mutation);
			   \path (selOp) -- (sel);
			   \path (null3) -- (null0);
			   \end{scope}
		\end{tikzpicture}
		\end{adjustbox}
		Konwencje dotyczące rysowania schematów blokowych zostały wyjaśnione w podsekcji \ref{subsection:conventions}.
	\end{figure}
\end{document}