\documentclass[./FM_mgr.tex]{subfiles}

\begin{document}
\chapter{Przegląd literatury} \label{chapter:literature}

Badania nad algorytmami ewolucyjnymi wykorzystującymi płeć trwają już od pewnego czasu.
Niniejszy rozdział poświęcony jest na przegląd prac naukowych które mają temat podobny do badanego w tej pracy.
Większość pozycji zajmuje się optymalizacja jednokryterialną wykorzystując dość podobne rozwiązania.
Zostaną one opisane na początku tego rozdziału.
Jedna z prac znacząco różni się od innych, ponieważ zamiast analizować jakieś miary skuteczności obliczeń ewolucyjnych, bada ona zachowanie podgrup sztucznie utworzonej populacji.
Zostanie ona poruszona w dalszej kolejności.
Ostatnie dwie opisane prace traktują o optymalizacji wielokryterialnej i stosują podobne podejście.


W pracy \cite{GGA} przedstawiono heurystykę GGA.
Została ona wykorzystana do rozwiązania problemu partycjonowania grafu.
GGA czyli algorytm genetyczny oparty o płeć (\emph{ang. gender-based genetic algorithm}) to algorytm genetyczny w którym każdy osobnik w populacji miał cechę określającą jego płeć.
Była ona wykorzystywana jedynie podczas krzyżowania.
Algorytm GGA losował każdego z rodziców z osobnego podzbioru populacji, zawierającego osobniki jednej płci.

Pozycja \cite{SexualGA} opisuje kolejną heurystykę SexualGA.
W tej pracy nie badano wyników heurystyki, a ,,ciśnienie selekcyjne'' (\emph{ang. selection pressure}).
Termin ten oznacza pewną miarę prawdopodobieństwa tego, że pojedynczy osobnik zostanie rodzicem w danym pokoleniu.
Sama heurystyka SexualGA nie wymagała użycia cechy osobnika mówiącej o jego płci.
W zamian wykorzystywała operator selekcji do wyboru rodziców zamiast do selekcji naturalnej.
Ponadto, każdy z rodziców był wybierany za pomocą innego operatora.

W pracy \cite{ansotegui} wykorzystano zmodyfikowany algorytm ewolucyjny do dostrajania innych heurystyk.
Przedstawiona tutaj heurysyka nazywała się tak samo jak opisana w pracy \cite{GGA} - GGA, rozwijane i tłumaczone w ten sam sposób.
Nadawała ona każdemu z osobników cechę płci.
Podczas krzyżowania wybierani byli rodzice różniący się płcią.
Jeden z nich był wybierany elitystycznie, a drugi losowo.
Dodatkowo, zamiast selekcji naturalnej zasymulowano tutaj starzenie się osobników.
Jednostki które były w populacji więcej niż określoną liczbę pokoleń były z niej odrzucane.

Praca \cite{sanchez} przedstawia badania nad nienazwanym rozwiązaniem w którym osobnicy mają cechę płci.
Rodzice jednej z tych płci są wybierani za pomocą operatora selekcji. 
Drugie osobniki w każdej parze sa wybierane tym samym operatorem, jednak ich ocena jest obliczana jako pewna funkcja oceny używanej przy płci pierwszej.
Funkcja ta bierze pod uwagę wiek osobnika oraz pewną miarę tego jak dużą szansę drugi z rodziców ma na poprawienie jakości pierwszego rodzica poprzez krzyżowanie.
Miało to symulować szereg czynników jakie biorą pod uwagę samce wybierając swoje partnerki.
Podejście to zostało zbadane dla problemu komiwojażera oraz dwóch problemach przeznaczonych tylko do testów, opisanych funkcjami analitycznymi.

Pozycja \cite{simulating} różni się znacząco od wyżej opisanych.
Nie badano w niej efektywności w rozwiązywaniu problemu przy pomocy heurystyki, a raczej ewolucję pewnego systemu w którym pewne grupy osobników ze sobą współżyją.
Grupy te powstały przez podział populacji na 4 płcie: męską, żeńską, samo-zapładniającą (\emph{ang. self-fertilizing}) oraz obojniacką (\emph{ang. hermaphrodite}).
Osobnicy każdej z płci mogli dobierać się w pary jedynie z osobnikami wybranych płci.
Osobniki trzeciej z płci mogły dobierać się w pary z każdym, łącznie z osobnikami swojej płci.
Przedstawiciele pozostałych płci mogli wydawać potomstwo z osobnikiem dowolnej płci niż jego.
Ponadto, wprowadzono mechanizm mający zapobiec sytuacjom w których któraś z płci jest źle reprezentowana.
Losowe osobniki które przez pewną liczbę pokoleń nie wydawały potomstwa miały szansę z pewnym prawdopodobieństwem zmienić swoją płeć.

Ponadto, znaleziono dwie prace: \cite{msga} oraz \cite{allenson}.
Obie opisywały bardzo podobne podejście do optymalizacji wielokryterialnej.
Pierwsza z nich opisywała heurystykę MSGA (\emph{ang. Multisexual Genetic Algorithm}, czyli wielopłciowy algorytm genetyczny).
Przypisywała ona każdemu osobnikowi cechę płci.
Możliwych wartości tej cechy było tyle ile kryteriów w rozwiązywanym problemie.
Każda z nich miała do siebie przypisane jedną z wielu funkcji oceny.
Osobniki z tą wartością płci były oceniane przy pomocy tej wybranej funkcji.
Poza tym, podobnie jak w GGA wymuszano różne płci rodziców, wybierając ich losowo.
W pracy \cite{msga} zbadano działanie MSGA na dwóch testowych parach funkcji analitycznych.

Praca \cite{allenson} opisuje bardzo podobne podejście, nie nadając mu nazwy.
Jedyna różnica względem MSGA to zastosowanie atraktorów płciowych (\emph{ang. sexual attractors}).
Są to czynniki wpływające na prawdopodobieństwo dobrania się w parę z innym osobnikiem.
W opisywanej pracy zasymulowano to poprzez wykorzystanie dodatkowej cechy atrakcyjności, która dla rozwiązań początkowych jest tożsama z rangą.
Zbadano różne podejścia do dziedziczenia cechy, decydując się ostatecznie na obliczanie wypadkowej atrakcyjności rodziców.
Taka heurystyka została zastosowana do planowania ścieżki dla rurociągu, oceniając rozwiązanie zarówno po jego długości jak i po szkodliwym wpływie na środowisko.

\end{document}