\documentclass[./FM_mgr.tex]{subfiles}

\begin{document}
\section{Przegląd literatury} \label{chapter:literature}

Badania nad algorytmami ewolucyjnymi wykorzystującymi płeć trwają już od pewnego czasu, chociaż lista tych prac nie jest zbyt długa.
Większość pozycji zajmuje się optymalizacja jednokryterialną wykorzystując dość podobne do siebie rozwiązania.
Jedna z prac znacząco różni się od innych, ponieważ zamiast analizować jakieś miary skuteczności obliczeń ewolucyjnych, bada ona zachowanie podgrup sztucznie utworzonej populacji.
Ostatnie dwie opisane tu prace traktują o optymalizacji wielokryterialnej i opisują zbliżone do siebie podejścia.


W pracy \cite{GGA} przedstawiono metaheurystykę \emph{GGA}.
Została ona wykorzystana do rozwiązania problemu partycjonowania grafu.
\emph{GGA} czyli algorytm genetyczny oparty o płeć (\emph{ang. gender-based genetic algorithm}) to algorytm genetyczny, w którym każdy osobnik w populacji ma przypisaną cechę określającą jego płeć.
Jest ona wykorzystywana jedynie podczas krzyżowania.
W \emph{GGA} każdy z rodziców losowany jest z osobnego podzbioru populacji, który zawiera osobniki jednej płci.

Artykuł \cite{SexualGA} opisuje kolejną metaheurystykę nazywaną \emph{SexualGA}.
W tej pracy nie badano wyników metaheurystyki, a ,,ciśnienie selekcyjne'' (\emph{ang. selection pressure}).
Termin ten oznacza pewną miarę prawdopodobieństwa tego, że pojedynczy osobnik zostanie rodzicem w danym pokoleniu.
Sama heurystyka \emph{SexualGA} nie wymagała użycia cechy osobnika mówiącej o jego płci.
W zamian wykorzystywała operator selekcji do wyboru rodziców zamiast do selekcji naturalnej.
Ponadto, każdy z rodziców był wybierany za pomocą osobnego operatora.

W pracy \cite{ansotegui} wykorzystano zmodyfikowany algorytm ewolucyjny do dostrajania innych heurystyk.
Przedstawiona tutaj metaheurystyka nazywała się tak samo jak opisana w pracy \cite{GGA} - \emph{GGA}, ze skrótem rozwijanym i tłumaczonym w ten sam sposób.
Nadawała ona każdemu z osobników cechę płci.
Podczas krzyżowania wybierani byli rodzice różniący się płcią.
Jeden z nich był wybierany elitystycznie, a drugi losowo.
Dodatkowo, zamiast selekcji naturalnej zasymulowano tutaj starzenie się osobników.
Jednostki, które były w populacji przez więcej niż określoną liczbę pokoleń były z niej odrzucane.

Praca \cite{sanchez} przedstawia badania nad nienazwanym rozwiązaniem, w którym genotyp osobnika zawiera cechę płci.
Rodzice jednej z tych płci są wybierani za pomocą operatora selekcji. 
Drugie osobniki w każdej parze są wybierane tym samym operatorem, jednak ich ocena jest obliczana jako pewna funkcja optymalizowanego kryterium.
Funkcja ta bierze pod uwagę wiek osobnika oraz pewną miarę tego, jak dużą szansę drugi z rodziców ma na poprawienie jakości pierwszego rodzica poprzez krzyżowanie.
Miało to symulować szereg czynników, jakie biorą pod uwagę samce wybierając swoje partnerki.
Podejście to zostało zbadane dla problemu komiwojażera oraz dwóch problemów przeznaczonych tylko do testów, opisanych funkcjami analitycznymi.

Praca \cite{simulating} różni się znacząco od wyżej opisanych.
Nie badano w niej efektywności w rozwiązywaniu problemu za pomocą heurystyki, a raczej ewolucję pewnego systemu, w którym pewne grupy osobników ze sobą współżyją.
Grupy te powstały przez podział populacji na 4 płcie: męską, żeńską, samo-zapładniającą (\emph{ang. self-fertilizing}) oraz obojniacką (\emph{ang. hermaphrodite}).
Osobnicy każdej z płci mogli dobierać się w pary jedynie z osobnikami wybranych płci.
Osobniki trzeciej z płci mogły dobierać się w pary z każdym, łącznie z osobnikami swojej płci.
Przedstawiciele pozostałych płci mogli wydawać potomstwo z osobnikiem dowolnej płci niż jego.
Ponadto, wprowadzono mechanizm mający zapobiec sytuacjom, w których któraś z płci jest źle reprezentowana.
Losowe osobniki, które przez pewną liczbę pokoleń nie wydawały potomstwa, miały szansę z pewnym prawdopodobieństwem zmienić swoją płeć.

Oprócz wyżej omówionych, znaleziono dwie prace: \cite{msga} oraz \cite{allenson}.
Obie opisują bardzo podobne podejście do optymalizacji wielokryterialnej.
Pierwsza z nich opisuje heurystykę \emph{MSGA} (\emph{ang. Multisexual Genetic Algorithm}, czyli wielopłciowy algorytm genetyczny).
Przypisuje ona każdemu osobnikowi cechę płci.
Możliwych wartości tej cechy było tyle ile kryteriów rozważano w rozwiązywanym problemie.
Każda z płci miała do siebie przypisane jedną z wielu funkcji oceny.
Osobniki tej płci były oceniane za pomocą przypisanej jej funkcji.
Poza tym, podobnie jak w \emph{GGA}, wymuszano różne płci rodziców, wybierając ich losowo.
W pracy \cite{msga} zbadano działanie \emph{MSGA} na dwóch testowych problemach, z których każdy opisany był parą funkcji analitycznych.

Praca \cite{allenson} opisuje bardzo podobne podejście do opisanego wyżeh.
Nie nadano mu jednak nazwy.
Jedyna różnica względem \emph{MSGA} to zastosowanie atraktorów płciowych (\emph{ang. sexual attractors}).
Są to czynniki wpływające na prawdopodobieństwo dobrania się w parę z innym osobnikiem.
W opisywanej pracy zasymulowano to poprzez wykorzystanie dodatkowej cechy atrakcyjności, dołączanej do genotypu osobnika, która dla rozwiązań początkowych jest tożsama z rangą.
Zbadano różne podejścia do dziedziczenia tej cechy, decydując się ostatecznie na obliczanie wypadkowej atrakcyjności rodziców.
Taka heurystyka została zastosowana do planowania ścieżki dla rurociągu, oceniając rozwiązanie zarówno po jego długości, jak i po szkodliwym wpływie na środowisko.

\end{document}