\documentclass[./FM_mgr.tex]{subfiles}

\begin{document}
\chapter{Wnioski i spostrzeżenia}

Spośród dwóch zbadanych problemow w jednym heurystyka DSEA dała najlepsze rezultaty, a w obu lepsze niż pozostałe heurystyki inne niż klasyczny algorytm ewolucyjny.
Można z tego wnioskować, że dla wybranych klas problemow dodanie dodatkowego operatora selekcji przynosi pozytywne skutki.

W ogólności, operator selekcji płciowej działa lepiej, jeżeli korzysta z tego samego operatora wyboru co operator selekcji naturalnej.

Algorytm DSEA jest na tyle ogólny, zeby można było skutecznie zbadać i porównać różne istniejące rozwiązania przy jego użyciu.

Haremowy operator selekcji płciowej sprawdza się gorzej niż taki uogólniony operator.
Można to stwierdzić jedynie w sytuacji, w ktorej operator wyboru partnerów jest losowy.
Liczba osobników alfa i beta wydaje się nie mieć większego wpływu na efektywność heurystyki.




\end{document}