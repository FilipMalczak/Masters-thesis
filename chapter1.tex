\documentclass[./FM_mgr.tex]{subfiles}

\begin{document}
\chapter*{Wprowadzenie}
Ewolucja to proces zachodzący w naturze odpowiedzialny za dopasowywanie się osobników danego gatunku do środowiska w jakim żyją. 
Podstawą tego procesu jest zasada przetrwania lepiej przystosowanych osobników oraz zjawiska dziedziczenia i mutacji.

Algorytmy ewolucyjne to rodzina heurystyk naśladujących proces ewolucji w celu optymalizacji \cite{davis1991handbook}. 
Pojedynczy element przestrzeni rozwiązań jest w nich nazywany osobnikiem. 
Osobniki możemy między sobą porównywać pod względem wartości optymalizowanej funkcji dla nich, a relacja mniejszości (dla problemów minimalizacji) lub większości (dla problemów maksymalizacji) reprezentuje relację bycia lepiej przystosowanym do środowiska. 
Ponadto, na osobnikach określone są operatory mutacji i krzyżowania, które mają na celu symulację odpowiednich zjawisk występujących w przyrodzie. 
Heurystyka polega na wielokrotnym przetworzeniu populacji (czyli zbioru osobników) poprzez zastosowanie każdego z operatorów z pewnym prawdopodobieństwem. 
W każdym kroku (nazywanym w tym przypadku pokoleniem) do dotychczasowej populacji dołączane są wyniki działania tych operatorów, a następnie wybierana jest nowa populacja, używana w kolejnym kroku algorytmu ewolucyjnego. 
Aby odwzorować zasadę przetrwania najlepiej dopasowanych osobników, do kolejnej populacji wybierane są z wyższym prawdopodobieństwem osobniki lepiej przystosowane.

W naturze rozmnażanie się osobników wielu gatunków jest ściśle związane ze zjawiskiem podziału gatunku na płcie. 
W dostępnej literaturze dotyczącej tematu algorytmów ewolucyjnych rzadko można znaleźć prace, w których uwzględnia się ten aspekt procesu ewolucji. 
Powodem tego jest raczej chęć uproszczenia działania samej heurystyki niż lepsza jakość wyników uzyskiwanych z pominięciem tego aspektu (\cite{GGA}, \cite{SexualGA}). 

Celem niniejszej pracy jest opracowanie algorytmu ewolucyjnego uwzględniającego płeć osobników i zbadanie jego skuteczności na wybranych problemach, oraz porównanie go z klasycznym algorytmem ewolucyjnym i wybranymi rozwiązaniami znanymi z literatury.

Ponadto autor opracował opis formalny algorytmu ewolucyjnego uwzględniającego płeć.
Na jego podstawie powstała biblioteka programistyczna, wykorzystana w celu porównania jakości optymalizacji wybranych metaheurystyk.

\end{document}