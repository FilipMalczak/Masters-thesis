\documentclass[./FM_mgr.tex]{subfiles}

\begin{document}
\chapter{Przeprowadzone badania} \label{chapter:experiments}

Niniejsza praca ma na celu zbadanie i porównanie różnych podejść do zagadnienia płci wykorzystywanych w algorytmach ewolucyjnych.
Aby to osiągnąć zaimplementowano metody opisane w poprzednim rozdziale i przeprowadzono zestaw eksperymentów, których wyniki zostały porównane.

Ten rozdział zaczyna się od prezentacji wybranych problemów optymalizacji, na których przeprowadzone zostały badania. Kolejny podrozdział zawiera opis implementacji wykorzystanej w badaniach. Całość zamknięta jest podsumowaniem ich przebiegu i przedstawieniem wyników.

\section{Wybrane problemy testowe}

Aby zbadać jakość działania wybranych metod zaimplementowano i podjęto próbę rozwiązania dwóch problemów optymalizacji. 
Były to problem komiwojażera (\emph{ang. travelling salesman problem}, w skrócie \emph{TSP}) oraz binarny problem plecakowy.

Problem komiwojażera to zadanie znalezienia minimalnego cyklu Hamiltona w pełnym grafie ważonym. 
Cykl Hamiltona to cykl, w którym każdy wierzchołek jest odwiedzany tylko raz (z pominięciem pierwszego wierzchołka, który jest taki sam jak ostatni). 
Graf pełny, to taki, w którym między dowolnymi dwoma wierzchołkami istnieje krawędź, a graf ważony to taki, w którym krawędzie opisane są pewnymi wartościami, nazywanymi wagami. 
Minimalnym cyklem nazywamy taki, dla którego suma wag krawędzi jest najmniejsza.

Problem ten można przełożyć na rzeczywistą sytuację, w której mamy zbiór punktów na mapie (np. miast, reprezentowanych przez wierzchołki grafu) leżących w pewnych odległościach od siebie\footnote{
	Geograficzne odległości między punktami to tylko jedna z możliwych interpretacji.
	Zamiast tego między punktami może być określony czas przejazdu, jego koszt, itd.
	W ogólności, znane są pewne wartości związane z każdymi dwoma punktami, które oznaczają tym gorszą sytuację z punktu widzenia osoby rozwiązującej problem, im są wyższe.
} (które są reprezentowane przez wagi krawędzi). 
Rozwiązanie problemu komiwojażera sprowadza się do minimalizacji długości drogi przechodzącej przez wszystkie punkty. 
Wynik optymalizacji jest cenny w rzeczywistych zastosowaniach, ponieważ pozwala m.in. zaoszczędzić zasoby w firmach zajmujących się transportem.

Binarny problem plecakowy (nazywany czasem dyskretnym problemem plecakowym) to zadanie optymalizacji, w którym ze zbioru $n$ par $\{ (k_1, w_1),  (k_2, w_2), \ldots, (k_n, w_n)\}$ należy wybrać podzbiór, dla którego suma pierwszych elementów z pary będzie jak największa, a suma drugich elementów nie przekroczy zadanej wartości $W$.
Problem ten przekłada się na sytuację, w której mamy pewien pojemnik (popularnie nazywany plecakiem, skąd bierze się nazwa problemu) o określonej pojemności $W$ i zestaw przedmiotów, które możemy do niego włożyć. 
Każdy przedmiot jest opisany parą $(k, w)$, gdzie $k$ reprezentuje jego wartość, a $w$ jego rozmiar, czy też objętość. 
Rozwiązaniem problemu jest taki zestaw przedmiotów jaki zmieści się w plecaku i będzie miał jak największą wartość. 

Ogólna wersja problemu plecakowego polega na przypisaniu każdemu z przedmiotów (par) liczby naturalnej, określającej ile jego egzemplarzy wkładamy do plecaka, jednak w ramach niniejszej pracy zdecydowano się na binarną wariację tego problemu, która dopuszcza wykorzystanie co najwyżej jednego egzemplarza każdego obiektu.

Opisane problemy zostały wybrane, ponieważ reprezentują dwie klasy zadań powszechnie rozwiązywane za pomocą algorytmów ewolucyjnych. 
Binarny problem plecakowy to popularne zadanie w którym rozwiązania są reprezentowane jako wektory bitów, a problem komiwojażera to znane zagadnienie, powszechnie wykorzystywane do badania jakości działania heurystyk. 
Ponadto, problem plecakowy to zadanie optymalizacji z ograniczeniami, których nie da się łatwo zachować w ramach operatorów genetycznych (w przeciwieństwie np. do problemu komiwojażera, w którym można zastosować operatory, które gwarantują, że zwrócą rozwiązania nie przekraczające ograniczeń, jeżeli osobniki wejściowe ich nie przekraczały).

\section{Implementacja}

Aby zbadać skuteczność wybranych metod potrzebna była ich implementacja.
Metaheurystyka DSEA pozwala na zasymulowanie pozostałych wybranych metod, więc to ją właśnie zaimplementowano.
W wyniku tego powstała biblioteka programistyczna w języku Groovy 2.3. Jest to język działający w oparciu o maszynę wirtualną Java, a biblioteki w nim pisane są automatycznie dostępne dla wszystkich języków działających w tej technologii, łącznie z samym językiem Java.
Poza samym algorytmem zaimplementowano również reprezentacje i operatory genetyczne dla dwóch opisanych wyżej problemów.
Ponadto, powstał zestaw operatorów selekcji płciowej, naturalnej oraz operatorów wyboru opisanych w kolejnych podrozdziałach.

Dodatkowo, prócz biblioteki powstał zestaw skryptów w języku Groovy pozwalających na utrwalanie wyników i powracanie do przerwanych badań, oraz opisujących wykonanie i analizę wyników wszystkich eksperymentów potrzebnych w tej pracy.

Szczegółowy opis implementacji wykonanej na potrzeby badań znajduje się w dodatku \ref{appendix:impl_det}. Przedstawiono tam sposób działania ruletkowego i turniejowego operatora wyboru oraz funkcję oceny i operatory genetyczne dla obu badanych problemów.

W niniejszej pracy wykorzystano operator losowy, ruletkowy ważony rangą, oraz operatory turniejowe o rozmiarze 2 i 3. 
Wykorzystywane są one wraz z operatorem selekcji naturalnej opisanym algorytmem \ref{algorithm:natSel_choose}.
Ponadto, użyto operatorów selekcji płciowej opisanych za pomocą algorytmów \ref{algorithm:common_genSel} i \ref{algorithm:harem}.
Są one kolejno nazywane uogólnionym i haremowym operatorem selekcji płciowej.
Niezależnie od eksperymentu wykorzystano kryterium zatrzymania przerywające działanie algorytmu po określonej liczbie pokoleń. 

Tabela \ref{table:op_symbols} przedstawia oznaczenia różnych elementów wraz z ich opisem.
Przykładowo, zapis \opName{stdGenSel}($\top$, \opName{R}, \opName{TS}(2)) oznacza standardowy operator selekcji naturalnej, parametryzowany przez $\param{plecMaZnaczenie}=\top$, wykorzystujący operator losowy jako $\param{opWyboru1}$ i operator turniejowy (z rozmiarem turnieju równym 2) jako $\param{opWyboru2}$.

\begin{table}
	\caption{Oznaczenia operatorów \label{table:op_symbols}}
	\begin{tabularx}{\linewidth}{|c|X|}
		\hline
		\textbf{Oznaczenie} & \textbf{Znaczenie} \\
		\hline \hline
		\opName{R} & Losowy operator wyboru $random(X)$ \\ 
		\hline
		\opName{RS} & Ruletkowy operator wyboru \\ 
		\hline
		\opName{TS}($n$) & Turniejowy operator wyboru o rozmiarze turnieju n \\ 
		\hline
		\opName{natSel}(X) & Operator selekcji naturalnej opisany algorytmem \ref{algorithm:natSel_choose}, korzystający z operatora wyboru X \\
		\hline
		\opName{stdGenSel}(p, W1, W2) & Operator selekcji płciowej używany do realizacji dotychczasowych rozwiązań, opisany algorytmem \ref{algorithm:common_genSel}, parametryzowany przez $\param{plecMaZnaczenie}=p$, $\param{opWyboru1}=W1$ oraz $\param{opWyboru2}=W2$ \\
		\hline
		\opName{harem}(a, b, WA, WB, WP) & Proponowany haremowy operator selekcji płciowej, opisany algorytmem \ref{algorithm:harem} (bez zmian opisanych algorytmem \ref{algorithm:harem_fix}), wykorzystujący $\param{liczbaAlfa}=a$, $\param{wspBeta}=b$, $\param{opWyboruAlfa}=WA$, $\param{opWyboruBeta}=WB$, $\param{opWyboruPartnerow}=WP$. \\
		\hline
	\end{tabularx}
\end{table}

W pracy przyjęto co najwyżej 2 płcie osobników. Jeżeli płeć była tylko jedna, to $\important{G} = \{ \odot \}$, w przeciwnym wypadku $\important{G} = \{ \male, \female \}$. 
Symbol $\odot$ to umowne oznaczenie na wartość cechy płci, jeżeli może ona przyjmować tylko jedną wartość.



\end{document}